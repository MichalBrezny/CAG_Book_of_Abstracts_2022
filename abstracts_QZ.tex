%start of conference contribution
\abstract
% Title 
{In-situ rock surface strain changes determination: use of resistance strain gauges} 
% EndOfTitle
%short author -- toc 
{Racek et al.} 
%End short author -- toc
% Author(s) 
{O. Racek\textsuperscript{1,2}*, Jan Blahůt\textsuperscript{1}, Jan Balek\textsuperscript{1}} 
% EndOfAuthor(s)
{\POtag} 
%Tag, can be: empty, \KLtag (keynote lecture), \IStag (invited speaker), \CTtag (contributed talk) or \ITtag (invited talk)
% Affiliation(s)
{
\textsuperscript{1}Department of engineering geology, Institute of rock structure and mechanics, Czech academy of sciences, Praha, Czech Republic\\
\textsuperscript{2}Department of physical geography and geoecology, Faculty of Science, Charles University, Prague, Czech Republic
}
%} % EndOfAffiliation(s)
{racek@irsm.cas.cz}  %e-mail
%keywords
{rock mass, strain, strain gauges, rock mass dynamic}
%EndOfKeywords
%abstract content
{This poster describes an ongoing GAMA2 project, is dealing with the method for in-situ rock mass strain measurement. Resistance tensiometry (Koshinuma et al., 1972) is widely used for strain measurements in the field of automotive, civil engineering or mining industries (Rocha et al., 1969). However, in such an application, strain gauges are measuring relatively homogeneous material, such as steel or concrete, usually in controlled laboratory conditions. Our ongoing project aims to modify this method for direct rock mass strain measurement. In comparison with traditional geotechnical strain monitoring methods (flat jack, strain tensiometer) is resistance tensiometry cheaper alternative, with potentially higher precision (Murray et al, 1992).

In the natural environment, the results of tensiometry are affected by meteorological variables mainly by temperature. Our goals are 1) to compile measuring instrumentation that can withstand natural conditions a can be placed directly within rock slopes, and 2) to propose a methodology for the elimination of meteorological factors influence. 

For that purpose new test site \enquote{Na požárech} 30 km from Prague was established. On this site, a weather station, thermocouples, and in-depth rock mass temperature monitoring were established. Recently strain gauges in different geometries were placed on rock slope surfaces. We are using simple quarter bridges or half-bridges, or more complex full bridges/tensiometric roses. By these, we can measure the one- or two-dimensional strain of rock slope surface. By complex full bridges/roses we should be able to determine the direction of rock mass surface strain. Simple one-dimensional strain gauges measure the dynamic of microcrack and undamaged rock mass. To validate the results of tensiometry, conventional resistance and vibrating wire crack meters are installed nearby the resistance crack gauges. 

In this semi-controlled environment, we can observe the influence of meteorological variables on results. The project’s final phase is to propose a methodology for in-situ rock mass tensiometry data processing and propose it for public use, together with proof measuring instrumentation. 
}
%EndOfAbstractContent
%references
{Koshinuma, M., Nakamura, A., Seimiya, T., \& Sasaki, T. (1972). Strain Gauge Electromicrobalance for Surface Tension Measurement. Bulletin of the Chemical Society of Japan, 45(2), 344-347.
	
Rocha, Manuel, and Arnaldo Silvério. "A new method for the complete determination of the state of stress in rock masses." Geotechnique 19.1 (1969): 116-132.

Murray, William M., and William R. Miller. The bonded electrical resistance strain gage: an introduction. Oxford university press, 1992.
}
%EndOfReferences
%end of conference contribution
%---------------------------------------------------------------------------------------------------

%start of abstract
\abstract
{Flash flood simulation in the urbanised catchment: A case study of Bratislava–Karlova Ves} % Title
{Rusinko and Horáčková} %short author -- toc
{Adam Rusinko\textsuperscript{1}*, Šárka Horáčková\textsuperscript{2}} % Author(s)
{\POtag} % Tag, can be: empty, \KLtag (keynote lecture), \IStag (invited speaker), \CTtag (contributed talk) or \ITtag (invited talk)
{
	\textsuperscript{1}Faculty of Natural Sciences, Comenius University, Bratislava, Slovakia\\
	\textsuperscript{2}Institute of Geography, Slovak Academy of Sciences, Bratislava, Slovakia
} % Affiliation(s)
{adam.rusinko@uniba.sk}  %e-mail
{Čierny potok stream, flash flood, GRASS GIS, LiDAR, CORINE Land Cover, Bratislava}%keywords
{Flash floods are a dangerous phenomenon that usually affects small drainage basins. They are primarily initiated in the upper parts of the slopes, but their injuring effects are manifested mostly in residential areas, where underground channelized streams disappeared from the surface. Therefore, there are not precise data about stream water levels available and only using surface runoff modelling is possible to simulate what happened during flash floods. Karlova Ves (Bratislava city District), formerly a small viniculture village, was threatened by floods (most probably including those of pluvial type) in the history. In this paper, we simulate surface runoff of a flash flood that occurred in the summer of 2014 in the catchment of Čierny potok using r.sim.water module in GRASS GIS. The flood in August 2014 was reported as with the highest rainfall per hour \textasciitilde40mm during the time of local meteorological measurements. Current orthophotomap was used to classify CLC land cover classes, which were assigned the value of the Manning’s roughness coefficient and infiltration rate. The topography was expressed by DTM from high resolution LiDAR data. Our preliminary results indicate that land cover and land use are the essential factors influencing flash floods initiation, although the main driver in lower infiltration and change in flow direction is caused by urbanisation and high proportion of impervious areas. The simulation using r.sim.water module showed that during 60-minute extreme rainfall (40mm/hr) a surface runoff can reach a water depth up to 2 meters in terrain depressions by maximum discharge of 25 cubic meters per second. Natural urban areas revitalisation with increasing the vegetation cover in the areas prone to water flow and accumulation during higher rainfalls helps to prevent the damage caused by floods.

\vspace{0.5em}
\noindent
\textbf{Acknowledgements:}
\textit{The research was financially supported by the Science Grant Agency (VEGA) of the Ministry of Education, Science, Research and Sport of the Slovak Republic and the Slovak Academy of Sciences [Grant Nr. 2/005/21]. Thanks to citizens of Karlova Ves for photographs of the flash flood from August 2014.}}%abstract
{Alizadehtazi B, DiGiovanni K, Foti R, Morin T, Shetty N H, Montalto F A, Sgurian P L (2016) Comparison of Observed Infiltration Rates of Different Permeable Urban Surfaces Using a Cornell Sprinkle Infiltrometer. Journal of Hydrologic Engineering 21(7): 06016003-1.

Hofierka J, Knutová M (2015) Simulating aspects of a flash flood using the Monte Carlo method and GRASS GIS: a case study of the Malá Svinka Basin (Slovakia). Open Geosciences 7(1): 118-125.

Lapin M, Mikulová K, Pecho J, Šťastný P (2019) Súčasná klimatická charakteristika MČ Bratislava – Karlova Ves a popis scenárov a dopadov zmeny klímy na riešené územie. SHMÚ, Bratislava.

Oťaheľ J, Feranec J, Kopecká M, Falťan V (2017) Modifikácia metódy CORINE Land Cover a legenda pre identifikáciu a zaznamenávanie tried krajinnej pokrývky v mierke 1:10 000 na báze príkladových štúdií z územia Slovenska. Geografický časopis 69(3): 189-224.

Prokešová R, Horáčková Š, Snopková Z (2022) Surface runoff response to long-term land use changes: Spatial rearrangement of runoff-generating areas reveals a shift in flash flood drivers. Science of the Total Environment 815: 151591.

Hydrology in GRASS GIS: A tutorial on hydrological modeling and simulation in GRASS GIS, \url{https://baharmon.github.io/hydrology-in-grass} cited in January 15, 2022

Letecké laserové skenovanie a DMR 5.0, \url{https://www.geoportal.sk/sk/zbgis/lls-dmr} cited in October 12, 2021
}%references
%end of abstract
%---------------------------------------------------------------------------------------------------

\abstract
{Fluvial habitat assessment using high-resolution 3D models} % Title
{Rusnák et al.} %short author -- toc
{Miloš Rusnák\textsuperscript{1}*, Peter Mihálik\textsuperscript{2}, Ján Sládek\textsuperscript{3}} % Author(s)
{\TLtag} % Tag, can be: empty, \KLtag (keynote lecture), \IStag (invited speaker), \CTtag (contributed talk) or \ITtag (invited talk)
{
\textsuperscript{1}Institute of Geography SAS, Bratislava, Slovakia\\
\textsuperscript{2}Faculty of Natural Sciences, Comenius University, Bratislava, Slovakia\\
\textsuperscript{3}Institute of Geography SAS, Bratislava; GEOTECH Bratislava, Slovakia
} % Affiliation(s)
{geogmilo@savba.sk}  %e-mail
{UAV, river channel, bathymetry, classification, fluvial habitats}%keywords
{The paper aims to create a detailed automatic classification of the river landscape habitats based on high-resolution data obtained by drones. We apply automatic classification of the floodplain landscape structure and in-channel physical habitats. As part of automated classification, we test the accuracy of pixel-based and object-oriented supervised classification, using data sets composed of traditional spectral characteristics (RGB) and the geometric properties of the point cloud. The channel bathymetry was identified by modelling the relationship between spectral parameters of the image and field measured water depth or by the correction of the photogrammetrically generated bathymetry model by the refraction coefficient. The accuracy of the automatic classification was evaluated based on KAPPA indices using the validation layer, and the RMSE error was used for bathymetric models. In total, we classify nine main classes of land cover: water; low vegetation (less than 0.5 m); medium vegetation (0.5 - 3 m); high vegetation (more than 3 m); gravel bar; flood facies; bedrock and LWD. The submerged physical morphology was divided into four classes based on water depth. For object-oriented classification on the data layer using only the colour spectrum RGB, the accuracy of vegetation classification was 98.1 \%. The water depth in the riverbed was identified based on a bathymetric model with a determination coefficient of 0.8161 and an RMSE error of 0.2419 m. The physical habitat in the river includes different continuums (grain size, water depth, topographic elevation and flow velocity) as a main physical river habitat parameter constituted by the flow regime (hydrology and hydraulics) and the physical template (fluvial sedimentology and geomorphology). Physical habitat parameters will be extracted from a detailed bathymetry model, reconstructing the channel bed structure, gravel bars and identifying habitat patterns essential for maintaining fish assemblage biodiversity. From hydraulic modelling, we used flow velocity. For final class mapping, we used velocity data, water depth and morphology unit classification (pool, glide, run, riffle).
	
\vspace{0.5em}
\noindent
\textbf{Acknowledgements:}\\
\textit{The research was supported by the Scientific Grant Agency VEGA, number 2/0086/21.}
}%abstract
{}%references

%---------------------------------------------------------------------------------------------------
%start of conference contribution
\abstract
% Title 
{Can abandoned underground mines be equal in their diversity to the cave system?} 
% EndOfTitle
%short author -- toc 
{Schuchová and Lenart} 
%End short author -- toc
% Author(s) 
{Kristýna Schuchová\textsuperscript{1}*, Jan Lenart\textsuperscript{1}} 
% EndOfAuthor(s)
{\TLtag} 
%Tag, can be: empty, \KLtag (keynote lecture), \IStag (invited speaker), \CTtag (contributed talk) or \ITtag (invited talk)
% Affiliation(s)
{
	\textsuperscript{1}Faculty of Science, University of Ostrava, Ostrava, Czech Republic
}
%} % EndOfAffiliation(s)
{kristyna.schuchova@osu.cz}  %e-mail
%keywords
{abandoned mines, geomorphology, shallow mines, database, geodiversity}
%EndOfKeywords
%abstract content
{Unlike caves, abandoned underground mines have been rather out of side scientists interest (Lenart, 216). Compared to the other subterranean geosystems, the geomorphology of abandoned underground mines remain low investigated (Hill and Forti, 1997). The complex spectrum of geomorphic forms was never described from abandoned underground mines. The works about abandoned mines are mostly focused on mature rock, hydrothermal veins composition analyses or discoveries of new minerals. From the risk assessment point of view, abandoned underground mines present threat for human health and environmental risk (Wichert, 2020). However, in recent years, these mines play more important role in nature conservation as habitat for bats, preservation of cultural heritage or tourist management. This research brings the results of one of the first complex geomorphological investigations of abandoned underground slate mines from Nízký Jeseník Upland. Various types of subterranean geomorphic forms were described from 114 mines. Forms, evaluated according to their type, scale, frequency and fragility, are independent on each other, or embedded. The range of described geomorphic forms together and measured crack shifts point to rich geodiversity and high dynamics within the abandoned mines.
}
%EndOfAbstractContent
%references
{Hill CA, Forti P (1997) Cave Minerals of the World. National Speleological Society, pp. 463. 

Lenart J (2016) Nízký Jeseník – Highland with Abandoned Deep Mines. In: Pánek T, Hradecký J, (eds) Landscapes and Landforms of the Czech Republic. World Geomorphological Landscapes. Springer, Cham. 305¬317. \doi{10.1007/978-3-319-27537-6\textunderscore24}

Wichert J (2020) Slate as Dimension Stone: Origin, Standarts, Properties, Mining and Deposits. Springer Mineralogy, Springer Nature Switzerland AG 9, 492 pp. \doi{10.1007/978-3-030-35667-5}
}
%EndOfReferences
%end of conference contribution
%---------------------------------------------------------------------------------------------------
%start of conference contribution
\abstract
% Title 
{From repeated measurements to early warning systems: DAEMON resistivity system in landslide monitoring} 
% EndOfTitle
%short author -- toc 
{Stemberk et al.} 
%End short author -- toc
% Author(s) 
{Jakub Stemberk\textsuperscript{1}*, Petr Tábořík\textsuperscript{1}, Filip Hartvich\textsuperscript{1}, Zdeněk Fučík\textsuperscript{1}} 
% EndOfAuthor(s)
{\POtag} 
%Tag, can be: empty, \KLtag (keynote lecture), \IStag (invited speaker), \CTtag (contributed talk) or \ITtag (invited talk)
% Affiliation(s)
{
\textsuperscript{1}Czech Academy of Sciences, Institute of Rock Structure and Mechanics, Prague, Czechia
}
%} % EndOfAffiliation(s)
{jakub.stemberk@irsm.cas.cz}  %e-mail
%keywords
{slope deformations, landslide monitoring, time-lapse resistivity tomography, thresholds, hazard assessment, early warning}
%EndOfKeywords
%abstract content
{Monitoring of slope deformations is a key tool for determining both the activity of the studied landforms and landslide hazard. Utilization of a geophysical monitoring represents one of the ways to monitor the physical parameters of the observed slope deformation. The contribution summarizes the advantages of landslide monitoring using the time-lapse electrical resistivity tomography method, and, presents the gradual development of such monitoring from simple repeated measurements on fixed electrodes to the state-of-the-art time-lapse DAEMON system (Directly Accessed geoElectrical MONitoring system) that we have invented for this purpose. This system has already been installed at the Čeřeniště Natural Lab, which represents a complex slope deformation with all types of slope processed involved. It is a comprehensive system of measurements on buried electrodes with a direct access and an option to adjust the parameters and frequency of measurements remotely. The data are, at the same time, automatically sent to a remote storage, where they are automatically statistically processed and presented in a selected graphical form (e.g. a graph). The entire system thus combines the automated resistivity measurements, its remote control and data transfer with an automatic assessment of the results using our self-invented DAEMON software. The resistivity measurements are carried out to capture moisture changes in the section across the flowslide part of the investigated slope deformation and the causal connections among \enquote{precipitation – infiltration – moisture changes – landslide reactivation} are analysed. If we would determine the threshold for the measured physical parameter (apparent resistivity) or for derived values of soil moisture respectively, a warning message can be sent when these set values are reached or exceeded. In principle, the entire system provides a functional basis for a possible early warning system that could be used to monitor dangerous slope deformations. The question for future experimental research is to find a way to determine the threshold values used in a landslide hazard assessment.
}
%EndOfAbstractContent
%references
{
}
%EndOfReferences
%end of conference contribution
%---------------------------------------------------------------------------------------------------

%start of conference contribution
\abstract
% Title 
{Laboratory of Dendrogeomorphology (Dendroman.cz): overview and offers} 
% EndOfTitle
%short author -- toc 
{Šilhán et al.} 
%End short author -- toc
% Author(s) 
{Karel Šilhán\textsuperscript{1,2}*, Olga Chalupová\textsuperscript{1}, Andrea Fabiánová\textsuperscript{1}, Michal Kozák\textsuperscript{2}, Filip Hampel\textsuperscript{1}} 
% EndOfAuthor(s)
{\POtag} 
%Tag, can be: empty, \KLtag (keynote lecture), \IStag (invited speaker), \CTtag (contributed talk) or \ITtag (invited talk)
% Affiliation(s)
{
	\textsuperscript{1}Faculty of Science, University of Ostrava, Ostrava, Czech Republic
	\textsuperscript{2}Faculty of Science, University of Hradec Králové, Czech Republic
	\textsuperscript{3}
}
%} % EndOfAffiliation(s)
{karel.silhan@osu.cz; karel.silhan@uhk.cz}  %e-mail
%keywords
{dendrogeomorphology, dendroman.cz, laboratory, offers}
%EndOfKeywords
%abstract content
{The Laboratory of Dendrogeomorphology (\url{dendroman.cz}) is the only specialized workplace in the Czech Republic focused exclusively on the dating of geomorphological processes through the analysis of tree rings. The main working method of the laboratory is dendrogeomorphological analysis, which uses dendrochronological techniques to date various geomorphological processes. The only condition is that the process being studied must be able to influence tree growth in some way (directly, e.g. by damaging the stem or indirectly, e.g. by destroying neighbouring trees). The laboratory has extensive experience in the analysis of debris flows, rockfall, floods, gully and sheet erosion, snow avalanches and aeolian processes. Currently, however, the key topic is the study of landslide movements. The activities of the laboratory are focused not only on solving case studies but also on the development of dendrogeomorphological methods. For the geomorphological processes studied, not only their chronology but also their spatio-temporal occurrence and potential triggers are reconstructed. The activities of the laboratory are spread over the Czech part of the Outer Western Carpathians, the Czech Hercynian Mountains and the České Středohoří Mts. The foreign regions are the High and Western Tatras, the Slovenské rudohoří Mts., the Borská nížina lowland (Slovakia), the Gory kamienne (Poland), the Calimani Mountains (Romania), and the Crimean Mountains (Ukraine). In these areas, the laboratory has contributed to the understanding of the behaviour of natural hazards over the last up to 300 years, including their seismic and hydrometeorological triggers. Research results are published exclusively in impacted scientific journals (mostly Q1 ranking; e.g. Catena, Geomorphology, Science of the Total Environment, Engineering Geology, Landslides, Dendrochronology, Earth Surface Processes and Landforms etc.). Since the establishment of the laboratory in 2007, more than 70 IF papers have been produced and more than 60 students (B.Sc., M.Sc. and PhD) have defended their theses on dendrogeomorphology. The laboratory also participated in 9 projects of the Czech Science Foundation (GAČR), three of which it was the main investigator. The laboratory also hosts foreign students (e.g. Slovakia, Poland, Romania). The laboratory has modern equipment for field sampling (sets of Pressler increment borers, increment hammers, hand and chain saws, etc.), their laboratory preparation (vibrating, oscillating, belt and cylindrical grinders, electric and cordless planers, band saw) and analysis (dendrochronological measuring tables, stereoscopic microscopes, specialized software). In addition, the laboratory is also equipped for microscopic analysis of the anatomical structure of the tree rings (sledge microtome, core microtome, chemical laboratory, specialised software). The laboratory cooperates with domestic and foreign workplaces (Switzerland, Slovakia, Romania, Hungary, Japan, China, Mexico, Brazil, USA, etc.) and offers further cooperation. 
	
The Laboratory offers cooperation on issues related to: 
\begin{itemize}
	\itemsep0em 
	\item chronology
	\item spatio-temporal occurrence
	\item determination of magnitude x frequency
	\item evaluation of potential triggers
\end{itemize}	
of various geomorphological processes from the areas
\begin{itemize}
	\itemsep0em 	
	\item slope processes
	\item fluvial-geomorphological processes
	\item erosional processes
	\item snow avalanches
	\item anthropogenic relief transformation
\end{itemize}		
Due to its long experience in project solutions, the laboratory is also open for cooperation in the preparation, submission and solving of scientific projects.
	
If you are interested in our offer, please do not hesitate to contact us at karel.silhan@osu.cz.
}
%EndOfAbstractContent
%references
{
}
%EndOfReferences
%end of conference contribution
%---------------------------------------------------------------------------------------------------

%start of conference contribution
\abstract
% Title 
{Late Pleistocene activity of Central European Fault driven by ice loading as revealed by geological data and modeling} 
% EndOfTitle
%short author -- toc 
{Štěpančíková et al.} 
%End short author -- toc
% Author(s) 
{Petra Štěpančíková\textsuperscript{1}*, Thomas K. Rockwell\textsuperscript{1,2}, Jakub Stemberk\textsuperscript{1}, Edward J. Rhodes\textsuperscript{3,4}, Filip Hartvich\textsuperscript{1}, Karen Luttrell\textsuperscript{5}, Madeline Myers\textsuperscript{5,6}, Petr Tábořík\textsuperscript{1}, Dylan H. Rood\textsuperscript{7}, Neta Wechsler\textsuperscript{8}, Daniel Nývlt\textsuperscript{9}, María Ortuño\textsuperscript{10}, Jozef Hók\textsuperscript{11}} 
% EndOfAuthor(s)
{\TLtag} 
%Tag, can be: empty, \KLtag (keynote lecture), \IStag (invited speaker), \CTtag (contributed talk) or \ITtag (invited talk)
% Affiliation(s)
{
	\textsuperscript{1}Inst. of Rock Structure and Mechanics, Czech Acad. of Sci., Prague, Czech Rep.\\
	\textsuperscript{2}Dept. of Geol. Sci., San Diego State Univ., California\\
	\textsuperscript{3}Dept. of Geography, Univ. of Sheffield, S10 2TN UK\\
	\textsuperscript{4}Dept. of Earth, Planetary and Space and Sciences, University of California Los Angeles, CA, USA\\
	\textsuperscript{5}Dept. of Geology and Geophysics, Louisiana State Univ., Baton Rouge, Louisiana\\
	\textsuperscript{6}Dept. of Geography and Planning, Queen’s University, Kingston, Ontario, Canada\\
	\textsuperscript{7}Dept. of Earth Science and Engineering, Royal School of Mines, Imperial College London, UK\\
	\textsuperscript{8}Neev Center for Geoinformatics, Fredy \& Nadine Herrmann Institute of Earth Sciences, The Hebrew University of Jerusalem, Israel\\
	\textsuperscript{9}Faculty of Science, Masaryk Univ., Brno, Czech Rep.\\
	\textsuperscript{10}Risknat-Institute Geomodels, Dept. of Earth and Ocean Dynamics, University of Barcelona, Spain\\
	\textsuperscript{11}Faculty of Science, Comenius Univ., Bratislava, Slovakia  
}
%} % EndOfAffiliation(s)
{stepancikova@irsm.cas.cz}  %e-mail
%keywords
{paleoseismology, intraplate earthquakes, ice-loading, Sudetic Marginal fault, Bohemian Massif, Central Europe}
%EndOfKeywords
%abstract content
{Southern part of the NW-SE trending Sudetic Marginal fault (SMF), situated at the northeastern limit of the Bohemian Massif in central Europe, was studied to evaluate its Quaternary activity. We excavated eighteen trenches and performed thirty-four electric resistivity profiles at Bílá Voda site to study the fault zone and 3-dimensional distribution of discovered beheaded alluvial fan on the NE side of the fault.  A small drainage, located about 29--45 m to the SE of the fan apex was interpreted as the only plausible source channel, which implies a left-lateral offset of that amount. Radiometric ages of the alluvial fan deposits range between about 24 and 63 ka, but postglacial deposits younger than 11 ka are not displaced, indicating that all motion occurred in the late Pleistocene. As the site lies \textasciitilde150 km south of the late Pleistocene Weichselian maximum (\textasciitilde20 ka) ice sheet front, we modelled the effects of the ice load on lithospheric flexure and resolved fault stresses showing that slip on the SMF was promoted by the presence of the ice sheet, resulting in a late Pleistocene slip rate of \textasciitilde1.1\textsuperscript{+2.3}/\textsubscript{\textminus0.6} mm/yr. As the modern maximum principal stress (s1) is oriented nearly parallel to the Sudetic Marginal fault (NNW-SSE), thus unfavorably for continued fault motion, probability of earthquake production in future is much lower in the absence of an ice sheet.
}
%EndOfAbstractContent
%references
{Štěpančíková P., Rockwell T. K., Stemberk J. jr., Rhodes E. J., Hartvich F., Luttrell K., Myers M., Tábořík P., Rood D. H., Wechsler N., Nývlt D., Ortuño M., Hók J. (2022) Acceleration of Late Pleistocene Activity of a Central European Fault Driven by Ice Loading. Earth and Planetary Science Letters, 2022, accepted 
}
%EndOfReferences
%end of conference contribution
%---------------------------------------------------------------------------------------------------
%start of conference contribution
\abstract
% Title 
{Structure of active faults revealed from 2D deep-reflection seismic profiles in the flysch belt of Outer Western Carpathians} 
% EndOfTitle
%short author -- toc 
{Šuťjak et al.} 
%End short author -- toc
% Author(s) 
{Martin Šuťjak\textsuperscript{1}*, Ivo Baroň\textsuperscript{2}, Rostislav Melichar\textsuperscript{1}, Filip Harvitch\textsuperscript{2} Lenka Kociánová\textsuperscript{1}, Václav Dušek\textsuperscript{1}, František Bárta\textsuperscript{1}} 
% EndOfAuthor(s)
{\TLtag} 
%Tag, can be: empty, \KLtag (keynote lecture), \IStag (invited speaker), \CTtag (contributed talk) or \ITtag (invited talk)
% Affiliation(s)
{
	\textsuperscript{1}Ústav geologických věd PřF MU, Kotlářská 2, 611 37 Brno, Czech Republic\\
	\textsuperscript{2}Ústav struktury a mechaniky hornin AV ČR, V Holešovičkách 94/41, 182 09 Praha, Czech Republic
}
%} % EndOfAffiliation(s)
{432503@mail.muni.cz}  %e-mail
%keywords
{Outer Western Carpathians, flysch belt, tectonics, fault, seismic interpretation}
%EndOfKeywords
%abstract content
{In course of past couple of years, two distinct active faults have been recognized and documented in the flysch belt of the Outer Western Carpathians, i.e. the Lidečko and Mikulov Faults. The polyphase and bi-modal Lidečko Fault with dominating dextral normal movements dissects an anticlinal ridge of the Kopce Hill in the Rača Unit of the Magura Nappe, while the N-S trending sinistral Mikulov Fault separates the Jurassic limestone klippens of the Palava Hills. The main aim of this study is to complement the field structural mapping, LiDAR digital terrain model analyses, near-surface geophysical ERT surveys, and radiometric dating with deeper subsurface faults´ architecture analysis using existing seismic profiles available from the Czech Geological Survey - Geofond. The two-dimensional deep-reflection seismic profiles were interpreted in the Petrel software in combination with the deep wells information. Using the seismic interpretation tool, several stratigraphically essential horizons were recognized. Based on the offsets of the known stratigraphy and faults´ position on the ground surface, we aim on determining the causative processes of these young faults.

\vspace{0.5em}
\noindent
\textbf{Acknowledgements:}
\textit{The research was funded by the Grant Agency of the Czech Republic (GC22-24206J).}
}
%EndOfAbstractContent
%references
{
}
%EndOfReferences
%end of conference contribution
%---------------------------------------------------------------------------------------------------

%start of conference contribution
\abstract
% Title 
{Joint interpretations of geophysical measurements as the key to understand slope evolution} 
% EndOfTitle
%short author -- toc 
{Tábořík P.} 
%End short author -- toc
% Author(s) 
{Petr Tábořík\textsuperscript{1,2}*} 
% EndOfAuthor(s)
{\POtag} 
%Tag, can be: empty, \KLtag (keynote lecture), \IStag (invited speaker), \CTtag (contributed talk) or \ITtag (invited talk)
% Affiliation(s)
{
	\textsuperscript{1}Czech Academy of Sciences, Institute of Rock Structure and Mechanics, Prague, Czechia
	\textsuperscript{2}Charles University, Faculty of Sciences, Prague, Czechia
}
%} % EndOfAffiliation(s)
{taborik@irsm.cas.cz}  %e-mail
%keywords
{applied geophysics, physical property, geological quality, joint interpretation, reinterpretation, landslide evolution}
%EndOfKeywords
%abstract content
{Applied geophysical survey represents fast, efficient and non-destructive way of acquiring information on the composition and conditions of the rock environment forming the investigated slope deformations. A combination of various geophysical surveying techniques provides a significantly wider range of measured physical parameters which allows much more complete information on the studied landslides (if compared to the use of a single geophysical method). As geophysical results often represent only one of many possible solutions, the interpretations of geophysical data have always to be carried out very carefully. The presented contribution illustrates -- using several selected case studies -- how easily an inaccurate or even misleading interpretation can be achieved. In many cases, there is no problem with data quality or erroneous settings the of computational model parameters, but the error arises during the interpretation of results when we assign specific geological quality to certain measured or modelled physical property or observed anomaly. Joint interpretations of geophysical methods, ideally further supported by additional information of \enquote{non-geophysical} nature (e.g. boreholes), can provide crucial information not only on the studied slope deformations, but also contribute to the methodological know-how, which is particularly important when using combinations of various geophysical methods. Such integrated interpretations may result in reinterpretations of the original hypotheses or partial results of the individual methods used.

This work evaluates the application potential of the selected geophysical methods in landslide investigation as illustrated by the presented case studies, but also analyses the practical aspects of used combinations of geophysical surveying methods. 
}
%EndOfAbstractContent
%references
{
}
%EndOfReferences
%end of conference contribution
%---------------------------------------------------------------------------------------------------

%start of conference contribution
\abstract
% Title 
{Special resistivity survey for imaging the sub-aquatic geology} 
% EndOfTitle
%short author -- toc 
{Tábořík et al.} 
%End short author -- toc
% Author(s) 
{Petr Tábořík\textsuperscript{1,2}*, Filip Hartvich\textsuperscript{1,2}, Jakub Stemberk	\textsuperscript{1}, Miroslav Šobr\textsuperscript{2}} 
% EndOfAuthor(s)
{\TLtag} 
%Tag, can be: empty, \KLtag (keynote lecture), \IStag (invited speaker), \CTtag (contributed talk) or \ITtag (invited talk)
% Affiliation(s)
{
	\textsuperscript{1}Czech Academy of Sciences, Institute of Rock Structure and Mechanics, Prague, Czechia
	\textsuperscript{2}Charles University, Faculty of Sciences, Prague, Czechia
}
%} % EndOfAffiliation(s)
{taborik@irsm.cas.cz}  %e-mail
%keywords
{river/lake sediments, river/lake bed, bedrock detection, volume estimation, sonar, waterborne resistivity tomography, suspended measurement system}
%EndOfKeywords
%abstract content
{
Determination of the thickness or even volume of bottom sediments is key for both scientific research and especially for water management practice. Sedimentation rates can be very rapid and the sediment increment per year very high. Particularly in the water management case, the quantification of the volume or at least the sediment thickness may be urgent after a flood event. Estimation of the volume then requires measurements on the entire network of profiles across a streams or lakes (both natural and artificial, such as a dams). The actual bottom, i.e. the surface of the investigated sediments, can be determined quite precisely using various bathymetric methods, such as sonar sounding or measurements with Acoustic Doppler Current Profiler. Nevertheless, to determine the sediment thickness (or volume), it is necessary to know the position of the bottom beneath the sediments. Drilling into the river or lake bed is demanding and provides only point information. To determine the river bed (or lake bed) beneath the sediments, we adapted the geophysical method of electrical resistivity tomography for measurements from the water surface. Based on the hypothesis that the sediments have different resistivity values than the underlying bedrock, we are thus able to determine the bottom position. In combination with a precise bathymetry (e.g. using sonar) we can easily calculate the sediment thickness or estimate the sediment volume. However, resistivity measurements from the water surface is a challenging event and requires careful planning and preparations. The contribution presents the main advantages and disadvantages of the waterborne ERT, its limitations and difficulties as well as examples of successful use of the described approach. Specifically, it describes the possibilities of measuring on large rivers, where measurements using floating pads are prevented by high discharge. For such cases, we have invented a special measurement technique that involves a hanging suspension strand across the water surface, supporting the curtain-like system of multi-electrode cable sections with measuring electrodes suspended to reach the water.
}
%EndOfAbstractContent
%references
{
}
%EndOfReferences
%end of conference contribution
%---------------------------------------------------------------------------------------------------

\abstract
{Dendrogeomorphic dating in the Ostrava-Karviná mining landscape: processes, events, responses} % Title
{Tichavský et al.} %short author -- toc
{Radek Tichavský\textsuperscript{1}*, Andrea Fabiánová\textsuperscript{1}, Eva Jiránková\textsuperscript{2}, Jan Lenart\textsuperscript{1}, Lucie Polášková\textsuperscript{1}, \\Radim Tolasz\textsuperscript{3}
} % Author(s)
{\TLtag} % Tag, can be: empty, \KLtag (keynote lecture), \IStag (invited speaker), \CTtag (contributed talk) or \ITtag (invited talk)
{
\textsuperscript{1}Department of Physical Geography and Geoecology, Faculty of Science, University of Ostrava, Chittussiho 10, 710 00 Ostrava, Slezská Ostrava, Czech Republic\\
\textsuperscript{2}Institute of Geonics of the Czech Academy of Sciences, Studentská 1768, 708 00 Ostrava, Czech Republic\\
\textsuperscript{3}Czech Hydrometeorological Institute, Na Šabatce 17, 143 06 Praha 4, Czech Republic
} % Affiliation(s)
{radek.tichavsky@osu.cz}  %e-mail
{mining landscape, dendrogeomorphology, subsidence, landslide, compression wood, Ostrava-Karviná mining district}%keywords
{Mining-induced subsidence is a worldwide environmental problem that leads to damage to infrastructure and property; thus, knowledge of its past–recent development is critically needed during land-use planning. We dated the activity of gravitational processes related to mining-induced subsidence using dendrogeomorphic methods at four sites in the Upper Silesian Coal Basin near the former mines of the Karviná mining district. The landslide/subsidence sites were characterised by scarps, rotated blocks, stepped relief with grabens, pressure folds, and shallow movements within landslide toes or within flat undulating relief. Based on 644 increment cores from 161 trees, we were able to identify subsidence/landslide activity, with the most frequent responses occurring from the late 1950s to the early 1970s and from 2005 to 2011. The highest number of tree ring records dates back to 1973. Moreover, more than 30\% of tilted trees from the landslide sites created compression wood at multiple directions indicated complex deformations. At the other sites, the occasional occurrence of single rings with isolated compression wood was included as a possible indicator of events due to frequent coincidence with common series of compression wood. A comparison of tree-ring-based chronology with in situ monitoring revealed good synchronicity with the periods of increased subsidence rates (between 1 and 3 m.year\textsuperscript{−1}). In addition, not only mining operations, but also the occurrence of extreme rainfall seems to be responsible for the recent surface activity. We identified that the 3-year precipitation and the Simple daily intensity index were significantly higher during event years at the site with a flatter topography and clay soils (p=0.02 and 0.01, respectively). Since dendrogeomorphic research on subsidence is influenced by multiple factors, including geological settings, soil conditions, or surface morphology, future research should focus on small research plots that provide detailed tree, surface, and subsurface characteristics. In addition, another challenge is to compare broadleaved and coniferous trees in terms of their sensitivity to recording subsidence movements on the same geomorphic forms.
}%abstract
{}%references
%end of abstract
%---------------------------------------------------------------------------------------------------

%start of conference contribution
\abstract
% Title 
{Spatial characteristics of a confluence hydrodynamic zones of the Vsetínská Bečva River (Javorníky Mts.): a relationship between watershed and reach scale morphology} 
% EndOfTitle
%short author -- toc 
{Vaverka and Škarpich} 
%End short author -- toc
% Author(s) 
{Lukáš Vaverka\textsuperscript{1}*, Václav Škarpich\textsuperscript{1}} 
% EndOfAuthor(s)
{\TLtag} 
%Tag, can be: empty, \KLtag (keynote lecture), \IStag (invited speaker), \CTtag (contributed talk) or \ITtag (invited talk)
% Affiliation(s)
{
	\textsuperscript{1}Department of Physical Geography and Geoecology, Faculty of Science, University of Ostrava, Chittussiho 10, Ostrava, Czech Republic
}
%} % EndOfAffiliation(s)
{lukas.vaverka@osu.cz}  %e-mail
%keywords
{confluence hydrodynamic zones, gravel-bed river, morphology, Javorníky Mts.}
%EndOfKeywords
%abstract content
{
River confluences are integral parts of the fluvial system with potential to disrupt downstream longitudinal trends in the main river through inputs of sediment, water, or woody debris (Best 1988; Benda et al. 2003; Bilal et al. 2020). These different inputs make confluence zones hotspots for morphological diversity with implications for aquatic biota (Benda et al. 2004). This study focuses on a possible connection between the spatial characteristics of the watershed and selected hydrodynamic zones (CHZ). Eighteen tributaries and adjacent CHZ were selected along the 50-km-long reach of the Vsetínská Bečva River in Javorníky Mts. The Vsetínská Bečva River is channelized, mostly by weirs and riverbank stabilizations. Tributaries with potential morphological activity were selected using the connectivity index (Cavalli et al. 2013) and with a preliminary field survey. Spatial data (mainly bed elevation) of each CHZ were measured with the total station in transverse profiles, perpendicular to the water flow. Additionally, the surface level of a streambed of each tributary was sampled using the Wolman method (Wolman 1954), by which 100 randomly selected coarse-grained sediment clasts (>5 mm) were measured. The sediment data did not show significant downstream fining, but rather a slight increase with abrupt fluctuations in the 50\textsuperscript{th} percentile (21 -- 61 mm), which corresponds to the results of the connectivity index. The drops in sediment size can be attributed to contributing streambed regulations. Biswas et al. 2019 showed that the morphological activity in CHZ can be associated with the size of the contributing catchment area. By comparison of the catchment area of contributing stream and the main-stem Vsetínská Bečva River (area upstream from tributary), there was no strong evidence to believe that it may be leading factor of CHZ morphology in this case. Instead, the angle between the main channel and a tributary was considered. Previous studies showed that angle is one of the main factors defining the morphology of CHZ (Mosley 1976; Chen et al. 2017). In this case, the comparison between bed elevation in confluence scour and angle between tributary and main-stem channel showed in most situations a possible connection. The most apparent changes in bed elevation appeared in CHZ with 60°-75° and >80° respectively. It is reasonable to assume that hydrodynamical activity, responsible for scour formation, can be manifested even with tributaries with relatively low energy.
}
%EndOfAbstractContent
%references
{Benda, L., Andras, K., Miller, D., Bigelow, P. (2004): Confluence effects in rivers: Interactions of basin scale, network geometry, and disturbance regimes: EFFECTS OF CONFLUENCES ON RIVER MORPHOLOGY. Water Resources Research, 5, 40. 

Benda, L., Veldhuisen, C., Black, J. (2003): Debris flows as agents of morphological heterogeneity at low-order confluences, Olympic Mountains, Washington. Geological Society of America Bulletin, 9, 115, 1110. 

Best, J. L. (1988): Sediment transport and bed morphology at river channel confluences. Sedimentology, 3, 35, 481–498. 

Bilal, A., Xie, Q., Zhai, Y. (2020): Flow, Sediment, and Morpho-Dynamics of River Confluence in Tidal and Non-Tidal Environments. Journal of Marine Science and Engineering, 8, 8, 591. 

Biswas, S. S., Pal, R., Pani, P. (2019): Application of Remote Sensing and GIS in Understanding Channel Confluence Morphology of Barakar River in Western Most Fringe of Lower Ganga Basin. In: Das, B. C., Ghosh, S., Islam, A. (eds.): Quaternary Geomorphology in India. Springer International Publishing, Cham, 139–153. 

Cavalli, M., Trevisani, S., Comiti, F., Marchi, L. (2013): Geomorphometric assessment of spatial sediment connectivity in small Alpine catchments. Geomorphology, 188, 31–41. 

Chen, X., Zhu, D. Z., Steffler, P. M. (2017): Secondary currents induced mixing at channel confluences. Canadian Journal of Civil Engineering, 12, 44, 1071–1083. 

Mosley, M. P. (1976): An Experimental Study of Channel Confluences. The Journal of Geology, 5, 84, 535–562. 

Wolman, M. G. (1954): A method of sampling coarse river-bed material. Transactions, American Geophysical Union, 6, 35, 951.
}
%EndOfReferences
%end of conference contribution
%---------------------------------------------------------------------------------------------------

%start of conference contribution
\abstract
% Title 
{Analysis of bomb craters using LiDAR data (Kędzierzyn-Koźle, Poland)} 
% EndOfTitle
%short author -- toc 
{Waga et al.} 
%End short author -- toc
% Author(s) 
{Jan Maciej Waga\textsuperscript{1}, Bartłomiej Szypuła\textsuperscript{1}*, Maria Fajer\textsuperscript{1}} 
% EndOfAuthor(s)
{\TLtag} 
%Tag, can be: empty, \KLtag (keynote lecture), \IStag (invited speaker), \CTtag (contributed talk) or \ITtag (invited talk)
% Affiliation(s)
{
	\textsuperscript{1}Institute of Earth Sciences, Faculty of Natural Sciences, University of Silesia in Katowice, Poland
}
%} % EndOfAffiliation(s)
{bartlomiej.szypula@us.edu.pl}  %e-mail
%keywords
{bomb craters, LiDAR, WWII, Koźle Basin, Poland}
%EndOfKeywords
%abstract content
{The geomorphological legacies of World War II are still present in many European landscapes - huge numbers of bomb craters from large-scale aerial bombardments. Research into landscapes remodeled by aerial bomb explosions has largely focused on the battlefields of Europe (Passmore et al., 2018; Passmore and Capps-Tunwell, 2020; De Matos-Machado \& Hupy, 2019; Dolejš et al., 2020; Seitsonen, 2021). In Poland, despite the presence of traces of hostilities, there are few publications dealing with this issue (Kobiałka, 2017, 2018; Waga and Fajer, 2021; Waga et al., 2022). In the vicinity of Kędzierzyn-Koźle (southern Poland), many craters have been preserved, which are the remains of numerous bombings from the World War II. At that time, these were the most effective strategic actions that destroyed the industry of the Third Reich and hastened the end of the war. Despite the passage of more than 80 years, the bomb craters are still very legible directly in the field and on photographic materials and altitude data. The study area is located in southern Poland, within Koźle Basin. Detailed research was conducted in 5.9 ha, located at the western border of the former IG Farbenindustrie AG Werke Heydebreck, Kędzierzyn – Racibórz railway route and Bierawa cargo-passenger station. In 1944 United States Air Force (USAAF) used on this area: 500-pound (227 kg), 250-pound (113 kg) demolition bombs (RDX) and 70-pound incendiary bombs. The aim of the research was to determine the number, morphometry, morphology of bomb craters, spatial distribution and arrangement of bomb craters, and attempt to identify geomorphic processes that modified these craters. The methodological aim, which often appears when working with modern research tools, was to test the suitability of DEMs with a resolution of 1x1m, 0.1x0.1m and 0.05x0.05m and the appropriate shaded relief rasters in detailed studies of the morphology and morphogenesis of these forms. It should be emphasized that the survival of numerous bomb craters from World War II in the area of Kędzierzyn-Koźle until today is an unusual phenomenon on a European scale. Their remains are found in forests as well as in open areas. The use of laser scanning (ALS) in the study of similar war traces facilitates the management of war heritage, including the conflict landscape as a war witness (Gheyle et al., 2018; De Matos-Machado \& Hupy, 2019). Areas with bomb craters are an undesirable element of space that complicates planning also due to the presence of unexploded bombs (UXB) (Barone, 2019; Waga et al., 2022). According to post-war estimates, 10--15\% of the used bombs were not detonated as planned (Dolejš et al., 2020). However, for historical and environmental reasons, the crater areas are an excellent field of research. For this reason, the group of forms from Kędzierzyn-Koźle is certainly one of the most valuable in Europe in terms of research potential. The results of such studies allow, among others, to develop a strategy for their protection as natural and educational sites (Passmore and Harrison, 2008; Kobiałka et al., 2018; Claeys et al., 2019; De Matos Machado and Hupy, 2019; Passmore and Capps-Tunwell, 2020). The conducted research allowed for the formulation of the following conclusions: 1) a high concentration of craters was found in the study area, which are visible in the topography - on average 47.8/ha, max. 77/ha in the places of concentration; 2) there are craters of three sizes (the largest 500-pound bombs, medium-sized 250-pound bombs and the smallest - probably containing UXB). The largest craters were formed by the overlapping of several eruptions; 3) crater diameters after the explosion of the same bombs (500 pounds) are larger in wetlands and smaller in drier areas; 4) the size of the craters was also influenced by the former land development, especially the presence of hardened, compact, embanked, artificially sodded and drained surfaces; 5) in the vicinity of former IG Farben Werke, many complexes of connected funnels filled with water have been preserved. In some of them, due to better water transparency, shaded relief rasters from LiDAR data contain a clear record of geomorphological processes, including those taking place in ponds formed by detonated bombs; 6) analysis of DEM fragments that were generated from high-resolution LiDAR data in a resolution of 1x1m, 0.1x0.1m and 0.05x0.05m, showed: an adverse effect of the \enquote{floating islands} that make up the leaves, on the quality (accuracy) of modeling, including, in particular, a change in the image of the crater bottom relief.
}
%EndOfAbstractContent
%references
{Barone P. M (2019) Bombed Archaeology: Towards a precise identification and a safe management of WWII’s dangerous unexploded bombs. Heritage 2 (4), 2704–2711.

Claeys D., Van Dyck C., Verstraeten G., Segers Y (2019) The importance of the Great War compared to long-term developments in restructuring the rural landscape in Flanders (Belgium). Appl. Geogr. 111, 102063.

De Matos-Machado R., Hupy J.P (2019) The Conflict Landscape of Verdun, France: Conserving Cultural and Natural Heritage After WWI, in: Lookingbill T., Smallwood P.  (Eds.), Collateral Values. Landsc. Ser., 25. Springer, Cham. pp 111–132.

Dolejš M., Samek V., Veselý M., Elznicová J (2020) Detecting World War II bombing relics in markedly transformed landscapes (city of Most, Czechia). App. Geogr. 119, 102225.

Gheyle W., Stichelbaut B., Saey T., Note N., Van den Berghe H., Van Eetvelde V., Van 

Meirvenne M., Bourgeois J (2018) Scratching the surface of war. Airborne laser scans of the Great War conflict landscape in Flanders (Belgium). App. Geogr. 90, 55–68.

Kobiałka D (2017) Airborne laser scanning and 20th century military heritage in the woodlands. Analecta Archaeol. Ressoviensia 12, 247-269.

Kobiałka D (2018) Being in the woodlands: archaeological sensibility and landscapes as naturecultures. Pol. J. Landsc. Stud. 1, 43–53.

Passmore D.G., Capps-Tunwell D (2020) Conflict archaeology of tactical air power: The Forêt Domaniale de la Londe-Rouvray and the Normandy Campaign of 1944. Int. J. Hist. Archaeol. 24 (3), 674–706.

Passmore D.G., Capps-Tunwell D., Harrison S (2018) Revisiting the US military ‘Levels of War’ model as a conceptual tool in conflict archaeology: A case study of WW2 landscapes in Normandy, France. Fields of Conflict, Conf. Proc., 1, 10th Bienn. Int. Conf., 26–30 September 2018, Mashantucket Peguot Museum \& Research Center.

Seitsonen O (2021) Archaeologies of Hitler’s Arctic War Heritage of the Second World War German Military Presence in Finnish Lapland. Routledge, London.

Waga J.M., Fajer M (2021) The heritage of the Second World War: bombing in the forests and wetlands of the Koźle Basin). Antiq. 95 (380), 417–434.

Waga J. M., Fajer M., Szypuła B., 2022: The scars of war: A programme for the identification of the environmental effects of Word War II bombings for the purposes of spatial management in the Koźle Basin, Poland. Environmental \& Socio-economic Studies, vol.10, no.1, pp.57-67. 
}
%EndOfReferences
%end of conference contribution
%---------------------------------------------------------------------------------------------------

%start of conference contribution
\abstract
% Title 
{Interpretation of the small water-filled bomb craters morphology based on LiDAR data (Koźle Basin, Poland)} 
% EndOfTitle
%short author -- toc 
{Waga et al.} 
%End short author -- toc
% Author(s) 
{Jan Maciej Waga\textsuperscript{1}, Bartłomiej Szypuła\textsuperscript{1}*, Maria Fajer\textsuperscript{1}} 
% EndOfAuthor(s)
{\POtag} 
%Tag, can be: empty, \KLtag (keynote lecture), \IStag (invited speaker), \CTtag (contributed talk) or \ITtag (invited talk)
% Affiliation(s)
{
		\textsuperscript{1}Institute of Earth Sciences, Faculty of Natural Sciences, University of Silesia in Katowice, Poland
}
% % EndOfAffiliation(s)
{bartlomiej.szypula@us.edu.pl}  %e-mail
%keywords
{water-filled bomb craters, LiDAR, Koźle Basin}
%EndOfKeywords
%abstract content
{
During the study of the World War II bombing remains in the Kozielska Basin (Waga et al, 2022), it was found that high-resolution images of shaded relief show the elements of the morphology of the bottom of the flooded craters. Some of the pulses, despite the wavelength of 1064 nm (not typical for bathymetric scanning) reached the bottom of the craters and were recorded in the receiver. A green laser is used as standard in bathymetric tests. Without it, it was decided to verify the usefulness of the images of these shallow water reservoirs for geomorphological analyzes. It turned out that although these data are not precise enough for geodetic applications, they can be useful in geomorphological research.

The research was carried out in two areas with a total area of 14 ha, to the west of Nitrogen Works Kędzierzyn. Currently, they are among the most spectacular areas with existing bomb craters in Europe (average up to 47.8 / ha, max. 77/ha). In the first phase of the research, high-resolution * .las type data (min. density 12 points/m\textsuperscript{2}, average height error 0.1 m; derived from aerial scanning taken with the Leica ALS70 scanner) were analysed. After filtering the *.las files the points of class 2 (ground) and 9 (water) were selected from the first and second reflections and based on them digital elevation models were created with a resolution of 0.1 x 0.1 m and 0.05 x 0.05 m in the PUWG-1992 coordinate system (EPSG: 2180). Subsequently, these models were used for create shaded relief rasters (with the same resolution as models) and with standard lighting settings (azimuth 315°, angle of the sunlight source above the horizon 45°). Such prepared materials became the basis for conducting detailed research on the morphology of the craters and, in the next step, their verification in the field. Depth of the examined reservoirs turned out to be small (0.3--1.35 m, most often circa 0.7 m). Their bottoms are covered with a 5-12 cm thick compact organic layer. Above it, in deeper places a \enquote{soft bottom} sediment accumulates and above a very fine, dispersed suspension organic. In flooded forms, one was made an attempt to recreate complex ones a sequence of outbursts occurring in partially overlapping craters. Course of the events was written in them by the arrangement of forms within the bottoms of the reservoirs. 

The most useful shaded relief raster for the analysis of the morphology of the bottom was obtained from a model with a resolution of 0.1x0.1 m. In reservoirs with an increased content of detritic organic material and where the activity of animals caused turbidity of the water, the image from LiDAR data is blurred. Another factor that hindered the correct reflection of the laser rays from the bottom of the reservoir were the clusters of sticks and leaves forming "floating islands" on the water. They reflect the laser light like a roach. The shaded relief rasters show then circular bends in the representation of the bottoms of the reservoirs.

Another issue is the apparent presence of depressions "phantom pockets" at the bottom of the water reservoirs, in the vicinity of their shores. It should be assumed that this phenomenon could have been influenced by the shape of very steep or even locally vertical slopes of the craters, as well as the accumulation of floating leaves at the edge. Further research is required to recognize the factors responsible for the formation of shadow zones and to trap some of the feedback signals after the rays are reflected from the bottom of the reservoir. Similar disturbances in the mapping of steep slopes and river channels as well as places covered with snow were also found when using topo-bathymetric and bathymetric scanners.
}
%EndOfAbstractContent
%references
{Waga JM, Fajer M, Szypuła B (2022) The scars of war: A programme for the identification of the environmental effects of World War II bombings for the purposes of spatial management in the Koźle Basin. Environmental \& Socio-economic Studies, 10, 1: 57-67.
}
%EndOfReferences
%end of conference contribution
%---------------------------------------------------------------------------------------------------




\abstract
{Impact of flood events and Eurasian beaver (Castor fiber) activity on spontaneous renaturalization of Łososina River in the Polish Western Carpathians} % Title
{Wąs and Gorczyca} %short author -- toc
{Joanna Wąs\textsuperscript{1}*, Elżbieta Gorczyca\textsuperscript{2}} % Author(s)
{\POtag} % Tag, can be: empty, \KLtag (keynote lecture), \IStag (invited speaker), \CTtag (contributed talk) or \ITtag (invited talk)
{\textsuperscript{1}Institute of Geography and Spatial Organization, Polish Academy of Sciences\\
	\textsuperscript{2}Institute of Geography and Spatial Management, Jagiellonian University in Cracow} % Affiliation(s)
{joanna.was@zg.pan.krakow.pl}  %e-mail
{spontaneous renaturalization, river channel development, channel migration zone, flood, European Beaver (Castor fiber), Beskid Wyspowy Mts.}%keywords
{Rivers in Polish part of Carpathians are heavily modified. Valleys are urbanized, banks are enforced, channels are straightened and various types of barriers are employed to regulate the flow. All these alterations cause disturbance of natural morphodynamic processes and destruction of habitats. Anthropogenic pressure may also increase risks related to floods and droughts.
	
	Contrary to the trends prevailing in the industrial era, more and more attention is now being paid to the negative effects of river regulation (e.g. Gregory, 2006). Growing awareness of the importance of preserving or restoring the natural state of rivers and water resources for the proper functioning of the environment and people leads to implementation of river restoration projects (e.g. Wohl et al., 2005). However, such actions are expensive and therefore not carried out on a large scale. That is why all factors that may cause renaturalization process to occur without technical treatment deserve special consideration.
	
	Some rivers display tendencies to partially compensate for the disturbances via spontaneous changes of various parameters such as sinuosity and wideness (e.g. Bollati et al., 2014). When left not maintained training structures may deteriorate in time. Eventually such constructions collapses and therefore natural hydromorphological processes may be restored to some extent, especially with external help.
	
	One of the key features characteristic for the multithread natural or semi-natural river is heterogeneity of the flow. Frequently such conditions are not met in regulated part of rivers. To help restore natural state we may employ various expensive technical measures or we can let it be done by the beavers (Castor fiber) which are known for their ability to create heterogeneous (aquatic and terrestrial) habitats (e.g. Rosell et al., 2005).
	
	We examined channel development of gravel-bed river Łososina in Beskid Wyspowy Mts. (Western Carpathians). During this project both of these factors were found to be important for spontaneous renaturalization. Analyses were conducted using data concerning river training made by RZGW (1973–2015), orthophotos, aerial photos, topographic maps (1845-1877, 1963-2018) and field surveys (2018-2019). 
	
	Historically (in XIX century) Łososina had multithread planar course on most of its length and wide channel migration zone. Intensification of the river training in second half of XX century resulted in simplification of planar pattern and shrinking of migration zone. From XX to XXI century riverbed was also incised by 2.5 m. During our study we determined five distinctive river sections that spontaneously increased its sinuosity and migration area despite continuous river training efforts.
	
	So defined process of spontaneous renaturalization took place both gradually and abruptly during research period. The greatest changes in sinuosity coincided in time with flood events. Such processes resulted in development of wandering pattern. While manoeuvring on a greater area river gets access to a wider range and bigger amounts of materials to promote channel heterogeneity and evolution. When it erodes new areas it acquires wood debris, gravel and sand which later becomes foundations for bars, islands, backwaters etc. Newly created sediments becomes unique dynamic habitats for pioneer plants and animals in constant cycle of destruction and recreation.
	
	Beavers occupy almost the entire length of Łososina river but their dams were observed only on the analyzed five sections. In zones of active channel migration water in old abandoned threads is preserved for years by impoundments created by beavers. Furthermore thus created more stagnant waters present new ecological niches.
	
	Though the area covered by those five more dynamic sections is much smaller than the historical extent of active migration it serves as so called “beads on string” (as described by Ward et al., 2002). Such system of retention zones and areas more resilient to disturbances may work as stepping stones for river connected species. Unfortunately due to continuous river management ability to spontaneously restore natural processes in Łososina is restricted.
}%abstract

{Bollati I. M., Pellegrini L., Rinaldi M., Duci G., Pelfini M. (2014), Reach-scale morphological adjustments and stages of channel evolution: The case of the Trebbia River (northern Italy), Geomorphology 221: 176-186. 
	
	Gregory K. J. (2006) The human role in changing river channels, Geomorphology 79(3-4): 172-191.
	
	Rosell F., Bozsér O., Collen P., Parker H. (2005) Ecological impact of beavers Castor fiber and Castor canadensis and their ability to modify ecostystems, Mammal Review 35(3-4): 248-276. 
	
	Ward J. V., Tockner K., Arscott D. B., Claret C. (2002) Riverine landscape diversity, Freshwater Biology, 47(4): 517-539.
	
	Wohl E., Angermeier P.L., Bledsoe B., Kondolf G.M., MacDonnell L., Merritt D.M., Palmer M.A., Poff N.L., Tarboton D. (2005) River restoration, Water Resources Research 41: W10301.
}%references

\abstract
{Changes of fluvial processes caused by the restoration of an incised mountain stream} % Title
{Wyżga et al.} %short author -- toc
{Bartłomiej Wyżga\textsuperscript{1}*, Maciej Liro\textsuperscript{1}, Paweł Mikuś\textsuperscript{1}, Artur Radecki-Pawlik\textsuperscript{2}, Józef Jeleński\textsuperscript{3},\\Joanna Zawiejska\textsuperscript{4}, Karol Plesiński\textsuperscript{5}} % Author(s)
{\TLtag} % Tag, can be: empty, \KLtag (keynote lecture), \IStag (invited speaker), \CTtag (contributed talk) or \ITtag (invited talk)
{\textsuperscript{1}Institute of Nature Conservation, Polish Academy of Sciences, Kraków, Poland\\
	\textsuperscript{2}Faculty of Civil Engineering, Cracow University of Technology, Kraków, Poland\\
	\textsuperscript{3}‘Upper Raba River Spawning Grounds’ Project Coordinator, Myślenice, Poland\\
	\textsuperscript{4}Institute of Geography, Pedagogical University of Cracow, Kraków, Poland\\
	\textsuperscript{5}Department of Hydraulic Engineering, University of Agriculture in Kraków, Poland
} % Affiliation(s)
{wyzga@iop.krakow.pl}  %e-mail
{channel incision, stream restoration, block ramp, hydraulic modelling, floodwater retention, hydromorphological quality, Polish Carpathians}%keywords
{Construction of a high check dam on mountain Krzczonówka Stream, Polish Carpathians, in the mid-20th century resulted in a number of detrimental changes to the downstream reach. Entrapment of bed material behind the dam caused long-lasting sediment starvation of the downstream reach leading to channel incision and transformation of the former alluvial channel into a bedrock-alluvial or bedrock channel. High flow capacity of the incised channel was reflected in high velocity and bed shear stress associated with flood discharges of given recurrence interval, which prevented in-channel deposition of bed material in case of its delivery from the upstream reach. Concentration of flood flows in the incised channel considerably reduced floodwater retention in the floodplain area, hence contributing to rapid downstream passage of flood waves and increase in their peak discharges. Finally, hydromorphological quality of the stream was degraded as a result of morphological, sedimentary and hydraulic changes in the downstream reach coupled with the disruption of longitudinal stream continuity for aquatic biota caused by the check dam. 
	
	In 2012 a restoration project was initiated to lower the check dam and make the structure passable for fish. To trap the sediment flushed out from the dam reservoir in the incised channel, several block ramps were constructed in 2013, before the onset of the works on the check dam. The check dam was lowered in 2014 and when the works were underway, a 7-year flood occurred on the stream, flushing out a considerable amount of sediment from the dam reservoir. The sediment was efficiently trapped by the block ramps in the downstream reach. This study aims at investigating how the environmental problems caused by the long-term sediment starvation of the stream were mitigated by the restoration works. 
	
	Channel morphology was surveyed after the installation of block ramps but with still unmodified check dam (2013) and after the check-dam lowering (2015). These surveys were done in 10 cross-sections delimited in the downstream reach of the stream. Data about cross-sectional stream morphology, channel slope as well as channel and floodplain roughness were used in hydraulic modelling of flood conditions typifying the stream before (2013) and after (2015) deposition of the sediment trapped by block ramps. The modelling was performed using HEC-RAS software. Moreover, hydromorphological quality of the stream was evaluated in 2012 and 2015 according to the River Hydromorphological Quality method, which is especially suitable for the assessment of effects of river restoration activities (Hajdukiewicz et al., 2017). 
	
	Deposition of the sediment flushed out from above the lowered check dam caused burying of the boulder ramps on the distance of 1.2 km from the dam, whereas the sediment wave reached 1.6 km from the dam. About 15650 m3 of bed material were retained in the stream, resulting in re-establishment of alluvial channel bed and an average aggradation of the channel bed by 0.50 m. Bed aggradation and the resultant increase in the elevation of low-flow water surface were relatively large close to the check dam, attaining maximum values of around 1 m at the distance of 440 m from the dam, and decreased in the downstream direction. As bed aggradation reduced flow capacity of the channel, unit stream power and bed shear stress in the channel zone of the stream decreased, with the largest decrease of these parameters by 36\% and 30\%, respectively, recorded for a 20-year flood. These changes were reflected in reduced competence of the stream, with the average reduction of entrainable grain size of bed material by 18\% for a 2-year flood and by 31\% for the 20-year flood. The reduction in flow capacity of the channel increased retention potential of the floodplain, i.e. a proportion of the total cross-sectional flow area in which floodwater would remain motionless, thus being temporarily retained on the floodplain (Wyżga, 1999; Czech et al., 2016). However, this effect was not statistically significant in the set of 10 study cross-section, but was relatively large where the channel bed aggraded substantially, while small in the cross-sections with a small increase in bed elevation. Before the restoration works, only 1 of the 5 evaluated stream cross-sections was classified as representing good hydromorphological quality, whereas after the works 4 cross-sections fell in this class of hydromorphological quality. The hydromorphological quality improvement mainly reflected changes in bed substrate, erosional and depositional channel features and longitudinal stream connectivity. 
	
	To conclude, inventories performed before and after the restoration works demonstrated effectiveness of block ramps in mitigating problems in the physical functioning of an incised mountain stream. With the entrapment of bed material by block ramps, channel bed considerably aggraded and changed from the bedrock to an alluvial one. The bed aggradation significantly decreased bed shear stress and entrainable grain size of bed material. Floodwater retention in the floodplain area increased, although this effect was largely dependent on the amount of bed aggradation in the study cross-sections. The hydromorphological quality of the stream improved in 4 out of the 5 evaluated cross-sections, with 3 cross-sections being upgraded from moderate to good quality class. 
	
	This study was prepared within the scope of Research Project 2019/33/B/ST10/00518 financed by the National Science Centre of Poland.
}%abstract
{Czech W (2016) Modelling the flooding capacity of a Polish Carpathian river: A comparison of constrained and free channel conditions. Geomorphology 272: 32–42. 
	
Hajdukiewicz H, Wyżga B, Zawiejska J, Amirowicz A, Oglęcki P, Radecki-Pawlik A (2017) Assessment of river hydromorphological quality for restoration purposes: an example of the application of RHQ method to a Polish Carpathian river. Acta Geophysica 65: 423–440. 
	
Wyżga B (1999) Estimating mean flow velocity in channel and floodplain areas and its use for explaining the pattern of overbank deposition and floodplain retention. Geomorphology 28: 281–297. 
}%references