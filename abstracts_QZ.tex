%start of abstract
\abstract
{Flash Flood Simulation in the Urbanised Catchment: A Case Study of Bratislava–Karlova Ves} % Title
{Rusinko and Horáčková} %short author -- toc
{Adam Rusinko\textsuperscript{1}*, Šárka Horáčková\textsuperscript{2}} % Author(s)
{\KLtag} % Tag, can be: empty, \KLtag (keynote lecture), \IStag (invited speaker), \CTtag (contributed talk) or \ITtag (invited talk)
{
	\textsuperscript{1}Faculty of Natural Sciences, Comenius University, Bratislava, Slovakia\\
	\textsuperscript{2}Institute of Geography, Slovak Academy of Sciences, Bratislava, Slovakia
} % Affiliation(s)
{adam.rusinko@uniba.sk}  %e-mail
{Čierny potok stream, flash flood, GRASS GIS, LiDAR, CORINE Land Cover, Bratislava}%keywords
{Flash floods are a dangerous phenomenon that usually affects small drainage basins. They are primarily initiated in the upper parts of the slopes, but their injuring effects are manifested mostly in residential areas, where underground channelized streams disappeared from the surface. Therefore, there are not precise data about stream water levels available and only using surface runoff modelling is possible to simulate what happened during flash floods. Karlova Ves (Bratislava city District), formerly a small viniculture village, was threatened by floods (most probably including those of pluvial type) in the history. In this paper, we simulate surface runoff of a flash flood that occurred in the summer of 2014 in the catchment of Čierny potok using r.sim.water module in GRASS GIS. The flood in August 2014 was reported as with the highest rainfall per hour \textasciitilde40 during the time of local meteorological measurements. Current orthophotomap was used to classify CLC land cover classes, which were assigned the value of the Manning’s roughness coefficient and infiltration rate. The topography was expressed by DTM from high resolution LiDAR data. Our preliminary results indicate that land cover and land use are the essential factors influencing flash floods initiation, although the main driver in lower infiltration and change in flow direction is caused by urbanisation and high proportion of impervious areas. The simulation using r.sim.water module showed that during 60-minute extreme rainfall (40mm/hr) a surface runoff can reach a water depth up to 2 meters in terrain depressions by maximum discharge of 25 cubic meters per second. Natural urban areas revitalisation with increasing the vegetation cover in the areas prone to water flow and accumulation during higher rainfalls helps to prevent the damage caused by floods.
}%abstract
{Alizadehtazi B, DiGiovanni K, Foti R, Morin T, Shetty N H, Montalto F A, Sgurian P L (2016) Comparison of Observed Infiltration Rates of Different Permeable Urban Surfaces Using a Cornell Sprinkle Infiltrometer. Journal of Hydrologic Engineering 21(7): 06016003-1.

Hofierka J, Knutová M (2015) Simulating aspects of a flash flood using the Monte Carlo method and GRASS GIS: a case study of the Malá Svinka Basin (Slovakia). Open Geosciences 7(1): 118-125.

Lapin M, Mikulová K, Pecho J, Šťastný P (2019) Súčasná klimatická charakteristika MČ Bratislava – Karlova Ves a popis scenárov a dopadov zmeny klímy na riešené územie. SHMÚ, Bratislava.

Oťaheľ J, Feranec J, Kopecká M, Falťan V (2017) Modifikácia metódy CORINE Land Cover a legenda pre identifikáciu a zaznamenávanie tried krajinnej pokrývky v mierke 1:10 000 na báze príkladových štúdií z územia Slovenska. Geografický časopis 69(3): 189-224.

Prokešová R, Horáčková Š, Snopková Z (2022) Surface runoff response to long-term land use changes: Spatial rearrangement of runoff-generating areas reveals a shift in flash flood drivers. Science of the Total Environment 815: 151591.

Hydrology in GRASS GIS: A tutorial on hydrological modeling and simulation in GRASS GIS, \url{https://baharmon.github.io/hydrology-in-grass} cited in January 15, 2022

Letecké laserové skenovanie a DMR 5.0, \url{https://www.geoportal.sk/sk/zbgis/lls-dmr} cited in October 12, 2021
}%references
%end of abstract

\abstract
{Fluvial Habitat Assessment Using High-Resolution 3D Models} % Title
{Rusnák et al.} %short author -- toc
{Miloš Rusnák$^1$*, Peter Mihálik$^2$, Ján Sládek$^3$} % Author(s)
{\KLtag} % Tag, can be: empty, \KLtag (keynote lecture), \IStag (invited speaker), \CTtag (contributed talk) or \ITtag (invited talk)
{
	$^1$Institute of Geography SAS, Bratislava, Slovakia\\
	$^2$Faculty of Natural Sciences, Comenius University, Bratislava, Slovakia\\
	$^3$Institute of Geography SAS, Bratislava; GEOTECH Bratislava, Slovakia\\
} % Affiliation(s)
{geogmilo@savba.sk}  %e-mail
{UAV, river channel, bathymetry, classification, fluvial habitats}%keywords
{The paper aims to create a detailed automatic classification of the river landscape habitats based on high-resolution data obtained by drones. We apply automatic classification of the floodplain landscape structure and in-channel physical habitats. As part of automated classification, we test the accuracy of pixel-based and object-oriented supervised classification, using data sets composed of traditional spectral characteristics (RGB) and the geometric properties of the point cloud. The channel bathymetry was identified by modelling the relationship between spectral parameters of the image and field measured water depth or by the correction of the photogrammetrically generated bathymetry model by the refraction coefficient. The accuracy of the automatic classification was evaluated based on KAPPA indices using the validation layer, and the RMSE error was used for bathymetric models. In total, we classify nine main classes of land cover: water; low vegetation (less than 0.5 m); medium vegetation (0.5 - 3 m); high vegetation (more than 3 m); gravel bar; flood facies; bedrock and LWD. The submerged physical morphology was divided into four classes based on water depth. For object-oriented classification on the data layer using only the colour spectrum RGB, the accuracy of vegetation classification was 98.1 \%. The water depth in the riverbed was identified based on a bathymetric model with a determination coefficient of 0.8161 and an RMSE error of 0.2419 m. The physical habitat in the river includes different continuums (grain size, water depth, topographic elevation and flow velocity) as a main physical river habitat parameter constituted by the flow regime (hydrology and hydraulics) and the physical template (fluvial sedimentology and geomorphology). Physical habitat parameters will be extracted from a detailed bathymetry model, reconstructing the channel bed structure, gravel bars and identifying habitat patterns essential for maintaining fish assemblage biodiversity. From hydraulic modelling, we used flow velocity. For final class mapping, we used velocity data, water depth and morphology unit classification (pool, glide, run, riffle).
	
	The research was supported by the Scientific Grant Agency VEGA, number 2/0086/21.
}%abstract
{}%references

\abstract
{Dendrogeomorphic Dating in the Ostrava-Karviná Mining Landscape: Processes, Events, Responses} % Title
{Tichavský et al.} %short author -- toc
{Radek Tichavský\textsuperscript{1}*, Andrea Fabiánová\textsuperscript{1}, Eva Jiránková\textsuperscript{2}, Jan Lenart\textsuperscript{1}, Lucie Polášková\textsuperscript{1}, \\Radim Tolasz\textsuperscript{3}
} % Author(s)
{\TLtag} % Tag, can be: empty, \KLtag (keynote lecture), \IStag (invited speaker), \CTtag (contributed talk) or \ITtag (invited talk)
{
\textsuperscript{1}Department of Physical Geography and Geoecology, Faculty of Science, University of Ostrava, Chittussiho 10, 710 00 Ostrava, Slezská Ostrava, Czech Republic\\
\textsuperscript{2}Institute of Geonics of the Czech Academy of Sciences, Studentská 1768, 708 00 Ostrava, Czech Republic\\
\textsuperscript{3}Czech Hydrometeorological Institute, Na Šabatce 17, 143 06 Praha 4, Czech Republic
} % Affiliation(s)
{radek.tichavsky@osu.cz}  %e-mail
{mining landscape, dendrogeomorphology, subsidence, landslide, compression wood, Ostrava-Karviná mining district}%keywords
{Mining-induced subsidence is a worldwide environmental problem that leads to damage to infrastructure and property; thus, knowledge of its past–recent development is critically needed during land-use planning. We dated the activity of gravitational processes related to mining-induced subsidence using dendrogeomorphic methods at four sites in the Upper Silesian Coal Basin near the former mines of the Karviná mining district. The landslide/subsidence sites were characterised by scarps, rotated blocks, stepped relief with grabens, pressure folds, and shallow movements within landslide toes or within flat undulating relief. Based on 644 increment cores from 161 trees, we were able to identify subsidence/landslide activity, with the most frequent responses occurring from the late 1950s to the early 1970s and from 2005 to 2011. The highest number of tree ring records dates back to 1973. Moreover, more than 30\% of tilted trees from the landslide sites created compression wood at multiple directions indicated complex deformations. At the other sites, the occasional occurrence of single rings with isolated compression wood was included as a possible indicator of events due to frequent coincidence with common series of compression wood. A comparison of tree-ring-based chronology with in situ monitoring revealed good synchronicity with the periods of increased subsidence rates (between 1 and 3 m.year\textsuperscript{−1}). In addition, not only mining operations, but also the occurrence of extreme rainfall seems to be responsible for the recent surface activity. We identified that the 3-year precipitation and the Simple daily intensity index were significantly higher during event years at the site with a flatter topography and clay soils (p=0.02 and 0.01, respectively). Since dendrogeomorphic research on subsidence is influenced by multiple factors, including geological settings, soil conditions, or surface morphology, future research should focus on small research plots that provide detailed tree, surface, and subsurface characteristics. In addition, another challenge is to compare broadleaved and coniferous trees in terms of their sensitivity to recording subsidence movements on the same geomorphic forms.
}%abstract
{}%references
%end of abstract

\abstract
{Impact of Flood Events and Eurasian Beaver (Castor Fiber) Activity on Spontaneous Renaturalization of Łososina River in the Polish Western Carpathians} % Title
{Wąs and Gorczyca} %short author -- toc
{Joanna Wąs\textsuperscript{1}*, Elżbieta Gorczyca\textsuperscript{2}} % Author(s)
{\KLtag} % Tag, can be: empty, \KLtag (keynote lecture), \IStag (invited speaker), \CTtag (contributed talk) or \ITtag (invited talk)
{\textsuperscript{1}Institute of Geography and Spatial Organization, Polish Academy of Sciences\\
	\textsuperscript{2}Institute of Geography and Spatial Management, Jagiellonian University in Cracow} % Affiliation(s)
{joanna.was@zg.pan.krakow.pl}  %e-mail
{spontaneous renaturalization, river channel development, channel migration zone, flood, European Beaver (Castor fiber), Beskid Wyspowy Mts.}%keywords
{Rivers in Polish part of Carpathians are heavily modified. Valleys are urbanized, banks are enforced, channels are straightened and various types of barriers are employed to regulate the flow. All these alterations cause disturbance of natural morphodynamic processes and destruction of habitats. Anthropogenic pressure may also increase risks related to floods and droughts.
	
	Contrary to the trends prevailing in the industrial era, more and more attention is now being paid to the negative effects of river regulation (e.g. Gregory, 2006). Growing awareness of the importance of preserving or restoring the natural state of rivers and water resources for the proper functioning of the environment and people leads to implementation of river restoration projects (e.g. Wohl et al., 2005). However, such actions are expensive and therefore not carried out on a large scale. That is why all factors that may cause renaturalization process to occur without technical treatment deserve special consideration.
	
	Some rivers display tendencies to partially compensate for the disturbances via spontaneous changes of various parameters such as sinuosity and wideness (e.g. Bollati et al., 2014). When left not maintained training structures may deteriorate in time. Eventually such constructions collapses and therefore natural hydromorphological processes may be restored to some extent, especially with external help.
	
	One of the key features characteristic for the multithread natural or semi-natural river is heterogeneity of the flow. Frequently such conditions are not met in regulated part of rivers. To help restore natural state we may employ various expensive technical measures or we can let it be done by the beavers (Castor fiber) which are known for their ability to create heterogeneous (aquatic and terrestrial) habitats (e.g. Rosell et al., 2005).
	
	We examined channel development of gravel-bed river Łososina in Beskid Wyspowy Mts. (Western Carpathians). During this project both of these factors were found to be important for spontaneous renaturalization. Analyses were conducted using data concerning river training made by RZGW (1973–2015), orthophotos, aerial photos, topographic maps (1845-1877, 1963-2018) and field surveys (2018-2019). 
	
	Historically (in XIX century) Łososina had multithread planar course on most of its length and wide channel migration zone. Intensification of the river training in second half of XX century resulted in simplification of planar pattern and shrinking of migration zone. From XX to XXI century riverbed was also incised by 2.5 m. During our study we determined five distinctive river sections that spontaneously increased its sinuosity and migration area despite continuous river training efforts.
	
	So defined process of spontaneous renaturalization took place both gradually and abruptly during research period. The greatest changes in sinuosity coincided in time with flood events. Such processes resulted in development of wandering pattern. While manoeuvring on a greater area river gets access to a wider range and bigger amounts of materials to promote channel heterogeneity and evolution. When it erodes new areas it acquires wood debris, gravel and sand which later becomes foundations for bars, islands, backwaters etc. Newly created sediments becomes unique dynamic habitats for pioneer plants and animals in constant cycle of destruction and recreation.
	
	Beavers occupy almost the entire length of Łososina river but their dams were observed only on the analyzed five sections. In zones of active channel migration water in old abandoned threads is preserved for years by impoundments created by beavers. Furthermore thus created more stagnant waters present new ecological niches.
	
	Though the area covered by those five more dynamic sections is much smaller than the historical extent of active migration it serves as so called “beads on string” (as described by Ward et al., 2002). Such system of retention zones and areas more resilient to disturbances may work as stepping stones for river connected species. Unfortunately due to continuous river management ability to spontaneously restore natural processes in Łososina is restricted.
}%abstract

{Bollati I. M., Pellegrini L., Rinaldi M., Duci G., Pelfini M. (2014), Reach-scale morphological adjustments and stages of channel evolution: The case of the Trebbia River (northern Italy), Geomorphology 221: 176-186. 
	
	Gregory K. J. (2006) The human role in changing river channels, Geomorphology 79(3-4): 172-191.
	
	Rosell F., Bozsér O., Collen P., Parker H. (2005) Ecological impact of beavers Castor fiber and Castor canadensis and their ability to modify ecostystems, Mammal Review 35(3-4): 248-276. 
	
	Ward J. V., Tockner K., Arscott D. B., Claret C. (2002) Riverine landscape diversity, Freshwater Biology, 47(4): 517-539.
	
	Wohl E., Angermeier P.L., Bledsoe B., Kondolf G.M., MacDonnell L., Merritt D.M., Palmer M.A., Poff N.L., Tarboton D. (2005) River restoration, Water Resources Research 41: W10301.
}%references

\abstract
{Changes of Fluvial Processes Caused by the Restoration of
	an Incised Mountain Stream
} % Title
{Wyżga et al.} %short author -- toc
{Bartłomiej Wyżga$^1$*, Maciej Liro$^1$, Paweł Mikuś$^1$, Artur Radecki-Pawlik$^2$, Józef Jeleński$^3$, Joanna Zawiejska$^4$, Karol Plesiński$^5$} % Author(s)
{\KLtag} % Tag, can be: empty, \KLtag (keynote lecture), \IStag (invited speaker), \CTtag (contributed talk) or \ITtag (invited talk)
{$^1$Institute of Nature Conservation, Polish Academy of Sciences, Kraków, Poland\\
	$^2$Faculty of Civil Engineering, Cracow University of Technology, Kraków, Poland\\
	$^3$‘Upper Raba River Spawning Grounds’ Project Coordinator, Myślenice, Poland\\
	$^4$Institute of Geography, Pedagogical University of Cracow, Kraków, Poland\\
	$^5$Department of Hydraulic Engineering, University of Agriculture in Kraków, Poland
} % Affiliation(s)
{wyzga@iop.krakow.pl}  %e-mail
{channel incision, stream restoration, block ramp, hydraulic modelling, floodwater retention, hydromorphological quality, Polish Carpathians}%keywords
{Construction of a high check dam on mountain Krzczonówka Stream, Polish Carpathians, in the mid-20th century resulted in a number of detrimental changes to the downstream reach. Entrapment of bed material behind the dam caused long-lasting sediment starvation of the downstream reach leading to channel incision and transformation of the former alluvial channel into a bedrock-alluvial or bedrock channel. High flow capacity of the incised channel was reflected in high velocity and bed shear stress associated with flood discharges of given recurrence interval, which prevented in-channel deposition of bed material in case of its delivery from the upstream reach. Concentration of flood flows in the incised channel considerably reduced floodwater retention in the floodplain area, hence contributing to rapid downstream passage of flood waves and increase in their peak discharges. Finally, hydromorphological quality of the stream was degraded as a result of morphological, sedimentary and hydraulic changes in the downstream reach coupled with the disruption of longitudinal stream continuity for aquatic biota caused by the check dam. 
	
	In 2012 a restoration project was initiated to lower the check dam and make the structure passable for fish. To trap the sediment flushed out from the dam reservoir in the incised channel, several block ramps were constructed in 2013, before the onset of the works on the check dam. The check dam was lowered in 2014 and when the works were underway, a 7-year flood occurred on the stream, flushing out a considerable amount of sediment from the dam reservoir. The sediment was efficiently trapped by the block ramps in the downstream reach. This study aims at investigating how the environmental problems caused by the long-term sediment starvation of the stream were mitigated by the restoration works. 
	
	Channel morphology was surveyed after the installation of block ramps but with still unmodified check dam (2013) and after the check-dam lowering (2015). These surveys were done in 10 cross-sections delimited in the downstream reach of the stream. Data about cross-sectional stream morphology, channel slope as well as channel and floodplain roughness were used in hydraulic modelling of flood conditions typifying the stream before (2013) and after (2015) deposition of the sediment trapped by block ramps. The modelling was performed using HEC-RAS software. Moreover, hydromorphological quality of the stream was evaluated in 2012 and 2015 according to the River Hydromorphological Quality method, which is especially suitable for the assessment of effects of river restoration activities (Hajdukiewicz et al., 2017). 
	
	Deposition of the sediment flushed out from above the lowered check dam caused burying of the boulder ramps on the distance of 1.2 km from the dam, whereas the sediment wave reached 1.6 km from the dam. About 15650 m3 of bed material were retained in the stream, resulting in re-establishment of alluvial channel bed and an average aggradation of the channel bed by 0.50 m. Bed aggradation and the resultant increase in the elevation of low-flow water surface were relatively large close to the check dam, attaining maximum values of around 1 m at the distance of 440 m from the dam, and decreased in the downstream direction. As bed aggradation reduced flow capacity of the channel, unit stream power and bed shear stress in the channel zone of the stream decreased, with the largest decrease of these parameters by 36\% and 30\%, respectively, recorded for a 20-year flood. These changes were reflected in reduced competence of the stream, with the average reduction of entrainable grain size of bed material by 18\% for a 2-year flood and by 31\% for the 20-year flood. The reduction in flow capacity of the channel increased retention potential of the floodplain, i.e. a proportion of the total cross-sectional flow area in which floodwater would remain motionless, thus being temporarily retained on the floodplain (Wyżga, 1999; Czech et al., 2016). However, this effect was not statistically significant in the set of 10 study cross-section, but was relatively large where the channel bed aggraded substantially, while small in the cross-sections with a small increase in bed elevation. Before the restoration works, only 1 of the 5 evaluated stream cross-sections was classified as representing good hydromorphological quality, whereas after the works 4 cross-sections fell in this class of hydromorphological quality. The hydromorphological quality improvement mainly reflected changes in bed substrate, erosional and depositional channel features and longitudinal stream connectivity. 
	
	To conclude, inventories performed before and after the restoration works demonstrated effectiveness of block ramps in mitigating problems in the physical functioning of an incised mountain stream. With the entrapment of bed material by block ramps, channel bed considerably aggraded and changed from the bedrock to an alluvial one. The bed aggradation significantly decreased bed shear stress and entrainable grain size of bed material. Floodwater retention in the floodplain area increased, although this effect was largely dependent on the amount of bed aggradation in the study cross-sections. The hydromorphological quality of the stream improved in 4 out of the 5 evaluated cross-sections, with 3 cross-sections being upgraded from moderate to good quality class. 
	
	This study was prepared within the scope of Research Project 2019/33/B/ST10/00518 financed by the National Science Centre of Poland.
}%abstract
{Czech W (2016) Modelling the flooding capacity of a Polish Carpathian river: A comparison of constrained and free channel conditions. Geomorphology 272: 32–42. 
	
	Hajdukiewicz H, Wyżga B, Zawiejska J, Amirowicz A, Oglęcki P, Radecki-Pawlik A (2017) Assessment of river hydromorphological quality for restoration purposes: an example of the application of RHQ method to a Polish Carpathian river. Acta Geophysica 65: 423–440. 
	
	Wyżga B (1999) Estimating mean flow velocity in channel and floodplain areas and its use for explaining the pattern of overbank deposition and floodplain retention. Geomorphology 28: 281–297. 
}%references