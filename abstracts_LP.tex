%start of conference contribution
\abstract
% Title 
{Human intervention and flood events as key factors affecting the recent degradation of the Hornád River} 
% EndOfTitle
%short author -- toc 
{Labaš and Kidová} 
%End short author -- toc
% Author(s) 
{Peter Labaš\textsuperscript{1}*, Anna Kidová\textsuperscript{1}} 
% EndOfAuthor(s)
{\KLtag} 
%Tag, can be: empty, \KLtag (keynote lecture), \IStag (invited speaker), \CTtag (contributed talk) or \ITtag (invited talk)
% Affiliation(s)
{
\textsuperscript{1}Department of Physical Geography, Geomorphology and Natural Hazards, Institute of Geography, Slovak Academy of Science, Štefánikova 49, 814 73 Bratislava
}
%} % EndOfAffiliation(s)
{geoglaba@savba.sk}  %e-mail
%keywords
{meandering; river degradation; multi-temporal analysis; flood event; the Hornád River}
%EndOfKeywords
%abstract content
{At the turn of the 19th and 20th century, the Hornád River in Slovakia in its middle reach was typical by passing the Hornád basin, typical by several levels of river terraces (Michaeli 2001), well-developed floodplains and water-gaps. The wide floodplains were characteristic by free meanders which had been in the 20th century affected by human interventions, peaked in the 50s. The anthropogenic impact resulted in channel shortening and narrowing, river sinuosity lowering, erosion-accumulation processes decreasing and free meanders loss. Additionally, the long-term discharge reduction had caused simplification of channel planform in stream sections without or with minimal anthropogenic impact. This trend was later abrupted by the flood event series in 2004, 2008 and 2010. The flood events were also documented as an important factor in a morphological change in the lower part of the Hornád River in Hungary (Kiss and Blanka, 2011; Kiss and Blanka, 2012). Environmentally changed conditions (anthropogenic impact and floods) on the 72 km long channel planform of the meandering Hornád River in Slovakia were observed on three types of river segments (natural, regulated, water-gap) by seven sets of data, including the second and third military survey and five orthophoto mosaics and aerial photos (1949, 1986, 2002, 2013, 2016). The Hornád River in the pre-regulation period was represented by a natural meandering river planform (45.8\%) with a high occurrence of in-channel landforms, where the lateral bar area prevailed. At present, due to the simplification of the river channel planform, only 26\% of the river segments with ongoing natural erosion-accumulation processes and the last two locations with free meanders on the middle reach of the Hornád River remained.

\textit{This research was supported by the Science Grant Agency (VEGA) of the Ministry of Education of the Slovak Republic and the Slovak Academy of Sciences (02/0086/21).}
}
%EndOfAbstractContent
%references
{Kiss, T. and Blanka, V. (2011) Effect of different water stages on bank erosion, case study on River Hernád, Hungary, Carpathian Journal of Earth and Environmental Sciences, 6(2), pp. 101-108

Kiss, T. and Blanka, V. (2012) River channel response to climate- and human-induced hydrological changes: Case study on the meandering Hernád River, Hungary, Geomorphology, 175–176, pp. 115–125. \doi{10.1016/j.geomorph.2012.07.003}.

Labaš, P. and Kidová, A., (2022). Anthropogenic and environmental impacts on the recent morphological degradation of the meandering Hornád River, Geografický časopis,  74(2), in press

Michaeli, E. (2001). Georeliéf hornádskej kotliny, Geografické práce, 9(2), 153 p.
}
%EndOfReferences
%end of conference contribution

%---------------------------------------------------------------------------------------------------
%start of abstract
\abstract
{Geomorphic-Sedimentary Adjustment of a River Reach With Groynes to Channel Bypassing} % Title
{Lehotský et al.} %short author -- toc
{Milan Lehotský\textsuperscript{1}*, Miloš Rusnák\textsuperscript{1}, Šárka Horáčková\textsuperscript{1}, 
	Tomáš Štefaničk\textsuperscript{2}, Jaroslav Kleň\textsuperscript{3}} % Author(s)
{\KLtag} % Tag, can be: empty, \KLtag (keynote lecture), \IStag (invited speaker), \CTtag (contributed talk) or \ITtag (invited talk)
{\textsuperscript{1}Department of Physical Geography, Geomorphology and Natural Hazards, Institute of Geography, Slovak Academy of Sciences, Štefánikova 49, 814 73 Bratislava, Slovakia\\
\textsuperscript{2}Department of Theoretical Geodesy, Slovak University of Technology in Bratislava, Radlinského 11, Block A, 810 05 Bratislava, Slovakia\\
\textsuperscript{3}Data Intelligence Engineer, DELL Technologies, Fazuľová 7, 81107 Bratislava, Slovakia 
} % Affiliation(s)
{geogleho@savba.sk}  %e-mail
{Gabcikovo Waterworks; spatio-temporal variability; groyne-induced bench; vertical accretion; conveyance; flood; Danube }%keywords
{The article is focused on the investigation of spatio-temporal variability of the vertical accretion thickness as the response of the Danube River reach to bypassing. Five groyne-induced benches (GIB) of the bypassed channel were developed after water diversion in 1992 and represent our study area. Their topography was created from LiDAR point cloud dataset and DEM models for tree time spans (for original gravel surface, for surface before flood 2013, surface after flood 2013) were calculated and the allostratigraphic approach was applied on 548 drilling probes at five GIB cross-sections.  Head, supra-platform, tail and back channel geomorphic units have been identified at each GIB. The accretion was influenced mostly by large flood events when the 100-yr contributed to the its total volume by almost 26\%. The head geomorphic unit, as the main impact zone on the stream, exhibits the highest median values of vertical accretion in time and space. The median of the vertical accretion thickness does not decrease with height above mean channel water level and the thickness of accretion varied likely due to variability of the vegetation cover conditioning variable hydraulic conditions. The bend setting, lower radius of curvature, lower width-depth ratio, connectivity with side arm and well-developed alternate bench upstream conditioned higher vertical accretion of the benches and its slight migration downstream. Moreover, higher portion of the area was formed as backwater geomorphic unit. The comparison of sediment depth over time spans (22 and 20 years and a large flood event) allowed us to conclude that the it is spatially variable by individual GIB, however its sedimentation trend over time is the same.
	
	\noindent	
	\textbf{Ackowledgements:} This research was supported by the Scientific Grant Agency of the Ministry of Education, Science and Sport of the Slovak Republic (VEGA 2/0086/21). We appreciate LiDAR data provided by National Forest Centre in Zvolen.
}%abstract
%references
%end of abstract

%start of conference contribution
\abstract
% Title 
{Erosion Process and Abandonement as Main Drivers of Traditional Vineyards Degradation: A Case Study Vráble Viticulture District, Slovakia} 
% EndOfTitle
%short author -- toc 
{Lieskovský and Kenderessy} 
%End short author -- toc
% Author(s) 
{Juraj Lieskovský\textsuperscript{1}*, Pavol Kenderessy\textsuperscript{1}} 
% EndOfAuthor(s)
{\KLtag} 
%Tag, can be: empty, \KLtag (keynote lecture), \IStag (invited speaker), \CTtag (contributed talk) or \ITtag (invited talk)
% Affiliation(s)
{
\textsuperscript{1}Institute of Landscape Ecology, Slovak Academy of Sciences, Bratislava, Slovak Republic
}
%} % EndOfAffiliation(s)
{juraj.lieskovsky@savba.sk}  %e-mail
%keywords
{erosion, deposition, erosion measurements, levelling method, vineyards, abandonment}
%EndOfKeywords
%abstract content
{We evaluated the effect of increased erosion events and decreased management on degradation of traditional vineyards near Vráble (Slovakia). For the erosion measurements we used poles height method, that uses vineyards poles as a passive marker. To assess the accuracy of the poles height method, we used the measurements on vineyard located on relatively flat area. The measurements were first performed between 2008 and 2011 (Lieskovský and Kenderessy, 2014) and repeated after 10 years in 2021.

To evaluate the degree of abandonment, we visited all 633 fields and mapped the land cover classes and management practices. 

The erosion rates on ploughed vineyards were doubled in period 2010-2011, but the total soil loss decreased due to the increased deposition. The erosion rates on vineyard that has been hoed and later covered by grass did not changed due to the protective effect of the grass cover. The land cover and management practices has been dramatically changed in last decade. Only 27.2\% of vineyard area near Vráble is now tilled and potentially threatened by soil erosion. On the other hand, 35.7\% of vineyards is abandoned and it seems that abandonment will also continue within next years. The results also show that the most significant factors causing the degradation of traditional vineyards is not soil erosion anymore, but decreasing management that leads to the abandonment.
}
%EndOfAbstractContent
%references
{Lieskovský, J., \& Kenderessy, P. (2014). Modelling the effect of vegetation cover and different tillage practices on soil erosion in vineyards: a case study in Vráble (Slovakia) using WATEM/SEDEM. Land Degradation \& Development, 25(3), 288-296.
}
%EndOfReferences
%end of conference contribution

%---------------------------------------------------------------------------------------------------



\abstract
{Using Airborne Lidar Data for Detecting Anthropogenic Landforms in Poiplie Region} % Title
{Lieskovský et al.} %short author -- toc
{Juraj Lieskovský\textsuperscript{1}*, Dana Lieskovská\textsuperscript{1}, Tibor Lieskovský\textsuperscript{2}, Petra Gašparovičová\textsuperscript{1}} % Author(s)
{\KLtag} % Tag, can be: empty, \KLtag (keynote lecture), \IStag (invited speaker), \CTtag (contributed talk) or \ITtag (invited talk)
{\textsuperscript{1}Institute of Landscape Ecology, Slovak Academy of Sciences, Bratislava, Slovakia\\
\textsuperscript{2}Department of Theoretical Geodesy and Geoinformatics, Slovak University of Technology, Bratislava, Slovakia
} % Affiliation(s)
{juraj.lieskovsky@savba.sk}  %e-mail
{LiDAR, landscape palimpsest, Kiarov, Kováčovce, landscape archaeology, anthropogenic landforms}%keywords
{We present the use of Light Detection and Ranging  (LiDAR) data for mapping the anthropogenic landforms in Kiarov and Kováčovce villages, located in Poiplie region. The benefits of LiDAR are the speed of data acquisition on a landscape scale, the quality of its surface data, and its ability to identify macro- and micro-topographical features that are otherwise undetected by terrestrial surveys. LiDAR is now used for detailed topographical research in many disciplines including geomorphology. The area of Slovakia is being scanned from 2017 at a resolution of 20–30 points per square metre and is scheduled to be completed by 2023. 
	
	For the interpolation of the LiDAR points to digital elevation model with resolution 25cm, we used points classified as ground and buildings. We applied the visualisation techniques based on a simple local relief model (Kokalj \& Hesse, 2017) but we also added a local dominance to relief curvature representation and also added a combination of slope steepness and sky view factor to relief contrast visualisation (Lieskovský et al., in press). We visually identified the topographic signatures of geomorphic anthropogenic landforms and consulted our findings with historical maps and aerial images. We followed the anthropogenic landform ontology proposed by Tarolli et al. (2019) and we categorized the landforms in the following categories: (1) ‘symbolic’ were represented by cemeteries and chapel; (2) ‘habitation’ were represented by habitation areas and dispersed settlements; (3) ‘transport and exchange’ were represented by paved and unpaved roads, including traces of roads that are not used anymore; (4) ‘subsistence’ were represented by traditional and socialistic agricultural terraces, house gardens, agricultural enterprises, vineyards and orchards and  hedgerows; (5) ‘mining’ were represented by surface mining pits (6) ‘water infrastructure’ consisted of melioration canals and artificial river bed of Ipel river; (7) ‘waste disposal’ was found in central part of Kováčovce cadastral area; and (8) ‘warfare infrastructure’ was represented by trenches and ditches from the Second World War.
}%abstract
{Kokalj Ž, Hesse R (2017) Airborne Laser Scanning Raster Data Visualization. Vol. 14. Prostor, Kraj, Čas. ZRC SAZU, Založba ZRC, Ljubljana
	
	Lieskovský J, Lieskovský T, Hladíková K, Štefunková D, Hurajtová N (in press) Potential of Airborne LiDAR Data in Detecting Cultural Landscape Features in Slovakia. Landscape Research.
	
	Tarolli P, Wenfang C, Giulia S, Damian E, and Erle C-E(2019) From Features to Fingerprints: A General Diagnostic Framework for Anthropogenic Geomorphology. Progress in Physical Geography: Earth and Environment 43, no. 1: 95–128.}%references
%start of conference contribution
\abstract
% Title 
{Land Surface Temperature Controls Stability in Gentle Clay Slopes} 
% EndOfTitle
%short author -- toc 
{Loche et al.} 
%End short author -- toc
% Author(s) 
{Marco Loche\textsuperscript{1}*, Gianvito Scaringi\textsuperscript{1}, Luigi Lombardo\textsuperscript{2}} 
% EndOfAuthor(s)
{\KLtag} 
%Tag, can be: empty, \KLtag (keynote lecture), \IStag (invited speaker), \CTtag (contributed talk) or \ITtag (invited talk)
% Affiliation(s)
{
\textsuperscript{1}Faculty of Science, Charles University, Prague, Czech Republic\\
\textsuperscript{2}Faculty of Geo-Information Science and Earth Observation (ITC), University of Twente, Enschede, Netherlands
}
%} % EndOfAffiliation(s)
{marco.loche@natur.cuni.cz}  %e-mail
%keywords
{Slow flows; land surface temperature; landslide susceptibility; slope unit; clay}
%EndOfKeywords
%abstract content
{The effect of temperature on the stability of slopes in temperate climates is poorly constrained. Experiments demonstrate a clear thermo-hydro-mechanical (THM) response in expansive soils, and evidence of thermally-induced activity exists for some landslides (Scaringi and Loche, 2022). However, building a representative thermal variable suitable for catchment or regional-scale studies is challenging, owing to heterogeneities in materials and stress histories and the complexity of THM processes (Loche et al., 2022. We performed a landslide susceptibility modelling of the portion of Italian territory featuring clay deposits. We utilised the geo-lithological map of Italy and the Italian National Inventory (IFFI), that differentiates among landslide types, and we focused on slow flows, often associated with creep movements. We relied, as one of the inputs, on a ten-year series of Land Surface Temperature (LST) data from MODIS, freely available in Google Earth Engine, and implemented a slope unit-based Generalized Additive Model (GAM) approach to account for nonlinearities in the possible temperature-slope stability relationship (Alvioli et al., 2016). We produced a susceptibility map for clay deposits over the entire Italian territory and observed a positive dependence of landslide abundance on LST on warmer and gentle slopes, where creep phenomena are common. Higher temperatures are in fact associated with decreased soil and water viscosity and hence enhanced shear creep rates in clays.
}
%EndOfAbstractContent
%references
{Scaringi, G., \& Loche, M. (2022). A thermo-hydro-mechanical approach to soil slope stability under climate change. Geomorphology, 108108.

Loche, M., Scaringi, G., Yunus, A. P., Catani, F., Tanyaş, H., Frodella, W., ... \& Lombardo, L. (2022). Surface temperature controls the pattern of post-earthquake landslide activity. Scientific reports, 12(1), 1-11.

Alvioli et al. (2016) Automatic delineation of geomorphological slope units with r. slopeunits v1.0 and their optimization for landslide susceptibility modeling. Geoscientific Model Development, 9(11), pp.3975-3991.
	
}
%EndOfReferences
%end of conference contribution

%---------------------------------------------------------------------------------------------------

%start of abstract
\abstract
{Long-Term Monitoring of the Recruitment and Dynamics of Large Wood in Kamienica Stream, Polish Carpathians} % Title
{Mikuś and Wyżga} %short author -- toc
{Paweł Mikuś\textsuperscript{1}*, Bartłomiej Wyżga\textsuperscript{1}} % Author(s)
{\TLtag} % Tag, can be: empty, \KLtag (keynote lecture), \IStag (invited speaker), \CTtag (contributed talk) or \ITtag (invited talk)
{\textsuperscript{1}Institute of Nature Conservation, Polish Academy of Sciences, Kraków, Poland
} % Affiliation(s)
{mikus@iop.krakow.pl}  %e-mail
{large wood, wood dynamics, wood monitoring, wood inventory, Polish Carpathians}%keywords
{Quantifying wood delivery and mobility in small mountain streams requires long-term and repeatable observations, so far very scarcely described. Recently, observations using a number of remote sensing methods have gained popularity. However, classical methods of fieldwork still seem to be indispensable in more detailed studies. Such observations were conducted on the length of 8.7 km of the upper course of Kamienica Stream, Polish Carpathians, where recent bark beetle infestation of riparian spruce forest might have considerably increased the delivery of fallen trees to the channel. This part of the stream course is located in the Gorce Mountains National Park and large wood is not removed from the stream under the national park regulations.

In October 2009, numbered metal plates were installed on 429 trees growing along three stream sections located 2500–2950 m (section A), 4000–4450 m (section B) and 7850–8300 m (section C) from the stream source. Different metals were used in each section to allow for finding tagged trees with a metal detector in case the plates on them are inaccessible. The monitoring of standing and fallen trees tagged with metal plates has been conducted a few times per year, especially after heavy rainfall and windstorms. Moreover, the mode of location and the degree of decay of wood pieces stored in the study reach were determined in 2012.

During twelve years of observations, 111 trees (26\% of the tagged sample) were supplied to the stream as a result of bank erosion, windthrow of living trees or those killed by bark beetle infestation, snow overload and landslides. In October 2009, 80 cm of wet snow fell in two days and snow overload caused breaking of two tagged trees. Five events with high water stage occurred in May 2010, May 2014, July 2016, May 2018 and May 2019. Those from 2016 and 2018 can be considered as major floods as they caused significant channel changes and damage to local infrastructure. About half of trees supplied to the channel were not transported, and numerous wood dams occurring in the stream limited transport of any fallen trees. Forty-six fallen tagged trees (48\%) were transported during some of the five floods. In sections A and B, mean distance of the displacement of tagged trees over the study period was small and did not exceed 32 m, whereas in section C it was a few times longer. As a result of the flood in 2010, three trees were displaced relatively short distances (Mean = 42 m, Max = 100 m) and retained in in-channel jams. A definitely larger flood from 2014 was marked only in the lowermost section C, where the wood occurring in the channel was crushed into small pieces and flushed out downstream. The flood of July 2016 was the only major flood in all study sections; during this event 11 trees were displaced a mean distance of 97 m (Max = 230 m). In May 2018, a major flood caused by heavy rainfall resulted in considerable bank erosion in the lowest study section. During this event, 41 trees from this section were recruited to the channel and transported (Mean = 275 m, Max = 1003 m), while no transport occurred in the other sections. As trees in the upstream sections remained practically untouched, the course of this flood showed a strongly localized occurrence of the triggering rainfall. A small flood in May 2019 did not displace any tagged trees occurring in the stream. 

A large wood inventory performed in 2012 indicated that in the second-order reach wood was relatively uniformly distributed among different location types (Wyżga et al., 2015), with logs with their top or bottom located in the channel being the most abundant (21\% of all wood pieces). Moreover, near-perpendicular orientation of wood pieces in relation to the channel axis prevailed in the reach (Mikuś et al., 2016). All this indicates a negligible role of transport and redistribution of wood, which was predominantly retained where it fell. This reach was typified by the largest proportion of logs forming wood dams (12\%) and spanning the channel (10\%) among the study reaches, which can be attributed to the relatively large length of wood pieces in relation to channel width. The third-order reach was characterized by a similar pattern of wood location as the second-order reach. It was distinguished by only one feature: a small amount of wood spanning the channel. In the fourth-order reach, large wood spanning the channel was very scarce because of larger stream width and considerably larger flood discharges. Here, large wood was predominantly retained along channel margins (41\%), on gravel bars (19\%), and with only the top or bottom located in the channel (17\%). This reach was typified by variable orientation of wood pieces in relation to the channel axis, with a proportion of the pieces being apparently reoriented by the stream current. Most of wood pieces were shorter than channel width, as they originated from the breakage of trees into smaller fragments during their fall to the channel or transport by flood flows. 

The inventory also indicated that in the second-order reach of Kamienica, 16\% of wood pieces were in a relatively good condition, representing class 1 and 2 of wood decay, whereas a more advanced degree of decomposition, typical of class 3 and 4, typified 84\% of pieces. In the third-order stream reach, this distribution was more even, with 41\% of pieces representing class 1 and 2 of wood decay, and 59\% class 3 and 4. In the fourth-order reach, classes 1 and 2 constituted 31\% of all wood pieces, and 69\% had typical features of classes 3 and 4.

To conclude, large wood is recruited to the upper course of Kamienica Stream by a few processes, with bank erosion and windthrow having been most effective during 12 years of monitoring. Large wood was recruited to the stream only during high-intensity meteorological and hydrological events. With 22\% of tagged trees recruited to the channel during 12 years, the rate of turnover of the riparian trees was estimated at 45 years. As the riparian area supports trees with ages up to ~160 years, the rate evidences substantial intensification of large wood recruitment to the channel in the recent period resulting from bark beetle infestation of riparian trees. The mobility of wood in the stream increases downstream because of increasing flood discharges and the decreasing ability of fallen trees to anchor on the banks of increasingly wide channel. Wood is transported longer distances only during major floods, which tend to deposit wood pieces along channel margins and on gravel bars, where wood is subjected to relatively rapid decomposition under subaerial conditions. During a subsequent large flood, most wood pieces already occurring in the channel are thus likely to rapidly disintegrate, rather than being flushed out downstream. Thus, large wood retained in the upper stream course does not constitute an important flood hazard to downstream, inhabited valley reaches.
}%abstract

{Wyżga B, Zawiejska J, Mikuś P, Kaczka RJ (2015) Contrasting patterns of wood storage in mountain watercourses narrower and wider than the height of riparian trees, Geomorphology, 228, 275-285.  

Mikuś P, Wyżga B, Ruiz-Villanueva V, Zawiejska J, Kaczka RJ, Stoffel M (2016) Methods to assess large wood dynamics and the associated flood hazard in Polish Carpathian watercourses of different size. In: Kundzewicz ZW, et al. (eds.), Flood Risk in the Upper Vistula Basin. Springer, Cham, pp. 77-101.
}%references
%end of abstract

\abstract
{Large landslides along the eastern margin of Patagonian Ice Sheet: distribution, causes and timing} % Title
{Pánek et al.} %short author -- toc
{Tomáš Pánek\textsuperscript{1}*, Michal Břežný\textsuperscript{1}, Elisabeth Schönfeldt\textsuperscript{2}, Veronika Kapustová\textsuperscript{1}, Diego Winocur\textsuperscript{3}, Rachel Smedley\textsuperscript{4}} % Author(s)
{\KLtag} % Tag, can be: empty, \KLtag (keynote lecture), \IStag (invited speaker), \CTtag (contributed talk) or \ITtag (invited talk)
{\textsuperscript{1}University of Ostrava, Department of Physical Geography and Geoecology, Chittussiho 10, Slezská Ostrava, Czech Republic \\
\textsuperscript{2}University of Potsdam, Institute of Geosciences, Karl-Liebknecht-Straße 24-25, 14476, Potsdam, Germany \\
\textsuperscript{3}Universidad de Buenos Aires, Facultad de Ciencias Exactas y Naturales, Departamento de Ciencias Geologicas, Intendente Güiraldes 2416, C1428EGA, CABA, Argentina \\
\textsuperscript{4}University of Liverpool, Department of Geography and Planning, Chatham Street, Liverpool, L69 7ZT, UK} % Affiliation(s)
{tomas.panek@osu.cz}  %e-mail
{}%keywords
{Although ice retreat is widely considered to be an important factor in landslide origin, many links between deglaciation and slope instabilities are yet to be discovered. Here we focus on the origin and chronology of exceptionally large landslides situated along the eastern margin of the former Patagonian Ice Sheet (PIS). Accumulations of the largest rock avalanches in the former PIS territory are concentrated in the Lago Pueyrredón valley at the eastern foothills of the Patagonian Andes in Argentina. Long-runout landslides have formed along the rims of sedimentary and volcanic mesetas, but also on the slopes of moraines from the Last Glacial Maximum. At least two rock avalanches have volumes greater than 1 km\textsuperscript{3} and many other landslide accumulations have volumes in the order of tens to hundreds of million m\textsuperscript{3}. Using cross-cutting relationships with glacial and lacustrine sediments and using OSL and \textsuperscript{14}C dating, we found that the largest volume of landslides occurred between \textasciitilde17 and \textasciitilde11 ka. This period coincides with the most rapid phase of PIS retreat, the greatest intensity of glacial isostatic uplift, and the existence of a dropping glacial lake along the foothills of the Patagonian Andes. The position of paleoshorelines in the landslide bodies and, in many places, folded and thrusted lacustrine sediments at the contact with rock avalanche deposits indicate that the landslides collapsed directly into the glacial lake. Although the landslides along the former glacial lobe of Lago Pueyrredón continue today, they are at least an order of magnitude smaller than the rock and debris avalanches that occurred before the drainage of the glacial lake around 10-11 ka. Preliminary numerical modeling results indicate that large postglacial landslides may have been triggered by a combination of rapid sequential glacial lake drawdowns and seismicity due to glacial isostatic adjustment. We conclude that in addition to direct links such as glacial oversteepening, debuttressing and permafrost degradation, the retreat of ice sheets and the subsequent formation of transient large glacial lakes can fundamentally alter stability conditions, especially if the slopes are built by weak sedimentary and volcanic rocks.}%abstract
{}%references