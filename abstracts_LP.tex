%start of conference contribution
\abstract
% Title 
{Human intervention and flood events as key factors affecting the recent degradation of the Hornád River} 
% EndOfTitle
%short author -- toc 
{Labaš and Kidová} 
%End short author -- toc
% Author(s) 
{Peter Labaš\textsuperscript{1}*, Anna Kidová\textsuperscript{1}} 
% EndOfAuthor(s)
{\TLtag} 
%Tag, can be: empty, \KLtag (keynote lecture), \IStag (invited speaker), \CTtag (contributed talk) or \ITtag (invited talk)
% Affiliation(s)
{
\textsuperscript{1}Department of Physical Geography, Geomorphology and Natural Hazards, Institute of Geography, Slovak Academy of Science, Štefánikova 49, 814 73 Bratislava
}
%} % EndOfAffiliation(s)
{geoglaba@savba.sk}  %e-mail
%keywords
{meandering; river degradation; multi-temporal analysis; flood event; the Hornád River}
%EndOfKeywords
%abstract content
{At the turn of the 19\textsuperscript{th} and 20\textsuperscript{th} century, the Hornád River in Slovakia in its middle reach was typical by passing the Hornád basin, typical by several levels of river terraces (Michaeli 2001), well-developed floodplains and water-gaps. The wide floodplains were characteristic by free meanders which had been in the 20\textsuperscript{th} century affected by human interventions, peaked in the 50s. The anthropogenic impact resulted in channel shortening and narrowing, river sinuosity lowering, erosion-accumulation processes decreasing and free meanders loss. Additionally, the long-term discharge reduction had caused simplification of channel planform in stream sections without or with minimal anthropogenic impact. This trend was later abrupted by the flood event series in 2004, 2008 and 2010. The flood events were also documented as an important factor in a morphological change in the lower part of the Hornád River in Hungary (Kiss and Blanka, 2011; Kiss and Blanka, 2012). Environmentally changed conditions (anthropogenic impact and floods) on the 72 km long channel planform of the meandering Hornád River in Slovakia were observed on three types of river segments (natural, regulated, water-gap) by seven sets of data, including the second and third military survey and five orthophoto mosaics and aerial photos (1949, 1986, 2002, 2013, 2016). The Hornád River in the pre-regulation period was represented by a natural meandering river planform (45.8\%) with a high occurrence of in-channel landforms, where the lateral bar area prevailed. At present, due to the simplification of the river channel planform, only 26\% of the river segments with ongoing natural erosion-accumulation processes and the last two locations with free meanders on the middle reach of the Hornád River remained.

\vspace{0.5em}
\noindent
\textbf{Acknowledgements:}
\textit{This research was supported by the Science Grant Agency (VEGA) of the Ministry of Education of the Slovak Republic and the Slovak Academy of Sciences (02/0086/21).}
}
%EndOfAbstractContent
%references
{Kiss, T. and Blanka, V. (2011) Effect of different water stages on bank erosion, case study on River Hernád, Hungary, Carpathian Journal of Earth and Environmental Sciences, 6(2), pp. 101-108

Kiss, T. and Blanka, V. (2012) River channel response to climate- and human-induced hydrological changes: Case study on the meandering Hernád River, Hungary, Geomorphology, 175–176, pp. 115–125. \doi{10.1016/j.geomorph.2012.07.003}.

Labaš, P. and Kidová, A., (2022). Anthropogenic and environmental impacts on the recent morphological degradation of the meandering Hornád River, Geografický časopis,  74(2), in press

Michaeli, E. (2001). Georeliéf hornádskej kotliny, Geografické práce, 9(2), 153 p.
}
%EndOfReferences
%end of conference contribution

%---------------------------------------------------------------------------------------------------

\abstract
{Identification of benches and ledges within the braided-wandering floodplain formation} % Title
{Labaš et al.} %short author -- toc
{Peter Labaš\textsuperscript{1}*, Anna Kidová\textsuperscript{1}, Šárka Horáčková\textsuperscript{1}, Milan Lehotský\textsuperscript{1}, Miloš Rusnák\textsuperscript{1}
} % Author(s)
{\TLtag} % Tag, can be: empty, \KLtag (keynote lecture), \IStag (invited speaker), \CTtag (contributed talk) or \ITtag (invited talk)
{\textsuperscript{1}Department of Physical Geography, Geomorphology and Natural Hazards, Institute of Geography, Slovak Academy of Science, Štefánikova 49, 814 73 Bratislava} % Affiliation(s)
{geoglaba@savba.sk}  %e-mail
{braided-wandering river; floodplain evolution; benches; ledges; gravel-bed river}%keywords
{The Belá River in the north part of Slovakia represents a braided-wandering river system (Kidová et al., 2016) with repeatedly destroying and re-forming floodplain, as a result of lateral shifting of an active river zone of braided rivers (Haschenburger and Cowie, 2009). These processes depend on flood events and a large volume of sediment transportation. Benches as unstable (Erskine and Livingstone, 1999), fine sediment storage (Kemp, 2004; Vietz et al., 2005a, 2005b) are formed by flood events (Webb, Erskine a Dragovich, 2002) in the margin of the active river zone of the Belá River and they are also destroyed by another even larger flood events (Erskine a Peacock, 2002). However, in the last decades, we are able to observe changes in the magnitude of flood events and the volume of sediment transportation of the Belá River (Kidová, et al., 2016b). The changes caused a simplification of the river planform, leading to a better-developed mature floodplain. The benches in these parts are a well-preserved component of the floodplain. Moreover, local incision of the river channel led to more-level floodplain development with benches set into them. While incision forbids to another floodplain development results of its mutual effect with flood events are ledges, as a level of bank erosion (Lehotský et al., 2015). Following previous research supplemented by new field surveys, we are focused to study of river floodplain evolution of the Belá River. In the end, it will be possible to formulate a process-oriented hypothesis of the recent lateral and vertical development of the floodplain.

\vspace{0.5em}
\noindent
\textbf{Acknowledgements:}
\textit{This research was supported by the Science Grant Agency (VEGA) of the Ministry of Education of the Slovak Republic and the Slovak Academy of Sciences (02/0086/21).}
}%abstract
{Erskine, W. D. and Livingstone, E. (1999) In-channel benches: the role of floods in their formation and destruction on bedrock confined rivers, Varieties of Fluvial Form, (January 1999), s. 445–475.
	
Erskine, W. D. and Peacock, C. T. (2002) Late holocene flood plain development following a cataclysmic flood, The Structure, Function and Management Implications of Fluvial Sedimentary Systems, (276), s. 177–184.
	
Haschenburger, J. K. and Cowie, M. (2009) Floodplain stages in the braided Ngaruroro River, New Zealand”, Geomorphology, 103(3), s. 466–475. \doi{10.1016/j.geomorph.2008.07.016}.
	
Kemp, J. (2004) Flood channel morphology of a quiet river, the Lachlan downstream from Cowra, southeastern Australia, Geomorphology, 60(1–2), s. 171–190. \doi{10.1016/j.geomorph.2003.07.007}.
	
Kidová, A., Lehotský, M. and Rusnák, M. (2016a) Geomorphic diversity in the braided-wandering Belá River, Slovak Carpathians, as a response to flood variability and environmental changes, Geomorphology, 272, s. 137–149.\doi{10.1016/j.geomorph.2016.01.002}.
	
Kidová, A., Lehotský, M. and Rusnák, M. (2016b) Morfologické zmeny a manažment \\divočiaco-migrujúceho vodného toku Belá, GEOMORPHOLOGIA SLOVACA ET BOHEMICA, 16(2), s. 60.
	
Lehotský, M., Kidová, A. and Rusnák, M. (2015) Slovensko-anglické názvoslovie morfológie vodných tokov, GEOMORPHOLOGIA SLOVACA et BOHEMICA, 15(1), s. 61.
	
Vietz, G., Stewardson, M. and Rutherfurd, I. (2005a) Not all benches are created equal: Proposing and field testing an in-channel river bench classification, v Proceedings of the 4th Australian Stream Management Conference, s. 629–635.
	
Vietz, G., Stewardson, M. and Rutherfurd, I. (2005b) Variability in river bench elevation and implications for environmental flow studies, v, s. 8.
	
Webb, A. A., Erskine, W. D. and Dragovich, D. (2002) Flood-driven formation and destruction of a forested flood plain and in-channel benches on a bedrock-confined stream: Wheeney Creek, southeast Australia, IAHS-AISH Publication, (276), s. 203–210.
}%references

%---------------------------------------------------------------------------------------------------
%start of abstract
\abstract
{Geomorphic-sedimentary adjustment of a river reach with groynes to channel bypassing} % Title
{Lehotský et al.} %short author -- toc
{Milan Lehotský\textsuperscript{1}, Miloš Rusnák\textsuperscript{1}, Šárka Horáčková\textsuperscript{1}*, 
Tomáš Štefaničk\textsuperscript{2}, Jaroslav Kleň\textsuperscript{3}} % Author(s)
{\TLtag} % Tag, can be: empty, \KLtag (keynote lecture), \IStag (invited speaker), \CTtag (contributed talk) or \ITtag (invited talk)
{\textsuperscript{1}Department of Physical Geography, Geomorphology and Natural Hazards, Institute of Geography, Slovak Academy of Sciences, Štefánikova 49, 814 73 Bratislava, Slovakia\\
\textsuperscript{2}Department of Theoretical Geodesy, Slovak University of Technology in Bratislava, Radlinského 11, Block A, 810 05 Bratislava, Slovakia\\
\textsuperscript{3}Data Intelligence Engineer, DELL Technologies, Fazuľová 7, 81107 Bratislava, Slovakia 
} % Affiliation(s)
{sarka.horackova@savba.sk}  %e-mail
{Gabcikovo Waterworks; spatio-temporal variability; groyne-induced bench; vertical accretion; conveyance; flood; Danube }%keywords
{The article is focused on the investigation of spatio-temporal variability of the vertical accretion thickness as the response of the Danube River reach to bypassing. Five groyne-induced benches (GIB) of the bypassed channel were developed after water diversion in 1992 and represent our study area. Their topography was created from LiDAR point cloud dataset and DEM models for tree time spans (for original gravel surface, for surface before flood 2013, surface after flood 2013) were calculated and the allostratigraphic approach was applied on 548 drilling probes at five GIB cross-sections.  Head, supra-platform, tail and back channel geomorphic units have been identified at each GIB. The accretion was influenced mostly by large flood events when the 100-yr contributed to the its total volume by almost 26\%. The head geomorphic unit, as the main impact zone on the stream, exhibits the highest median values of vertical accretion in time and space. The median of the vertical accretion thickness does not decrease with height above mean channel water level and the thickness of accretion varied likely due to variability of the vegetation cover conditioning variable hydraulic conditions. The bend setting, lower radius of curvature, lower width-depth ratio, connectivity with side arm and well-developed alternate bench upstream conditioned higher vertical accretion of the benches and its slight migration downstream. Moreover, higher portion of the area was formed as backwater geomorphic unit. The comparison of sediment depth over time spans (22 and 20 years and a large flood event) allowed us to conclude that the it is spatially variable by individual GIB, however its sedimentation trend over time is the same.

\vspace{0.5em}
\noindent
\textbf{Acknowledgements:} \textit{This research was supported by the Scientific Grant Agency of the Ministry of Education, Science and Sport of the Slovak Republic (VEGA 2/0086/21). We appreciate LiDAR data provided by National Forest Centre in Zvolen.}
}%abstract
{}%references
%end of abstract


%start of conference contribution
\abstract
% Title 
{The abandoned underground mine as a semi-natural ecosystem: the story of Flaschar's Mine (Czechia)} 
% EndOfTitle
%short author -- toc 
{Lenart et al.} 
%End short author -- toc
% Author(s) 
{Jan Lenart\textsuperscript{1}*, Kristýna Schuchová\textsuperscript{1}, Martin Kašing\textsuperscript{2}, Lukáš Falteisek\textsuperscript{3}, Šárka Cimalová\textsuperscript{1}} 
% EndOfAuthor(s)
{\TLtag} 
%Tag, can be: empty, \KLtag (keynote lecture), \IStag (invited speaker), \CTtag (contributed talk) or \ITtag (invited talk)
% Affiliation(s)
{
\textsuperscript{1}Faculty of Science, University of Ostrava, Czechia\\
\textsuperscript{2}Faculty of Mining and Geology, VSB – Technical University of Ostrava, Czechia\\
\textsuperscript{3}Faculty of Science, Charles University, Prague, Czechia
}
%} % EndOfAffiliation(s)
{jan.lenart@osu.cz}  %e-mail
%keywords
{geomorphology; geophysical surveys; hydrogeology; mining geology; slate deposits; structural geology}
%EndOfKeywords
%abstract content
{We have investigated a typical abandoned underground slate mine situated in the Bohemian Massif (Central Europe) in order to describe its elementary components and to outline the varied relations among them. Based on the results from geological, geomorphological, hydrological, microclimatic, and biological investigation, we have defined the abandoned underground mine as an important but overlooked semi-natural ecosystem that represents an azonal and relatively fast evolving environment. Unlike any other studies published so far, we have found it also vulnerable in terms of fragility and time-limited stability. Our results together with a comprehensive discussion highlight the fundamental features of the abandoned underground mine and finally serve as a basis on which we introduce a conceptual model of the abandoned underground mine. The complex and interdisciplinary perception of abandoned underground mines is crucial for appropriate environmental assessment, tourist management, natural protection or remediation.
}
%EndOfAbstractContent
%references
{Lenart, J., Schuchová, K., Kašing, M., Falteisek, L., Cimalová, Š., Bílá, J., Ličbinská, M., Kupka, J., 2022: The abandoned underground mine as a semi-natural ecosystem: The story of Flaschar's mine (Czechia). Catena 213: 106178. https://doi.org/10.1016/j.catena.2022.106178.
}
%EndOfReferences
%end of conference contribution
%---------------------------------------------------------------------------------------------------

%start of conference contribution
\abstract
% Title 
{Erosion process and abandonement as main drivers of traditional vineyards degradation: a case study Vráble viticulture district, Slovakia} 
% EndOfTitle
%short author -- toc 
{Lieskovský and Kenderessy} 
%End short author -- toc
% Author(s) 
{Juraj Lieskovský\textsuperscript{1}*, Pavol Kenderessy\textsuperscript{1}} 
% EndOfAuthor(s)
{\TLtag} 
%Tag, can be: empty, \KLtag (keynote lecture), \IStag (invited speaker), \CTtag (contributed talk) or \ITtag (invited talk)
% Affiliation(s)
{
\textsuperscript{1}Institute of Landscape Ecology, Slovak Academy of Sciences, Bratislava, Slovak Republic
}
%} % EndOfAffiliation(s)
{juraj.lieskovsky@savba.sk}  %e-mail
%keywords
{erosion; deposition; erosion measurements; levelling method; vineyards; abandonment}
%EndOfKeywords
%abstract content
{We evaluated the effect of increased erosion events and decreased management on degradation of traditional vineyards near Vráble (Slovakia). For the erosion measurements we used poles height method, that uses vineyards poles as a passive marker. To assess the accuracy of the poles height method, we used the measurements on vineyard located on relatively flat area. The measurements were first performed between 2008 and 2011 (Lieskovský and Kenderessy, 2014) and repeated after 10 years in 2021.

To evaluate the degree of abandonment, we visited all 633 fields and mapped the land cover classes and management practices. 

The erosion rates on ploughed vineyards were doubled in period 2010-2011, but the total soil loss decreased due to the increased deposition. The erosion rates on vineyard that has been hoed and later covered by grass did not changed due to the protective effect of the grass cover. The land cover and management practices has been dramatically changed in last decade. Only 27.2\% of vineyard area near Vráble is now tilled and potentially threatened by soil erosion. On the other hand, 35.7\% of vineyards is abandoned and it seems that abandonment will also continue within next years. The results also show that the most significant factors causing the degradation of traditional vineyards is not soil erosion anymore, but decreasing management that leads to the abandonment.
}
%EndOfAbstractContent
%references
{Lieskovský, J., \& Kenderessy, P. (2014). Modelling the effect of vegetation cover and different tillage practices on soil erosion in vineyards: a case study in Vráble (Slovakia) using WATEM/SEDEM. Land Degradation \& Development, 25(3), 288-296.
}
%EndOfReferences
%end of conference contribution

%---------------------------------------------------------------------------------------------------

\abstract
{Using airborne lidar data for detecting anthropogenic landforms in Poiplie region} % Title
{Lieskovský et al.} %short author -- toc
{Juraj Lieskovský\textsuperscript{1}*, Dana Lieskovská\textsuperscript{1}, Tibor Lieskovský\textsuperscript{2}, Petra Gašparovičová\textsuperscript{1}} % Author(s)
{\TLtag} % Tag, can be: empty, \KLtag (keynote lecture), \IStag (invited speaker), \CTtag (contributed talk) or \ITtag (invited talk)
{\textsuperscript{1}Institute of Landscape Ecology, Slovak Academy of Sciences, Bratislava, Slovakia\\
\textsuperscript{2}Department of Theoretical Geodesy and Geoinformatics, Slovak University of Technology, Bratislava, Slovakia
} % Affiliation(s)
{juraj.lieskovsky@savba.sk}  %e-mail
{LiDAR; landscape palimpsest; Kiarov; Kováčovce; landscape archaeology; anthropogenic landforms}%keywords
{We present the use of Light Detection and Ranging  (LiDAR) data for mapping the anthropogenic landforms in Kiarov and Kováčovce villages, located in Poiplie region. The benefits of LiDAR are the speed of data acquisition on a landscape scale, the quality of its surface data, and its ability to identify macro- and micro-topographical features that are otherwise undetected by terrestrial surveys. LiDAR is now used for detailed topographical research in many disciplines including geomorphology. The area of Slovakia is being scanned from 2017 at a resolution of 20–30 points per square metre and is scheduled to be completed by 2023. 
	
For the interpolation of the LiDAR points to digital elevation model with resolution 25cm, we used points classified as ground and buildings. We applied the visualisation techniques based on a simple local relief model (Kokalj \& Hesse, 2017) but we also added a local dominance to relief curvature representation and also added a combination of slope steepness and sky view factor to relief contrast visualisation (Lieskovský et al., in press). We visually identified the topographic signatures of geomorphic anthropogenic landforms and consulted our findings with historical maps and aerial images. We followed the anthropogenic landform ontology proposed by Tarolli et al. (2019) and we categorized the landforms in the following categories: (1) ‘symbolic’ were represented by cemeteries and chapel; (2) ‘habitation’ were represented by habitation areas and dispersed settlements; (3) ‘transport and exchange’ were represented by paved and unpaved roads, including traces of roads that are not used anymore; (4) ‘subsistence’ were represented by traditional and socialistic agricultural terraces, house gardens, agricultural enterprises, vineyards and orchards and  hedgerows; (5) ‘mining’ were represented by surface mining pits (6) ‘water infrastructure’ consisted of melioration canals and artificial river bed of Ipel river; (7) ‘waste disposal’ was found in central part of Kováčovce cadastral area; and (8) ‘warfare infrastructure’ was represented by trenches and ditches from the Second World War.
}%abstract
{Kokalj Ž, Hesse R (2017) Airborne Laser Scanning Raster Data Visualization. Vol. 14. Prostor, Kraj, Čas. ZRC SAZU, Založba ZRC, Ljubljana
	
Lieskovský J, Lieskovský T, Hladíková K, Štefunková D, Hurajtová N (in press) Potential of Airborne LiDAR Data in Detecting Cultural Landscape Features in Slovakia. Landscape Research.
	
Tarolli P, Wenfang C, Giulia S, Damian E, and Erle C-E(2019) From Features to Fingerprints: A General Diagnostic Framework for Anthropogenic Geomorphology. Progress in Physical Geography: Earth and Environment 43, no. 1: 95–128.}%references
%---------------------------------------------------------------------------------------------------
%---------------------------------------------------------------------------------------------------

%---------------------------------------------------------------------------------------------------

%start of conference contribution
\abstract
% Title 
{Macroplastic debris in a mountain river: Does geomorphology matters?} 
% EndOfTitle
%short author -- toc 
{Liro et al.} 
%End short author -- toc
% Author(s) 
{Maciej Liro\textsuperscript{1}, Mateusz Kieniewicz\textsuperscript{2}*, Elżbieta Gorczyca\textsuperscript{2}} 
% EndOfAuthor(s)
{\POtag} 
%Tag, can be: empty, \KLtag (keynote lecture), \IStag (invited speaker), \CTtag (contributed talk) or \ITtag (invited talk)
% Affiliation(s)
{
\textsuperscript{1}Institute of Nature Conservation, Polish Academy of Sciences, Kraków, Poland\\
\textsuperscript{2}Institute of Geography and Spatial Management, Jagiellonian University in Kraków, Poland
}
%} % EndOfAffiliation(s)
{mateusz.kieniewicz@student.uj.edu.pl}  %e-mail
%keywords
{mountain river; macroplastic debris; river morphology; macroplastic storage }
%EndOfKeywords
%abstract content
{Identification of plastic accumulation zones in rivers is crucial to plan future mitigation strategies and to assess related risks (van Emmerik et al., 2022). Recent observations from the Polish Carpathians suggest that river morphology play an important role in the storage process of macroplastic debris (>5 mm) transported by river flow (Liro et al., 2022). This poster presents the concepts of master thesis aimed to explore the effects of different river morphologies on the amount of macroplastic stored in a mountain river. The field works will be conducted within twenty cross sections located along the Biała River in Polish Carpathians. These cross sections represent simplified morphology of channelized river (10 cross sections) and diverse morphology of unmanaged river (10 cross sections). We hypothesize that larger amounts of in-channel-stored macroplastic will occur in wider, unmanaged channel reaches typified by larger number of obstacles to river flow (e.g., bars, wood jams, channel island) than in the channelized reaches typified by simplified flow and morphological patterns. The morphology of riverbed in a multithread reaches additionally favours development of riparian vegetation, that can effectively traps macroplastic, especially during floods (Cesarini et al., 2022). In each of the cross-section, we will use hand collection to quantify the abundance of surface-stored macroplastic debris and will express it as items number and mass (g) per m2. Collected data allow us to demonstrate potential differences in macroplastic storage between contrasting styles of channel management and to indicate the elements of a mountain river retaining the highest amount of macroplastic.
}
%EndOfAbstractContent
%references
{Liro M, Mikuś P, Wyżga B, 2022. First Insight into the Macroplastic Storage in a Mountain River: The Role of In-River Vegetation Cover, Wood Jams and Channel Morphology. \doi{10.2139/ssrn.4060132}

van Emmerik, T, Mellink Y., Hauk R, Waldschläger K, Schreyers, L, 2022. Rivers as plastic reservoirs. Front. Water 3, 786936. \doi{10.3389/frwa.2021.786936}

Cesarini J, Scalici M, 2022. Riparian vegetation as a trap for plastic litter. Environmental Pollution, Volume 292, Part B. \doi{10.1016/j.envpol.2021.118410}
}
%EndOfReferences
%end of conference contribution
%---------------------------------------------------------------------------------------------------

%start of conference contribution
\abstract
% Title 
{Land surface temperature controls stability in gentle clay slopes} 
% EndOfTitle
%short author -- toc 
{Loche et al.} 
%End short author -- toc
% Author(s) 
{Marco Loche\textsuperscript{1}*, Gianvito Scaringi\textsuperscript{1}, Luigi Lombardo\textsuperscript{2}} 
% EndOfAuthor(s)
{\TLtag} 
%Tag, can be: empty, \KLtag (keynote lecture), \IStag (invited speaker), \CTtag (contributed talk) or \ITtag (invited talk)
% Affiliation(s)
{
\textsuperscript{1}Faculty of Science, Charles University, Prague, Czech Republic\\
\textsuperscript{2}Faculty of Geo-Information Science and Earth Observation (ITC), University of Twente, Enschede, Netherlands
}
%} % EndOfAffiliation(s)
{marco.loche@natur.cuni.cz}  %e-mail
%keywords
{slow flows; land surface temperature; landslide susceptibility; slope unit; clay}
%EndOfKeywords
%abstract content
{The effect of temperature on the stability of slopes in temperate climates is poorly constrained. Experiments demonstrate a clear thermo-hydro-mechanical (THM) response in expansive soils, and evidence of thermally-induced activity exists for some landslides (Scaringi and Loche, 2022). However, building a representative thermal variable suitable for catchment or regional-scale studies is challenging, owing to heterogeneities in materials and stress histories and the complexity of THM processes (Loche et al., 2022. We performed a landslide susceptibility modelling of the portion of Italian territory featuring clay deposits. We utilised the geo-lithological map of Italy and the Italian National Inventory (IFFI), that differentiates among landslide types, and we focused on slow flows, often associated with creep movements. We relied, as one of the inputs, on a ten-year series of Land Surface Temperature (LST) data from MODIS, freely available in Google Earth Engine, and implemented a slope unit-based Generalized Additive Model (GAM) approach to account for nonlinearities in the possible temperature-slope stability relationship (Alvioli et al., 2016). We produced a susceptibility map for clay deposits over the entire Italian territory and observed a positive dependence of landslide abundance on LST on warmer and gentle slopes, where creep phenomena are common. Higher temperatures are in fact associated with decreased soil and water viscosity and hence enhanced shear creep rates in clays.
}
%EndOfAbstractContent
%references
{Scaringi, G., \& Loche, M. (2022). A thermo-hydro-mechanical approach to soil slope stability under climate change. Geomorphology, 108108.

Loche, M., Scaringi, G., Yunus, A. P., Catani, F., Tanyaş, H., Frodella, W., ... \& Lombardo, L. (2022). Surface temperature controls the pattern of post-earthquake landslide activity. Scientific reports, 12(1), 1-11.

Alvioli et al. (2016) Automatic delineation of geomorphological slope units with r. slopeunits v1.0 and their optimization for landslide susceptibility modeling. Geoscientific Model Development, 9(11), pp.3975-3991.

}
%EndOfReferences
%end of conference contribution
%---------------------------------------------------------------------------------------------------

%start of conference contribution
\abstract
% Title 
{Preliminary flood hazard assessment based on detailed LiDAR data} 
% EndOfTitle
%short author -- toc 
{Michaleje L.} 
%End short author -- toc
% Author(s) 
{Lukáš Michaleje\textsuperscript{1}*} 
% EndOfAuthor(s)
{\TLtag} 
%Tag, can be: empty, \KLtag (keynote lecture), \IStag (invited speaker), \CTtag (contributed talk) or \ITtag (invited talk)
% Affiliation(s)
{
\textsuperscript{1}Institute of Geography Slovak Academy of Sciences, Bratislava, Slovakia
}
%} % EndOfAffiliation(s)
{geoglumi@savba.sk}  %e-mail
%keywords
{flood hazard; LiDAR; preliminary assessment}
%EndOfKeywords
%abstract content
{
Floods are one of the most common natural hazards in Slovakia, with the most serious consequences for the population and the economy. For the needs of management and the application of appropriate measures to reduce the flood hazard, it is necessary to make decisions based on relevant information from the flood hazard assessment (Albano et al. 2017). With the detailed data, the flood hazard assessment itself can becomes time-consuming and computationally demanding (Teng et al. 2017). Thanks to the preliminary assessment, we can exclude areas with a very low flood hazard and at the same time identify areas that deserve more detailed attention.
In our paper, we focus on preliminary flood hazard assessment. Our approach uses detailed LiDAR data with an average density of 32 points per m2 and an average height accuracy of 0.05 m from the Institute of Geodesy, Cartography and Cadastre Authority of the Slovak Republic in Bratislava. From these data, we created a DEM (digital elevation model) with a cell size of 1 m and derived layers that enter the preliminary flood hazard assessment. We also used a soil data from Soil Science and Conservation Research Institute and National Forest Centre. Land cover was created from the ZBGIS spatial database. Subsequently we recalculated the output for the river basin and for the affected municipalities according to the urban area.

\vspace{0.5em}
\noindent
\textbf{Acknowledgements:}
\textit{This research was supported by the Science Grant Agency (VEGA) of the Ministry of Education, science, research and sport of the Slovak Republic and the Slovak Academy of Sciences No. 02/0086/21 Assessment of the impact of extreme hydrological phenomena on the landscape in the context of a changing climate.}
}
%EndOfAbstractContent
%references
{Albano R, Mancusi L, Abbate A (2017) Improving flood risk analysis for effectively supporting the implementation of flood risk management plans: The case study of “Serio” Valley. Environmental Science and Policy 75 (1) 158-172 \doi{10.1016/j.envsci.2017.05.017}

Teng J. Jakeman J. A. Vaze J. Croke W.F.B. Dutta D. Kim S. (2017) Flood inundation modelling: A review of methods, recent advances and uncertainty analysis. Environmental Modelling \& Software (90) 201-2016 	\doi{10.1016/j.envsoft.2017.01.006}
}
%EndOfReferences
%end of conference contribution

%---------------------------------------------------------------------------------------------------

%start of abstract
\abstract
{Long-term monitoring of the recruitment and dynamics of large wood in Kamienica Stream, Polish Carpathians} % Title
{Mikuś and Wyżga} %short author -- toc
{Paweł Mikuś\textsuperscript{1}*, Bartłomiej Wyżga\textsuperscript{1}} % Author(s)
{\TLtag} % Tag, can be: empty, \KLtag (keynote lecture), \IStag (invited speaker), \CTtag (contributed talk) or \ITtag (invited talk)
{\textsuperscript{1}Institute of Nature Conservation, Polish Academy of Sciences, Kraków, Poland
} % Affiliation(s)
{mikus@iop.krakow.pl}  %e-mail
{large wood, wood dynamics, wood monitoring, wood inventory, Polish Carpathians}%keywords
{Quantifying wood delivery and mobility in small mountain streams requires long-term and repeatable observations, so far very scarcely described. Recently, observations using a number of remote sensing methods have gained popularity. However, classical methods of fieldwork still seem to be indispensable in more detailed studies. Such observations were conducted on the length of 8.7 km of the upper course of Kamienica Stream, Polish Carpathians, where recent bark beetle infestation of riparian spruce forest might have considerably increased the delivery of fallen trees to the channel. This part of the stream course is located in the Gorce Mountains National Park and large wood is not removed from the stream under the national park regulations.

In October 2009, numbered metal plates were installed on 429 trees growing along three stream sections located 2500–2950 m (section A), 4000–4450 m (section B) and 7850–8300 m (section C) from the stream source. Different metals were used in each section to allow for finding tagged trees with a metal detector in case the plates on them are inaccessible. The monitoring of standing and fallen trees tagged with metal plates has been conducted a few times per year, especially after heavy rainfall and windstorms. Moreover, the mode of location and the degree of decay of wood pieces stored in the study reach were determined in 2012.

During twelve years of observations, 111 trees (26\% of the tagged sample) were supplied to the stream as a result of bank erosion, windthrow of living trees or those killed by bark beetle infestation, snow overload and landslides. In October 2009, 80 cm of wet snow fell in two days and snow overload caused breaking of two tagged trees. Five events with high water stage occurred in May 2010, May 2014, July 2016, May 2018 and May 2019. Those from 2016 and 2018 can be considered as major floods as they caused significant channel changes and damage to local infrastructure. About half of trees supplied to the channel were not transported, and numerous wood dams occurring in the stream limited transport of any fallen trees. Forty-six fallen tagged trees (48\%) were transported during some of the five floods. In sections A and B, mean distance of the displacement of tagged trees over the study period was small and did not exceed 32 m, whereas in section C it was a few times longer. As a result of the flood in 2010, three trees were displaced relatively short distances (Mean = 42 m, Max = 100 m) and retained in in-channel jams. A definitely larger flood from 2014 was marked only in the lowermost section C, where the wood occurring in the channel was crushed into small pieces and flushed out downstream. The flood of July 2016 was the only major flood in all study sections; during this event 11 trees were displaced a mean distance of 97 m (Max = 230 m). In May 2018, a major flood caused by heavy rainfall resulted in considerable bank erosion in the lowest study section. During this event, 41 trees from this section were recruited to the channel and transported (Mean = 275 m, Max = 1003 m), while no transport occurred in the other sections. As trees in the upstream sections remained practically untouched, the course of this flood showed a strongly localized occurrence of the triggering rainfall. A small flood in May 2019 did not displace any tagged trees occurring in the stream. 

A large wood inventory performed in 2012 indicated that in the second-order reach wood was relatively uniformly distributed among different location types (Wyżga et al., 2015), with logs with their top or bottom located in the channel being the most abundant (21\% of all wood pieces). Moreover, near-perpendicular orientation of wood pieces in relation to the channel axis prevailed in the reach (Mikuś et al., 2016). All this indicates a negligible role of transport and redistribution of wood, which was predominantly retained where it fell. This reach was typified by the largest proportion of logs forming wood dams (12\%) and spanning the channel (10\%) among the study reaches, which can be attributed to the relatively large length of wood pieces in relation to channel width. The third-order reach was characterized by a similar pattern of wood location as the second-order reach. It was distinguished by only one feature: a small amount of wood spanning the channel. In the fourth-order reach, large wood spanning the channel was very scarce because of larger stream width and considerably larger flood discharges. Here, large wood was predominantly retained along channel margins (41\%), on gravel bars (19\%), and with only the top or bottom located in the channel (17\%). This reach was typified by variable orientation of wood pieces in relation to the channel axis, with a proportion of the pieces being apparently reoriented by the stream current. Most of wood pieces were shorter than channel width, as they originated from the breakage of trees into smaller fragments during their fall to the channel or transport by flood flows. 

The inventory also indicated that in the second-order reach of Kamienica, 16\% of wood pieces were in a relatively good condition, representing class 1 and 2 of wood decay, whereas a more advanced degree of decomposition, typical of class 3 and 4, typified 84\% of pieces. In the third-order stream reach, this distribution was more even, with 41\% of pieces representing class 1 and 2 of wood decay, and 59\% class 3 and 4. In the fourth-order reach, classes 1 and 2 constituted 31\% of all wood pieces, and 69\% had typical features of classes 3 and 4.

To conclude, large wood is recruited to the upper course of Kamienica Stream by a few processes, with bank erosion and windthrow having been most effective during 12 years of monitoring. Large wood was recruited to the stream only during high-intensity meteorological and hydrological events. With 22\% of tagged trees recruited to the channel during 12 years, the rate of turnover of the riparian trees was estimated at 45 years. As the riparian area supports trees with ages up to ~160 years, the rate evidences substantial intensification of large wood recruitment to the channel in the recent period resulting from bark beetle infestation of riparian trees. The mobility of wood in the stream increases downstream because of increasing flood discharges and the decreasing ability of fallen trees to anchor on the banks of increasingly wide channel. Wood is transported longer distances only during major floods, which tend to deposit wood pieces along channel margins and on gravel bars, where wood is subjected to relatively rapid decomposition under subaerial conditions. During a subsequent large flood, most wood pieces already occurring in the channel are thus likely to rapidly disintegrate, rather than being flushed out downstream. Thus, large wood retained in the upper stream course does not constitute an important flood hazard to downstream, inhabited valley reaches.
}%abstract

{Wyżga B, Zawiejska J, Mikuś P, Kaczka RJ (2015) Contrasting patterns of wood storage in mountain watercourses narrower and wider than the height of riparian trees, Geomorphology, 228, 275-285.  

Mikuś P, Wyżga B, Ruiz-Villanueva V, Zawiejska J, Kaczka RJ, Stoffel M (2016) Methods to assess large wood dynamics and the associated flood hazard in Polish Carpathian watercourses of different size. In: Kundzewicz ZW, et al. (eds.), Flood Risk in the Upper Vistula Basin. Springer, Cham, pp. 77-101.
}%references
%end of abstract
%---------------------------------------------------------------------------------------------------

%start of conference contribution
\abstract
% Title 
{Sediment discharge estimation of lowland rivers using Sentinel-2 images and machine learning algorithms} 
% EndOfTitle
%short author -- toc 
{Mohsen et al.} 
%End short author -- toc
% Author(s) 
{Ahmed Mohsen\textsuperscript{1,2}*, Ferenc Kovács\textsuperscript{1}, Tímea Kiss\textsuperscript{1}} 
% EndOfAuthor(s)
{\TLtag} 
%Tag, can be: empty, \KLtag (keynote lecture), \IStag (invited speaker), \CTtag (contributed talk) or \ITtag (invited talk)
% Affiliation(s)
{
\textsuperscript{1}Faculty of Science, University of Szeged, Szeged, Hungary 
\textsuperscript{2}Faculty of Engineering, Tanta University, Tanta, Egypt
}
%} % EndOfAffiliation(s)
{ahmed\textunderscore mohsen250@f-eng.tanta.edu.eg}  %e-mail
%keywords
{suspended sediment discharge; rating curves; hydraulic geometry theory; genetic algorithm; artificial neural network}
%EndOfKeywords
%abstract content
{The sediment transport of rivers is a complex process, as it is affected by a variety of natural and anthropogenic parameters, and it can considerably change spatially and temporally too. In addition, contaminant transport (e.g., microplastic and mining waste) is linked to the transport of natural sediments; therefore, it is important to understand and monitor these transport processes. Most rivers are simply monitored at limited number of stations or are not gauged at all. The remote sensing-based alternatives for sediment discharge (and pollutant) monitoring provide a solution to this problem.

This study aims to calibrate and validate water discharge and suspended sediment models for two Central European rivers using in-situ and Sentinel-2 data. The derived models could be also applied to estimate suspended sediment discharge at ungauged periods and sections.

The hydraulic geometry of river channel theory was employed to estimate water discharge at three gauging stations in the Tisza River (at Szeged and Algyő) and its tributary the Maros River (at Makó). The AHG (At-a-station Hydraulic Geometry) power-law and AMHG (At-Many-stations Hydraulic Geometry) methods were utilized to estimate water discharge, besides, we developed a novel method based on the AHG theory and machine learning algorithms. The surface reflectance of Sentinel-2 images was correlated to the monthly (2015--2020) measured suspended sediment concentration (SSC) by Support Vector Machine (SVM), Random Forest (RF), Artificial Neural Network (ANN), and combined algorithms. The best-performed water discharge and suspended sediment concentration models were applied to 122 Sentinel-2 images to produce a relatively long sediment discharge time series at the selected gauging stations. 

The main results revealed the usefulness of the hydraulic geometry of the river channel in terms of the discharge-width relationship for estimating water discharge from space; however, the morphological characteristics of the selected reach (i.e., size and shape of the cross-sectional profile) significantly affect the accuracy of the discharge estimation. The AMHG method outperformed the AHG power-law method, as previously reported in the literature; however, due to the capabilities of machine learning algorithms for producing complicated regression models, our novel AHG machine learning method performed equally well or slightly better than the AMHG method. Among the four tested machine learning algorithms (i.e., SVM, RF, ANN and combined algorithms), the ANN gave the highest water discharge accuracy for the gauging station upstream of the Tisza‒Maros confluence at Algyő (R\textsuperscript{2}=0.79 and RMSE=123.3 m\textsuperscript{3}/s), however the combined method was the best for the Tisza downstream of the confluence at Szeged (R\textsuperscript{2}=0.82 and RMSE=99.9 m\textsuperscript{3}/s) and for the Maros at Makó (R\textsuperscript{2}=0.88 and RMSE=25.9 m\textsuperscript{3}/s) gauging station. 

The machine learning algorithms gave also good estimations for the suspended sediment concentration, as based on the best performing models, the suspended sediment concentration was estimated with R2=0.82 and RMSE=15.43 mg/l (combined model) in the Tisza River and R2=0.90 and RMSE=19.97 mg/l (RF) in the Maros River. 

The greatest sediment discharge is concurrent with floods; however, usually, there is a clockwise (positive) hysteresis between the suspended sediment concentration and water discharge, particularly in the Tisza River.

The combination of remote sensing images and machine learning techniques could be used to efficiently monitor the discharge and suspended sediment concentration of rivers by producing robust regression models that could be applied to large-scale and frequent images. However, the limitations of this method should also be considered, as it is affected by a) the channel width and spatial resolution of the employed satellite images; b) channel cross-sectional shape, as some shape have better hydraulic geometry than others; c) bankfull level, as models typically underestimate or overestimate the actual discharge above this level; d) shadow, sun glint and waves of water pixels, which affects the suspended sediment concentration estimations. The application of this method to many rivers around the world could enhance our understanding of not only the spatiotemporal dynamism of the suspended sediment transport, but also the transport of suspended contaminants.
}
%EndOfAbstractContent
%references
{
}
%EndOfReferences
%end of conference contribution
%---------------------------------------------------------------------------------------------------

%start of conference contribution
\abstract
% Title 
{Geodiversity assessment of the Western Carpathians} 
% EndOfTitle
%short author -- toc 
{Novotný et al.} 
%End short author -- toc
% Author(s) 
{Ján Novotný\textsuperscript{1}*, Anna Chrobak-Žuffová\textsuperscript{2}, Paweł Struś\textsuperscript{2}} 
% EndOfAuthor(s)
{\TLtag} 
%Tag, can be: empty, \KLtag (keynote lecture), \IStag (invited speaker), \CTtag (contributed talk) or \ITtag (invited talk)
% Affiliation(s)
{
\textsuperscript{1}Institute of Geography, Slovak Academy of Sciences, Bratislava, Slovakia\\
\textsuperscript{2}Institute of Geography, Pedagogical University of Krakow, Cracow, Poland
}
%} % EndOfAffiliation(s)
{jan.novotny@savba.sk}  %e-mail
%keywords
{geodiversity; geotourism; mapping; assessment methods; map algebra; Western Carpathians}
%EndOfKeywords
%abstract content
{
The concept of geodiversity appeared in the 1990s and is defined as the natural diversity of features of geological structure, relief, and soil cover, including the relationships between these features, their properties, and their impact on other elements of the natural and cultural environment (e. g. Gray, 2004, 2013; Zwoliński, 2004). It is described and analysed using various types of quantitative, qualitative, or quantitative–qualitative methods. The concept of a geodiversity map presented in this contribution belongs to the third of these groups of methods. The basis of methodology for creating a synthetic map of the geodiversity of the Western Carpathians was taken from the work by Zwoliński (2009). Despite the use of optimization methods in the form of a hexagon grid or the analytic hierarchy process calculator, it still remains partially subjective. From this point of view, the classification of individual litotypes displayed in geological maps seems to be the most problematic. The use of this method to calculate the geodiversity of an entire province (the Western Carpathians) gives a general view of the natural diversity of this area and allows regions to be selected for more detailed analyses or comparisons to be made between them. The geodiversity map is also a very good background on which to illustrate geotourist potential, which is expressed in terms of the number and distribution of geosites. However, in the case of the Western Carpathians, these two variables do not correlate with each other. The applied geodiversity map, although still subjective, is sufficiently optimal that it can also be used in other mountain areas by adjusting the class intervals accordingly. The creation of new geotourist attractions or better sharing of already existing geosites should be based on the principles of sustainable development in geoethics.
}
%EndOfAbstractContent
%references
{Chrobak A, Novotný J, Struś P (2021) Geodiversity Assessment as a First Step in Designating Areas of Geotourism Potential. Case Study: Western Carpathians. Frontiers in Earth Science 9: Article 752669.
	
Gray, M (2004) Geodiversity Valuing and Conserving Abiotic Nature. Wiley.
	
Gray, M (2013) Geodiversity: Valuing and Conserving Abiotic Nature. 2nd Edition. Wiley-Blackwell, Chichester.
	
Zwoliński, Z. (2004). Geodiversity. In Goudie A S, ed. Encyclopedia of Geomorphology. Routledge, New York: 417–418.
	
Zwoliński, Z (2009) The routine of landform geodiversity map design for the Polish Carpathian Mts. Landform Analysis 11: 77–85.
}
%EndOfReferences
%end of conference contribution
%---------------------------------------------------------------------------------------------------

%start of conference contribution
\abstract
% Title 
{Practical experience of a fluvial geomorphologist in the context of landscape changes} 
% EndOfTitle
%short author -- toc 
{Ondráčková L.} 
%End short author -- toc
% Author(s) 
{Lenka Ondráčková\textsuperscript{1}*} 
% EndOfAuthor(s)
{\TLtag} 
%Tag, can be: empty, \KLtag (keynote lecture), \IStag (invited speaker), \CTtag (contributed talk) or \ITtag (invited talk)
% Affiliation(s)
{
\textsuperscript{1}Department of Water Management Studies, ENVIPARTNER, s.r.o., Vídeňská 55,
	639 00, Brno – Štýřice, Czech Republic
}
%} % EndOfAffiliation(s)
{ondrackova@envipartner.cz}  %e-mail
%keywords
{fluvial geomorphology; erosion; flash floods; field mapping; Gisella app; water management}
%EndOfKeywords
%abstract content
{The contribution focuses on the practical experience with the use of fluvial geomorphological knowledge. Few examples from the water management, soil erosion, flash floods consequences and flood inspection will be presented. All projects are the result of cooperation with municipalities in flood protection within the Czech Republic. The outputs from the Gisella app will be shown. Gisella app allows you to easily target everything in the field and export the data to a desktop GIS or, for example, to a WEGAS web application. With Gisella app you can create points, lines and polygons, record all the available features and even attach photos and videos from the field. You will get accurate data directly from the affected area. The first project is the Rajnochovice flood inspection. There will be many problems within the river corridor shown and described. We can easily divide the problems in the riverbed and in the active zone of the floodplain. After the fieldwork all the problems are prioritized and discussed with stakeholders. Another project deals with the flood protection measures on the Kozlanka River, where the problems with the massive contribution of sediments are expected after rain period. Last presented projects are about the effect of flash floods in the built-up areas. The effective measures on agricultural land drainage are necessary throughout the whole country. 
}
%EndOfAbstractContent
%references
{Ondráčková L (2019) Preventivní povodňová prohlídka Rajnochovice. ENVIPARTNER, s. r. o. Brno

Bureš F, Mach A, Vrba J, Ondráčková L (2021) Studie protizáplavových opatření na toku Kozlanka v obci Krásná ř. km 0,000--0,900. ENVIPARTNER, s. r. o. Brno

Ondráčková L, Bureš F, Mach A (2021) Koncepce realizace protipovodňových opatření obce Němčice. ENVIPARTNER, s. r. o. Brno

Ondráčková L, Bureš F, Štulc V (2021) Strategie boje se suchem – přívalové srážky v obci Domanín. ENVIPARTNER, s. r. o. Brno 
}
%EndOfReferences
%end of conference contribution
%---------------------------------------------------------------------------------------------------

\abstract
{Large landslides along the eastern margin of Patagonian Ice Sheet: distribution, causes and timing} % Title
{Pánek et al.} %short author -- toc
{Tomáš Pánek\textsuperscript{1}*, Michal Břežný\textsuperscript{1}, Elisabeth Schönfeldt\textsuperscript{2}, Veronika Kapustová\textsuperscript{1}, Diego Winocur\textsuperscript{3}, \\Rachel Smedley\textsuperscript{4}} % Author(s)
{\TLtag} % Tag, can be: empty, \KLtag (keynote lecture), \IStag (invited speaker), \CTtag (contributed talk) or \ITtag (invited talk)
{\textsuperscript{1}University of Ostrava, Department of Physical Geography and Geoecology, Chittussiho 10, Slezská Ostrava, Czech Republic \\
\textsuperscript{2}University of Potsdam, Institute of Geosciences, Karl-Liebknecht-Straße 24-25, 14476, Potsdam, Germany \\
\textsuperscript{3}Universidad de Buenos Aires, Facultad de Ciencias Exactas y Naturales, Departamento de Ciencias Geologicas, Intendente Güiraldes 2416, C1428EGA, CABA, Argentina \\
\textsuperscript{4}University of Liverpool, Department of Geography and Planning, Chatham Street, Liverpool, L69 7ZT, UK} % Affiliation(s)
{tomas.panek@osu.cz}  %e-mail
{}%keywords
{Although ice retreat is widely considered to be an important factor in landslide origin, many links between deglaciation and slope instabilities are yet to be discovered. Here we focus on the origin and chronology of exceptionally large landslides situated along the eastern margin of the former Patagonian Ice Sheet (PIS). Accumulations of the largest rock avalanches in the former PIS territory are concentrated in the Lago Pueyrredón valley at the eastern foothills of the Patagonian Andes in Argentina. Long-runout landslides have formed along the rims of sedimentary and volcanic mesetas, but also on the slopes of moraines from the Last Glacial Maximum. At least two rock avalanches have volumes greater than 1 km\textsuperscript{3} and many other landslide accumulations have volumes in the order of tens to hundreds of million m\textsuperscript{3}. Using cross-cutting relationships with glacial and lacustrine sediments and using OSL and \textsuperscript{14}C dating, we found that the largest volume of landslides occurred between \textasciitilde17 and \textasciitilde11 ka. This period coincides with the most rapid phase of PIS retreat, the greatest intensity of glacial isostatic uplift, and the existence of a dropping glacial lake along the foothills of the Patagonian Andes. The position of paleoshorelines in the landslide bodies and, in many places, folded and thrusted lacustrine sediments at the contact with rock avalanche deposits indicate that the landslides collapsed directly into the glacial lake. Although the landslides along the former glacial lobe of Lago Pueyrredón continue today, they are at least an order of magnitude smaller than the rock and debris avalanches that occurred before the drainage of the glacial lake around 10-11 ka. Preliminary numerical modeling results indicate that large postglacial landslides may have been triggered by a combination of rapid sequential glacial lake drawdowns and seismicity due to glacial isostatic adjustment. We conclude that in addition to direct links such as glacial oversteepening, debuttressing and permafrost degradation, the retreat of ice sheets and the subsequent formation of transient large glacial lakes can fundamentally alter stability conditions, especially if the slopes are built by weak sedimentary and volcanic rocks.}%abstract
{}%references
%---------------------------------------------------------------------------------------------------

%start of conference contribution
\abstract
% Title 
{Morphometry of Železné Hory Mts. Fault} 
% EndOfTitle
%short author -- toc 
{Patrnčiak and Štěpančíková} 
%End short author -- toc
% Author(s) 
{Lukáš Patrnčiak\textsuperscript{1}*, Petra Štěpančíková\textsuperscript{2}} 
% EndOfAuthor(s)
{\POtag} 
%Tag, can be: empty, \KLtag (keynote lecture), \IStag (invited speaker), \CTtag (contributed talk) or \ITtag (invited talk)
% Affiliation(s)
{
	\textsuperscript{1}Faculty of Science, Masaryk University, Brno, Czech Republic
	\textsuperscript{2}Institute of Rock Structure and Mechanics, Czech Academy of Sciences, Prague, Czech Republic
}
%} % EndOfAffiliation(s)
{lukas.patrnciak95@gmail.com}  %e-mail
%keywords
{Železné Hory; morphometry; indices; tectonics; fault}
%EndOfKeywords
%abstract content
{
The Železné Hory Mts. Fault presents a major rupture in the central Czech Republic in total length of about 70 kilometres. This thrust fault divides crystalline rocks of Železné Hory (Iron Mountains), that were risen along it, from the cretaceous sediments of Bohemian Cretaceous Basin. It is considered inactive and only a few study works were conducted on it in the past. We are presenting a geomorphometrical analysis of the fault zone along its length and showing potential areas of increased or recent tectonic activity, serving as a basis for further research. Although the fault being considered inactive in quaternary, certain anomalies was found using approach by computing geomorphologic indices, and that mainly on middle segment of the fault. Situation on northern part is complicated by abstention of river network, what makes further interpretation from certain analysis harder. Combination of different indicators and significantly different values in shared areas point us to valuable localities, such as deep valleys of Zlatý and Lovětínský stream cutting through a fault ridge. Nevertheless, also influence of lithology had to be considered, as fault zone consists of various types of bedrock, what complicates the results.
}
%EndOfAbstractContent
%references
{
Bíl, M. 2002. Využití geomorfometrických technik při studiu neotektoniky (na příkladu Vsetínských vrchů). MS disert. práce, Kated. geogr., PřF MU, Brno, 100 s. 

Bíl, M., Máčka, Z. (1999): Využití spádových indexů řek jako indikátorů tektonických pohybů na zlomech. Geologické výzkumy na Moravě a ve Slezsku, roč. 6, Brno, s. 2-5.

Bull, W. B., McFadden, L. D. (1980): Tectonic geomorphology north and south of the Garlock Fault, California. In: Doehring, D. O. (ed.): Geomorphology in Arid Regions. Allen \& Unwin, London, p. 115-138.

Keller, E. A., Pinter, N. (1996): Active Tectonics : Earthquakes, Uplift and Landscape. Prentice Hall : Upper Saddle River. 2nd edition. 362 p.

Strahler, A. N. (1952): Hypsometric (area-altitude) analysis of erosional topography. Bulletin of the Geological Society of America, 63, p. 1117-42.
}
%EndOfReferences
%end of conference contribution
%---------------------------------------------------------------------------------------------------

%start of conference contribution
\abstract
% Title 
{The impact of river engineering works and flood events on channel migration rates on the Orljava River (the Pannonian Basin, Croatia)} 
% EndOfTitle
%short author -- toc 
{Pavlek and Čanjevac} 
%End short author -- toc
% Author(s) 
{Katarina Pavlek\textsuperscript{1}*, Ivan Čanjevac\textsuperscript{1}} 
% EndOfAuthor(s)
{\TLtag} 
%Tag, can be: empty, \KLtag (keynote lecture), \IStag (invited speaker), \CTtag (contributed talk) or \ITtag (invited talk)
% Affiliation(s)
{
\textsuperscript{1}Department of Geography, Faculty of Science, University of Zagreb, Zagreb, Croatia
}
%} % EndOfAffiliation(s)
{kpavlek@geog.pmf.hr}  %e-mail
%keywords
{channel migration; flood events; engineering works; aerial images; Pannonian basin}
%EndOfKeywords
%abstract content
{Since the 19th century, hydromorphology of many lowland rivers in the Central Europe has been modified due to human interventions. River management measures often involve hard engineering strategies such as channelization and dredging in order to ensure flood protection and drainage. Unlike the majority of lowland rivers in Croatia, the Orljava River has not been extensively channelized. Still, its morphodynamics has been under significant human impact due to removal of riparian vegetation, artificial cut-offs, and construction of weirs. This study investigates changes in lateral channel migration rates on the Orljava River since the mid-20th century based on the analysis of topographic maps and aerial images in a GIS environment on selected river reaches. In addition, changes in the channel bank line have been monitored on distinct meanders using an unmanned aerial vehicle (UAV) and a GNSS receiver since 2021. According to preliminary results, the acceleration of lateral channel migration after 2011 seems to be strongly connected with recent engineering works, particularly removal of vegetation from river banks, since the occurrence of higher discharges in the previous decades was not followed by such high channel migration rates. Nevertheless, the largest reach-averaged migration rate of 3.5 m per year was apparently induced by a major flood event in 2014. In general, the largest rates of channel migration in the studied period were recorded downstream of artificial cut-offs and upstream and downstream of a breached mill weir. Since recently increased erosion of river banks leads to the loss of adjacent agricultural land, a change in the approach to river management is needed.
}
%EndOfAbstractContent
%references
{
}
%EndOfReferences
%end of conference contribution
%---------------------------------------------------------------------------------------------------

%start of conference contribution
\abstract
% Title 
{Large wood in contrast fluvial landscapes: A tale of different ecosystem services} 
% EndOfTitle
%short author -- toc 
{Połedniková and Galia} 
%End short author -- toc
% Author(s) 
{Zuzana Połedniková \textsuperscript{1}*, Tomáš Galia\textsuperscript{1}} 
% EndOfAuthor(s)
{\TLtag} 
%Tag, can be: empty, \KLtag (keynote lecture), \IStag (invited speaker), \CTtag (contributed talk) or \ITtag (invited talk)
% Affiliation(s)
{
\textsuperscript{1}Faculty of Science, University of Ostrava, Ostrava, Czech Republic
}
%} % EndOfAffiliation(s)
{zuzana.polednikova@osu.cz}  %e-mail
%keywords
{ecosystem services; fluvial landscape; river; vectorization}
%EndOfKeywords
%abstract content
{
Fluvial landscapes are one with the highest capacities of ecosystem services. Ecosystem services are ecological characteristics, functions, or processes that (in)directly contribute to society's wellbeing (Costanza et al.,1997). For example, the river provides freshwater, fish, or materials. This fact has been known for centuries as people settled around rivers and profited from their benefits. However, today’s rivers are under anthropogenic pressure. Therefore, the provided ecosystem services can be limited due to the degradation of the channels. We focussed on two contrast fluvial corridors located in the Odra and the Ostravice rivers, with the aim to i. map their general ecosystem services, ii. to highlight differences between them, and iii. to reveal ecosystem services of large wood that occurred in both localities but different spatiotemporal patterns. Both studied reaches are located in the city of Ostrava, in the Moravian-Silesian region, Czechia. The Ostravice River flows through an urbanized part of the city of Ostrava, so the river is channelized, and the river channel is incised below the level of the surrounding city. The Odra River flows through the Poodří protected landscape area. This river reach is still in a natural state with a meandering pattern. Based on the adjusted methodology of Burkhard et al. (2009) and Keele et al. (2020), we vectorized land use/land cover of 3km long and 300m wide fluvial corridors of both studied rivers.  We used aerial photos and additional map sources (e.g., agricultural map). Based on the vectorization, we noted the presence and absence of derived ecosystem services. In the next step, we focused on the differences between both studied areas. Large wood was also inventoried using the aerial photo, but in the case of the Ostravice River, we did not find large wood because of its regular post-flood removal. Therefore, we surrogated these data by field campaigns from the same year (2020) after one of the high-flow peak episode. Unsurprisingly, we found that both areas studied indicated contrasting patterns of ecosystem service. This outcome is based on the different representation of land use/land cover. In Ostravice, we detected only three general layers, however six layers were identified in Odra. Explicitly, we detected a lower number of ecosystem services (4), climate regulation, air quality, noise reduction, and recreation in Ostravice. In the case of Odra, we detected nine ecosystem services linked to climate regulation, air quality, biological control, erosion prevention, water regulation, recreation, and ecotourism, information and cognitive development, inspiration for culture, art and design, aesthetic information, and food. Also, ecosystem services of large wood differed, when there was also a difference between quantity a residence time. Large wood in Odra  has much higher wider variety of ecosystem services (10) than in Ostravice (3). This resulted mainly from contrast river management. In Odra, the agreement between river management and nature conservancy preserve large wood in natural state (Jarošek et al., 2022). Our findings showed that ecosystem services differ according to the different LU/LC and management strategy. The limitation of this research was that the map layers were examined by experts, which can also be related  to their specific science-based professional approach. Future research will focus on the sociological part of the research, which allows to go beyond this limitation.
}
%EndOfAbstractContent
%references
{
Burkhard, B., Kroll, F., Müller, F., Windhorst, W. (2009): Landscapes' capacities to provide ecosystem services - A concept for land-cover based assessments. Landscape Online, 15, 1-22.

Costanza, R., D'arge, R., De Groot, R., Farber, S., Grasso, M., Hannon, B., Limburg, K., Naeem, S., O'neill, R. V., Paruelo, J., Raskin, R. G., Sutton, P., Van Den Belt, M. (1997): The value of the world's ecosystem services and natural capital. Nature, 6630, 387, 253-260. 

Jarošek, R., Kletenský, D., Galia, T., Škarpich, V. (2022): Management a monitoring říčního dřeva v Odře v CHKO Poodří. Ochrana Přírody, 2, 2022, 18 - 22.

Keele, V., Gilvear, D., Large, A., Tree, A., Boon, P. (2019): A new method for assessing river ecosystem services and its application to rivers in Scotland with and without nature conservation designations. River Research and Applications, 35, 1338–1358.
}
%EndOfReferences
%end of conference contribution
%---------------------------------------------------------------------------------------------------