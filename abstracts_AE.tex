%start of conference contribution
\abstract
% Title 
{Coseismic Slope Failures Due to 2009 M\textsubscript{w} 6.2 Eq Near Poas, Costa Rica and Their Relevance for Landslide Research in Outer Western Carpathians} 
% EndOfTitle
%short author -- toc 
{Baroň et al.} 
%End short author -- toc
% Author(s) 
{Ivo Baroň1\textsuperscript{1}*, Petr Kycl\textsuperscript{2}, Joanna Mendez\textsuperscript{3}, Rostislav Melichar\textsuperscript{4}, Jan Klimeš\textsuperscript{1}, Filip Hartvich\textsuperscript{1}} 
% EndOfAuthor(s)
{\TLtag} 
%Tag, can be: empty, \KLtag (keynote lecture), \IStag (invited speaker), \CTtag (contributed talk) or \ITtag (invited talk)
% Affiliation(s)
{
\textsuperscript{1}Institute of Rock Structure and Mechanics, the Czech Academy of Sciences, V Holešovičkách 94/41, 182 09 Prague, Czech Republic\\
\textsuperscript{2}Czech Geological Survey, Klárov 3, 118 21, Prague, Czech Republic\\
\textsuperscript{3}San José, Costa Rica\\
\textsuperscript{4}Faculty of Science, Masaryk University, Kotlářská 2, 602 00 Brno, Czech Republic
}
%} % EndOfAffiliation(s)
{}  %e-mail
%keywords
{Co-seismic landslides, prehistoric earthquakes, Poas Volcano, Outer Western Carpathians}
%EndOfKeywords
%abstract content
{Co-seismic landslides belong among the most serious secondary effects of destructive earthquakes and are considered as an environmental measure of the particular earthquake´s intensity (Michetti et al., 2007). Correct identification and differentiation of respective rainfall and co-seismic prehistoric landslides may therefore significantly improve identification and characterization of particular paleoearthquakes and the seismic hazard assessment. The paper focuses on characterizing source areas of the primary co-seismic rotational landslides at Poas Volcano (Costa Rica), triggered on 8 of January 2009 by the Mw 6.2 Cinchona earthquake with the hypocentral depth of 4.5 km (Mendez et al., 2009). We aim to highlight the possible morphometric parameters for recognizing their paleoseismic origin. Our direct field observation revealed that the primary rotational landslides had undergone extremely high mass mobilization resulting in high depletion rate of the source areas even at relatively moderate slopes near their upper edges. Further, we applied these outcomes onto an alternative area and searched for similar landforms in the flysch belt of the Outer Western Carpathians in the regions that are situated close to recently recognized active faults with other evidence of the co-seismic effects and which might represent earthquake-triggered landslides.
}
%EndOfAbstractContent
%references
{Méndez J, Soto GJ, Zamora N, Vargas A, Sjöbohm L, Bonilla E, Barahona D, Solís L, Kycl P, \& Baroň I (2010) Geología de los deslizamientos provocados por el terremoto de Cinchona, Costa Rica (Mw 6,2; 8 de Enero del 2009) en la ruta 126 (Varablanca – San Miguel). X° Congreso Nacional de Geotecnia  San José, Costa Rica -- Agosto 2009.

Michetti AM et al. (2007) Environmental Seismic Intensity Scale 2007-ESI 2007. In Memorie Descrittive Della Carta Geologica d’Italia, Servizio Geologico d’Italia, Dipartimento Difesa del Suolo; APAT: Roma, Italy, 2007; Vol. 74, pp. 7--54
}
%EndOfReferences
%end of conference contribution
%start of conference contribution
\abstract
% Title 
{DEPOSITION AND MOBILIZATION OF MICROPLASTICS IN A LOW-ENERGY FLUVIAL ENVIRONMENT} 
% EndOfTitle
%short author -- toc 
{Balla and Kiss} 
%End short author -- toc
% Author(s) 
{Alexia Balla\textsuperscript{1}*, Tímea Kiss\textsuperscript{1}} 
% EndOfAuthor(s)
{\TLtag} 
%Tag, can be: empty, \KLtag (keynote lecture), \IStag (invited speaker), \CTtag (contributed talk) or \ITtag (invited talk)
% Affiliation(s)
{
\textsuperscript{1}Department of Geoinformatics, Physical and Environmental Geography, University of Szeged, Egyetem u. 2-6, 6722 Szeged, Hungary
}
%} % EndOfAffiliation(s)
{balla.alexia5@gmail.com}  %e-mail
%keywords
{microplastic, hotspots, fibre, downstream changes, in-channel forms, deposition}
%EndOfKeywords
%abstract content
{The emission of organic and inorganic pollutants and plastics through improperly treated garbage and wastewater is a major environmental issue. Many studies focus on this increasingly significant environmental risk, which affects rivers, lakes, and seas. Plastics emitted into the environment (e.g., synthetic clothing, tires, packaging, PET bottles, industrial and electronic waste) are exposed to various chemical and physical impacts resulting in their fragmentation. These processes result in microplastics (≤ 5 mm), which can be fibres, spheres, shreds, and fragments. These particles are transported along with the natural sediments in rivers, and their behaviour is probably similar to the natural sediments transported by the river, however, no precise data exist on it. 

Therefore, our aims are not only to quantify the microplastic pollution in fluvial sediments, but we also aimed to investigate the hydrological and geomorphological factors that affect their deposition. The sediments of the Tisza River (Central Europe) and its main tributaries were sampled in 2019 (Kiss et al. 2021) and 2020 (Kiss et al. 2022) from its source in Ukraine to its Danubian confluence in Serbia. Fine-grained (clay, silt) and coarse-grained (sand, gravel) sediment samples were collected from various in-channel forms (e.g., point-bars, side-bars, and sediment sheets). After laboratory extraction, the microplastic content of the sediments was counted using a microscope.

The microplastics in the sediment samples were mostly fibres (98\%), referring to the discharge of improperly treated wastewater into the river system. In 2020 the mean microplastic content of the Tisza’s sediments was 1770\textpm1329 item/kg, whereas the sediments of the tributaries contained more microplastics (1885\textpm1541 item/kg). The microplastic pollution increased downstream, thus the Middle Tisza was the most polluted section, however, the Lower Tisza had lower numbers. The patchy characteristic of the pollution is well reflected by the fact, that both the minimum (Záhony: 237 item/kg) and maximum (Tiszadada: 6707 item/kg) contamination were measured in the Middle Tisza, Hungary.
Contrary to our hypothesis, the fine-grained samples contained less microplastic contamination (mean: 1406\textpm788 item/kg) than the coarse-grained samples (2226\textpm1735 item/kg). 

Among the three studied in-channel forms, the point-bars were the most polluted (2049\textpm1218 item/kg), while the average pollution of the side-bars (1692\textpm1335 item/kg) and sediment sheets (1584±1464 item/kg) were lower by 18\% and 23\%, respectively. Besides, the microplastic pollution of the various forms shows different downstream changes. We recommend to collect samples from side-bars, which are common forms in rivers, and they reflect the average values of the microplastic content of river sections. Sediment sheets are the least favourable forms for microplastic sampling, as their longitudinal variations are different than of the other forms or the section averages, though, if the sampling aims to evaluate the microplastic content (in suspended form) of a falling limb of a flood, the sediment sheet is a perfect choice.

The comparison between the 2019 (Kiss et al. 2021) and 2020 data showed that by 2020, the number of microplastics in the sediments of the Tisza had decreased by 30\%, while in the sediments of the tributaries, it decreased even more (by 48\%). The former pollution hotspots have been emptied, and the plastic-contaminated sediments have been rearranged. Therefore, long-term monitoring is needed to understand the spatio-temporal distribution of microplastics in a fluvial environment.
}
%EndOfAbstractContent
%references
{Kiss T, Fórián Sz, Szatmári G, Sipos Gy, (2021) Spatial distribution of microplastics in the fluvial sediments of a transboundary river – A case study of the Tisza River in Central Europe, Science of the Total Environment 785, 147306

Kiss T, Gönczy S, Nagy T, Mesaroš M, Balla A, (2022) Deposition and Mobilization of Microplastics in a Low-Energy Fluvial Environment from a Geomorphological Perspective. Applied Sciences 2022, 12, doi:10.3390/app12094367.
}
%EndOfReferences
%end of conference contribution

\abstract
{Temporal Relationship of Deglaciation Phases and Palaeodischarges on the Catchment of River Maros, Central Europe} % Title
{Bartyik et al.} %short author -- toc
{Tamás Bartyik$^1*$, György Sipos$^1$, Dávid Filyó$^1$, Tímea Kiss$^1$, Petru Urdea$^2$ , Fabian Timofte$^2$} % Author(s)
{\KLtag} % Tag, can be: empty, \KLtag (keynote lecture), \IStag (invited speaker), \CTtag (contributed talk) or \ITtag (invited talk)
{$^1$Department of Geoinformatics, Physical and Environmental Geography, University of Szeged, Szeged, Hungary\\
	$^2$ Department of Geography, West University of Timișoara, Timișoara, Romania
} % Affiliation(s)
{bartyikt@geo.u-szeged.hu}  %e-mail
{OSL dating, River Maros deglaciation, luminescence sensitivity, sediment delivery}%keywords
{River Maros has one of the largest alluvial fans in the Carpathian Basin. On the surface of the fan several very wide, braided channels can be identified, resembling increased discharges during the Late Glacial. In our study we investigated the activity period of the largest channel of them, formed under a bankfull discharge three times higher than present day values. Previous investigations dated the formation of the palaeochannel to the very end of the Pleistocene by dating a point bar series upstream of the selected site (Kiss et al. 2014). Our aim was to obtain further data on the activity period of the channel and to investigate temporal relationships between maximum palaeodischarges, deglaciation phases on the upland catchment and climatic amelioration during the Late Pleistocene. The age of sediment samples was determined by optically stimulated luminescence (OSL). The investigation of the luminescence properties of the quartz extracts also enabled the assessment of sediment delivery dynamics in comparison to other palaeochannels on the alluvial fan. 
OSL age results suggest that the activity of the channel is roughly coincident with, but slightly older than the previously determined ages, meaning that the main channel forming period started at 13.50\textpm0.94 ka and must have ended by 8.64\textpm0.82 ka (Kiss et al. 2015). This period cannot directly be related to the major phases of glacier retreat on the upland catchments, and in terms of other high discharge channels only the activity of one overlaps with a major deglaciation phase at \textasciitilde17--18 ka (Ruszkiczay-Rüdiger et al. 2016). Based on these, high palaeodischarges can be rather related to increased Late Glacial runoff, resulted by increasing precipitation and scarce vegetation cover on the catchment. Meanwhile, the quartz luminescence sensitivity of the investigated channel refers to fast sediment delivery from upland subcatchments. Therefore, the retreat of glaciers could affect alluvial processes on the lowland by increasing sediment availability, which contributed to the development of large braided palaeochannels.
}%abstract
{Kiss T, Sümeghy B, Sipos Gy (2014) Late Quaternary paleo-drainage reconstruction of the Maros River Alluvial Fan. Geomorphology 204, 49–60.
	
Kiss T, Hernesz P, Sümeghy B, Györgyövics K, Sipos Gy (2015) The evolution of the Great Hungarian Plain fluvial system - Fluvial processes in a subsiding area from the beginning of the Weichselian. Quaternary International 388, 142–155. 
	
Ruszkiczay-Rüdiger Zs, Kern Z, Urdea P, Braucher R, Madarász B, Schimmelpfennig I, ASTER TEAM (2016) Revised deglaciation history of the Pietrele-Stânişoara glacial complex, Retezat Mts, Southern Carpathians, Romania. Quaternary International 415, 216–229.
}%references
%end of abstract

%start of conference contribution
\abstract
% Title 
{Chronology and Morphology of Rock Glaciers in the Western Tatra Mts., Western Carpathians} 
% EndOfTitle
%short author -- toc 
{Dlabáčková et al.} 
%End short author -- toc
% Author(s) 
{Tereza Dlabáčková\textsuperscript{1}*, Zbyněk Engel\textsuperscript{1}, Régis Braucher\textsuperscript{2}, Tomáš Uxa\textsuperscript{3}, Aster Team\textsuperscript{2}} 
% EndOfAuthor(s)
{\KLtag} 
%Tag, can be: empty, \KLtag (keynote lecture), \IStag (invited speaker), \CTtag (contributed talk) or \ITtag (invited talk)
% Affiliation(s)
{
\textsuperscript{1}Faculty of Science, Charles University, Prague, Czech Republic \\
\textsuperscript{2}CEREGE CNRS Aix Marseille Univ., IRD, INRA, Collége de France, Aix-en-Provence, France\\
\textsuperscript{3}Institute of Geophysics, Czech Academy of Science, Prague, Czech Republic
}
%} % EndOfAffiliation(s)
{tereza.dlabackova@natur.cuni.cz}  %e-mail
%keywords
{rock glacier, exposure dating, Schmidt hammer test, morphometry, Western Tatra Mts.}
%EndOfKeywords
%abstract content
{Relict rock glaciers are significant indicators of paleoclimate and permafrost conditions and their morphology reflects the development of mountain slopes after the deglaciation. The dating of rock glaciers therefore contributes to a better understanding of the post-glacial development of high mountain environments. In this paper we present initial results of cosmogenic 10Be exposure and relative-age dating of five rock glaciers in the Western Tatra Mts. Three to six rock samples were collected for the cosmogenic nuclide dating on investigated rock glaciers and the Schmidt hammer test was carried out on the same boulders (30 measurements per boulder). The weighted mean exposure ages for the sampled rock glaciers range from 11.9\textpm0.4 ka in the Smutná Valley to 17.3\textpm0.7 ka in the Spálená Valley whereas mean the R-values range from 37.7\textpm4.8 (Spálená Valley) to 42.0\textpm3.8 (Smutná Valley). The fronts of the rock glaciers extend to elevations of \textasciitilde1370 to 1810 m a.s.l. indicating the lower boundary of discontinuous paleopermafrost. Younger exposure ages obtained in the north-facing valleys suggest later ice decay in shaded mountain slopes compared to the south-facing valleys. The cosmogenic data imply that rock glaciers in the Western Tatra Mts. developed during the Lateglacial period following the glacier retreat from the main valleys.  Rock glacier stabilization in this region started around 15 ka and terminated at the turn of the Late Glacial and Holocene. The observed ages are consistent with the published exposure data from moraines and rock glaciers in the Tatra Mts.

\textit{The research was supported by the GAUK project No. 1528119 (Morphology and age chronology of rock glaciers in the Western Tatra Mts.).}
	
}
%EndOfAbstractContent
%references
{
}
%EndOfReferences
%end of conference contribution

%---------------------------------------------------------------------------------------------------

%start of conference contribution
\abstract
% Title 
{CGS Development Cooperation Project "Georgia 2021" - Assessment of Geological Hazards in the Kazbegi Area} 
% EndOfTitle
%short author -- toc 
{Dostalík et al.} 
%End short author -- toc
% Author(s) 
{Martin Dostalík\textsuperscript{1}*, Jan Novotný\textsuperscript{1}, Jan Jelének\textsuperscript{1}, Lucie Koucká\textsuperscript{1}, Petr Kycl\textsuperscript{1}, Vít Baldík\textsuperscript{1} Martin Kýhos\textsuperscript{1}} 
% EndOfAuthor(s)
{\KLtag} 
%Tag, can be: empty, \KLtag (keynote lecture), \IStag (invited speaker), \CTtag (contributed talk) or \ITtag (invited talk)
% Affiliation(s)
{
	\textsuperscript{1}Czech Geological Survey, Klárov 131/1, Prague, Czech Republic
}
%} % EndOfAffiliation(s)
{martin.dostalik@geology.cz}  %e-mail
%keywords
{development cooperation, Georgia, Caucasus, Kazbek, debris flow, landslides, hanging glacier, geological hazard}
%EndOfKeywords
%abstract content
{The Challenge Fund, as one of the components of the Czech-UNDP Partnership (\url{https://undp.cz}) between the Czech Development Agency and the UN Development Program, supported the CGS project called "Methodology for assessing the territory in terms of the danger of torrential currents using innovative technologies". The project was solved by CGS experts in engineering geology together with CGS experts in remote sensing. The role of partner organization has been accepted by the Georgian National Environment Agency (NEA) as the institution responsible for monitoring, assessing and mapping geological hazards in Georgia. 

The general objective was to eliminate the dangers associated with the very frequent catastrophic slope movements, technically called debris flow, which cause serious socio-economic damage and loss of life, especially in the mountainous areas of Georgia. Geodynamic natural processes are very difficult to prevent, but it is possible to mitigate their consequences to avoid disasters and losses. This is mainly due to effective spatial planning with high-quality engineering and geological data. The project offered conceptual solutions for risk reduction and prevention. Georgia's traditional risk management system is rather reactive. That means to deal with the consequences of natural disasters and therefore requires significant recovery costs. The results of this cooperation have contributed to the improvement of spatial planning, which will lead to a reduction in remediation costs. This model study has also the ambition to serve as a model approach for other highland regions of Georgia.

The aim of the cooperation was to develop a harmonized methodology for geological risk assessment and to implement it in the activities of the partner organization NEA. The object of interest of the research was the surroundings of Mount Kazbek, which is historically known for the high frequency of catastrophic debris flows threatening strategic infrastructure. However, this area is also threatened by other geodynamic processes, such as: various types of slope deformations, floods or avalanches, etc. This relatively young 5,000 meters high volcano on the northern border of Georgia is covered by glaciers that are melting according to a global trend. Some of the tragic debris flows, for example from 2002 and 2014, were caused by the rupture of a mountain glacier. It turns out that climate change in such glacial areas can increase the susceptibility to geodynamic processes. 

Two expeditions to the Kazbegi area were organized as part of this international project. In addition to theoretical and field workshops, CGS experts managed to visit and document several hard-to-reach alpine areas in the Kazbegi region. The most important results of the project are in particular:
\begin{enumerate}
\item  a series of maps of the main geological hazards of the Kazbegi area such as hanging glaciers and rock walls prone to collapse and activating debris flows, sources of transportable material, previously documented debris flow paths, glacial lakes threatening to rupture;
\item the interactive Flow 2020 web tool, which assists in the assessment of geological risks;
\item joint Czech-Georgian Scientific Publication (Dostalík et al. 2022*);
\item a series of theoretical and field workshops in which know-how in the field of geological risk prevention was transferred;
\item stimulation of cooperation between CGS and NEA.
\end{enumerate}	

The project was very positively evaluated both by NEA, which appreciated the amount of work done in a very short time, as well as by the sponsoring organization Czech-UNDP Partnership for the SDGs, which emphasized the thoroughness and complexity of processing the results and very good communication by CGS. 

This year, we would like to build on the successful cooperation with a new project proposal dealing with the issue of geological hazards in the Khevsureti region in eastern Georgia.

\noindent
\textbf{Acknowledgements:}
This research contributes to the Project “Methodology for the area assessment in terms of debris flow hazard using innovative technology” funded by the Czech-UNDP Partnership for SDGs – Challenge Fund - the Czech Solution for SDGs Ref. UNDPIRH-202005-CFP04-CZECH INNOVATION CHALLENGE. 
We also cooperates with the RENS project No. SS02030023 “Rock environment and natural resources".
The research is a part of the Strategic Research Plan of the CGS (DKRVO/ČGS) Topic: Geological risk research.
}
%EndOfAbstractContent
%references
{Dostalík M., Novotný J., Jelének J. (2020a): Koncepční inženýrskogeologický model na příkladu hodnocení geologického hazardu oblasti horského masivu Kazbek –Džimara. In Jana Frankovská, Martin Ondrášik: Inžinierska geológia 2020. ISBN 978-80-227-5014-1

Dostalík M., Novotný J., Kurtsikidze O., Gaprindashvili G. (2020b): Catastrophic Debris Flows in Kazbegi Mountain Area, Georgia – Use of Available Free Internet Information as a Source to Generate Conceptual Engineering Geological Model. – Lowland Technology International Journal 22, 1, pp. 48-63. ISSN 1344-9656
	
Dostalík M., Novotný J., Gaprindashvili M., Kurtsikidze O., Gaprindashvili G. (2022*): Examples of infrastructure objects failure and hazard in terms to the engineering geological model importance – Kazbegi municipality, Georgia, IAEG XIV Congress 2022 in Chengdu, China (*in review proceedings)
}
%EndOfReferences
%end of conference contribution

%start of conference contribution
\abstract
% Title 
{DEGLACIATION IN KAZBEGI HIGH MOUNTAIN REGION IN GEORGIA AND ITS IMPACT ON THE ACTIVATION OF CATASTROPHIC DEBRIS FLOWS} 
% EndOfTitle
%short author -- toc 
{Dostalík and Novotný} 
%End short author -- toc
% Author(s) 
{Martin Dostalík\textsuperscript{1}*, Jan Novotný\textsuperscript{1}} 
% EndOfAuthor(s)
{\KLtag} 
%Tag, can be: empty, \KLtag (keynote lecture), \IStag (invited speaker), \CTtag (contributed talk) or \ITtag (invited talk)
% Affiliation(s)
{
\textsuperscript{1}Czech Geological Survey, Klárov 131/1, Prague, Czech Republic
}
%} % EndOfAffiliation(s)
{martin.dostalik@geology.cz}  %e-mail
%keywords
{Caucasus, Kazbek, Debris flow, Deglaciation, Hanging glacier, Geological hazard}
%EndOfKeywords
%abstract content
{Alpine glaciers have been melting around the world due to climate change. With the retreat of these glaciers, certain parts of them get into a very unstable position on steep slopes. Dangerous hanging glaciers with a high susceptibility to collapse are generated this way. Likewise, the instability of the rock walls originally supported by glaciers increases. 
Research on this topic was conducted in the High Caucasus in the high mountain region of Kazbegi in Georgia. By comparing historical maps, it was possible to demonstrate the intensity of glacier retreat in this area, which is historically known for the frequent catastrophic debris flows initiated by the collapse of a hanging glacier. It turned out that in some glacial valleys of the Kazbek massif debris flows are produced much more often then in the others (Dostalík et al., 2020a). 

Currently, the greatest risk is posed in the Devdoraki Glacier Valley, where catastrophic debris flows have been regularly activated (Dostalík et al., 2020b). A map of the Devdoraki Glacier Valley and its surroundings was compiled by field expeditions, including drone survey combined with an analysis of the available satellite data. The map involves the extent of all recorded rock and glacial avalanches at the this valley closure and simultaneously presents locations with increased debris flow activity. It is clear from the map that these areas clearly correspond to the areas of mapped hanging glaciers as the source of debris flow phenomena.

\noindent
\textbf{Acknowledgements:}
\textit{This research contributes to the Project “Methodology for the area assessment in terms of debris flow hazard using innovative technology” funded by the Czech-UNDP Partnership for SDGs – Challenge Fund - the Czech Solution for SDGs Ref. UNDPIRH-202005-CFP04-CZECH INNOVATION CHALLENGE. 
Also we cooperates with the RENS project No. "SS02030023 Rock environment and natural resources" 
The research is a part of the Strategic Research Plan of the CGS (DKRVO/ČGS) Topic: Geological risk research}
}
%EndOfAbstractContent
%references
{Dostalík M., Novotný J., Jelének J. (2020a): Koncepční inženýrskogeologický model na příkladu hodnocení geologického hazardu oblasti horského masivu Kazbek –Džimara. In Jana Frankovská, Martin Ondrášik: Inžinierska geológia 2020. ISBN 978-80-227-5014-1
	
Dostalík M., Novotný J., Kurtsikidze O., Gaprindashvili G. (2020b): Catastrophic Debris Flows in Kazbegi Mountain Area, Georgia – Use of Available Free Internet Information as a Source to Generate Conceptual Engineering Geological Model. – Lowland Technology International Journal 22, 1, pp. 48-63. ISSN 1344-9656
}
%EndOfReferences
%end of conference contribution



