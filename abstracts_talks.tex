\abstract
{Human intervention and flood events as key factors affecting the recent degradation of the Hornád River} %short author -- toc
{Labaš and Kidová}
{Peter Labaš, Anna Kidová}
 % Author(s)
{\TLtag} % Tag, can be: empty, \KLtag (keynote lecture), \IStag (invited speaker), \CTtag (contributed talk) or \ITtag (invited talk)
{Department of Physical Geography, Geomorphology and Natural Hazards, Institute of Geography, Slovak Academy of Science, Štefánikova 49, 814 73 Bratislava} % Affiliation(s)
{geoglaba@savba.sk}  %e-mail
{meandering; river degradation; multi-temporal analysis; flood event; the Hornád River}%keywords
{
At the turn of the 19th and 20th century, the Hornád River in Slovakia in its middle reach was typical by passing the Hornád basin, typical by several levels of river terraces (Michaeli, 2001), well-developed floodplains and water-gaps. The wide floodplains were characteristic by free meanders which had been in the 20th century affected by human interventions, peaked in the 50s. The anthropogenic impact resulted in channel shortening and narrowing, river sinuosity lowering, erosion-accumulation processes decreasing and free meanders loss. Additionally, the long-term discharge reduction had caused simplification of channel planform in stream sections without or with minimal anthropogenic impact. This trend was later abrupted by the flood event series in 2004, 2008 and 2010. The flood events were also documented as an important factor in a morphological change in the lower part of the Hornád River in Hungary (Kiss and Blanka, 2011; Kiss and Blanka, 2012). Environmentally changed conditions (anthropogenic impact and floods) on the 72 km long channel planform of the meandering Hornád River in Slovakia were observed on three types of river segments (natural, regulated, water-gap) by seven sets of data, including the second and third military survey and five orthophoto mosaics and aerial photos (1949, 1986, 2002, 2013, 2016). The Hornád River in the pre-regulation period was represented by a natural meandering river planform (45.8\%) with a high occurrence of in-channel landforms, where the lateral bar area prevailed. At present, due to the simplification of the river channel planform, only 26\% of the river segments with ongoing natural erosion-accumulation processes and the last two locations with free meanders on the middle reach of the Hornád River remained.

\vspace{0.5em}
\noindent
\textbf{Acknowledgement:}\textit{This research was supported by the Science Grant Agency (VEGA) of the Ministry of Education of the Slovak Republic and the Slovak Academy of Sciences (02/0086/21).}}
{Kiss, T. and Blanka, V. (2011) Effect of different water stages on bank erosion, case study on River Hernád, Hungary, Carpathian Journal of Earth and Environmental Sciences, 6(2), pp. 101-108

Kiss, T. and Blanka, V. (2012) River channel response to climate- and human-induced hydrological changes: Case study on the meandering Hernád River, Hungary, Geomorphology, 175–176, pp. 115–125. doi: 10.1016/j.geomorph.2012.07.003.

Labaš, P. and Kidová, A., (2022). Anthropogenic and environmental impacts on the recent morphological degradation of the meandering Hornád River, Geografický časopis,  74(2), in press

Michaeli, E. (2001). Georeliéf hornádskej kotliny, Geografické práce, 9(2), 153 p. 
}%references

%TODO: zkontrolovat
\abstract
{Identification of benches and ledges within the braided-wandering floodplain formation} % Title
{Labaš et al.} %short author -- toc
{Peter Labaš, Anna Kidová, Šárka Horáčková, Milan Lehotský, Miloš Rusnák
} % Author(s)
{\KLtag} % Tag, can be: empty, \KLtag (keynote lecture), \IStag (invited speaker), \CTtag (contributed talk) or \ITtag (invited talk)
{Department of Physical Geography, Geomorphology and Natural Hazards, Institute of Geography, Slovak Academy of Science, Štefánikova 49, 814 73 Bratislava} % Affiliation(s)
{geoglaba@savba.sk}  %e-mail
{braided-wandering river; floodplain evolution; benches; ledges; gravel-bed river}%keywords
{The Belá River in the north part of Slovakia represents a braided-wandering river system (Kidová et al., 2016) with repeatedly destroying and re-forming floodplain, as a result of lateral shifting of an active river zone of braided rivers (Haschenburger and Cowie, 2009). These processes depend on flood events and a large volume of sediment transportation. Benches as unstable (Erskine and Livingstone, 1999), fine sediment storage (Kemp, 2004; Vietz et al., 2005a, 2005b) are formed by flood events (Webb, Erskine a Dragovich, 2002) in the margin of the active river zone of the Belá River and they are also destroyed by another even larger flood events (Erskine a Peacock, 2002). However, in the last decades, we are able to observe changes in the magnitude of flood events and the volume of sediment transportation of the Belá River (Kidová, et al., 2016b). The changes caused a simplification of the river planform, leading to a better-developed mature floodplain. The benches in these parts are a well-preserved component of the floodplain. Moreover, local incision of the river channel led to more-level floodplain development with benches set into them. While incision forbids to another floodplain development results of its mutual effect with flood events are ledges, as a level of bank erosion (Lehotský et al., 2015). Following previous research supplemented by new field surveys, we are focused to study of river floodplain evolution of the Belá River. In the end, it will be possible to formulate a process-oriented hypothesis of the recent lateral and vertical development of the floodplain.}%abstract
{Erskine, W. D. and Livingstone, E. (1999) In-channel benches: the role of floods in their formation and destruction on bedrock confined rivers, Varieties of Fluvial Form, (January 1999), s. 445–475.

Erskine, W. D. and Peacock, C. T. (2002) Late holocene flood plain development following a cataclysmic flood, The Structure, Function and Management Implications of Fluvial Sedimentary Systems, (276), s. 177–184.

Haschenburger, J. K. and Cowie, M. (2009) Floodplain stages in the braided Ngaruroro River, New Zealand”, Geomorphology, 103(3), s. 466–475. \doi{10.1016/j.geomorph.2008.07.016}.

Kemp, J. (2004) Flood channel morphology of a quiet river, the Lachlan downstream from Cowra, southeastern Australia, Geomorphology, 60(1–2), s. 171–190. \doi{10.1016/j.geomorph.2003.07.007}.

Kidová, A., Lehotský, M. and Rusnák, M. (2016a) Geomorphic diversity in the braided-wandering Belá River, Slovak Carpathians, as a response to flood variability and environmental changes, Geomorphology, 272, s. 137–149.\doi{10.1016/j.geomorph.2016.01.002}.

Kidová, A., Lehotský, M. and Rusnák, M. (2016b) Morfologické zmeny a manažment divočiaco-migrujúceho vodného toku Belá, GEOMORPHOLOGIA SLOVACA ET BOHEMICA, 16(2), s. 60.

Lehotský, M., Kidová, A. and Rusnák, M. (2015) Slovensko-anglické názvoslovie morfológie vodných tokov, GEOMORPHOLOGIA SLOVACA et BOHEMICA, 15(1), s. 61.

Vietz, G., Stewardson, M. and Rutherfurd, I. (2005a) Not all benches are created equal: Proposing and field testing an in-channel river bench classification, v Proceedings of the 4th Australian Stream Management Conference, s. 629–635.

Vietz, G., Stewardson, M. and Rutherfurd, I. (2005b) Variability in river bench elevation and implications for environmental flow studies, v, s. 8.

Webb, A. A., Erskine, W. D. and Dragovich, D. (2002) Flood-driven formation and destruction of a forested flood plain and in-channel benches on a bedrock-confined stream: Wheeney Creek, southeast Australia, IAHS-AISH Publication, (276), s. 203–210.
}%references

