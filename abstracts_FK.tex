%start of conference contribution
\abstract
% Title 
{Development and recent dynamics of rockslide in the Hercynian orogeny (the Rudohorský potok basin; the Hrubý Jeseník Mts.)} 
% EndOfTitle
%short author -- toc 
{Fabiánová A.} 
%End short author -- toc
% Author(s) 
{Andrea Fabiánová\textsuperscript{1}*} 
% EndOfAuthor(s)
{\KLtag} 
%Tag, can be: empty, \KLtag (keynote lecture), \IStag (invited speaker), \CTtag (contributed talk) or \ITtag (invited talk)
% Affiliation(s)
{
\textsuperscript{1}Faculty of Science, University of Ostrava, Ostrava, Czech Republic
}
%} % EndOfAffiliation(s)
{andrea.fabianova@osu.cz}  %e-mail
%keywords
{Rockslide, dendrogeomorphology, structural compass measurements, kinematic analysis, Schmidt-hammer, the Hrubý Jeseník Mts., I\textsubscript{t} index.}
%EndOfKeywords
%abstract content
{A rockslide in the Hrubý Jeseník mountains was studied to unravel its development and recent activity. Four methods were applied: dendrogeomorphologic dating, rock hardness measurement using Schmidt-hammer, structural compass measurement, and kinematic analysis. For the purposes of dendrogeomorphologic dating, reaction wood, eccentric growth and sudden growth suppression of 43 Norway spruces (\textit{Picea abies} (L.) Karst.) were used. The Schmidt-hammer was used to measure hardness of rocks of rock walls and colluvial accumulations. Structural compass measurements were used to reveal the strike and dip direction of joints and foliation planes of the landslide slope. Kinematic analysis was performed to reveal possible slope predisposition to planar failures, wedge failures and toppling. According to the results, the rockslide is in its dormant state. Despite its dormancy, movements of shallow-bedded blocks were recorded as well as probable movements of a deeper character of a larger landslide unit. Movements are indistinct, probably within mm to cm. The landslide blocks move by sliding in the direction of the foliation inclination, toppling of the blocks is present as well, and rotational character of rock block movement is also probable. Overall, the surface of the landslide body is considerably weathered evenly over its entire surface. It results in rock walls copying tectonic faults, and numerous rock accumulations originating from the rockfalls, covering the rest of the slope surface and accumulating at its base. In case of dendrogeomorphology, it was the first use of this dating method on almost bare rock surfaces. It turned out that it is possible to detect movements of rock blocks, but at least two trees per block are needed alongside with evaluation of their spatio-temporality, whether the trees react logically on the landslide blocks/units. Moreover, it is suitable to use a three-year period during which the trees on the blocks could react. The result of this work offers a unique view of the only landslide examined so far in the Hrubý Jeseník mountains and tests/complements the methodological procedures for dendrogeomorphologic dating of these specific slope deformations.
}
%EndOfAbstractContent
%references
{
}
%EndOfReferences
%end of conference contribution

%---------------------------------------------------------------------------------------------------


%start of conference contribution
\abstract
% Title 
{LOCAL GEOMORPHOLOGICAL FACTORS INFLUENCING FLOOD HAZARD, LOWER TISZA, HUNGARY} 
% EndOfTitle
%short author -- toc 
{Kiss and Fehérváry} 
%End short author -- toc
% Author(s) 
{Tímea Kiss\textsuperscript{1}*, István Fehérváry\textsuperscript{1,2}} 
% EndOfAuthor(s)
{\KLtag} 
%Tag, can be: empty, \KLtag (keynote lecture), \IStag (invited speaker), \CTtag (contributed talk) or \ITtag (invited talk)
% Affiliation(s)
{
	\textsuperscript{1}Department of Geoinformatics, Physical and Environmental Geography, University of Szeged, Szeged, 6722, Hungary, Egyetem u. 2–6
	\textsuperscript{2}Directorate for Environmental Protection and Water Management of Lower Tisza District,
	Szeged, 6722, Hungary, Stefánia 4.
}
%} % EndOfAffiliation(s)
{kisstimi@gmail.com}  %e-mail
%keywords
{flood hazard, local factors, aggradation, channel narrowing, riparian vegetation, levee lowering}
%EndOfKeywords
%abstract content
{Besides catchment-scale processes, various local processes influence the peak flow level, however, they are often neglected in flood management. The aim of our study is to analyse those local geomorphological processes that contribute to rising local flood levels in the regulated channel and on the artificially confined floodplain of the Tisza River, Hungary. Our goals were to evaluate the role of (1) cross-sectional channel changes, (2) overbank floodplain aggradation, (3) riparian vegetation changes on local flood level increases since the late 19th century and early 20th century; and (4) to assess the elevation changes of artificial levees. 
	
Along the Lower Tisza (92 km) since 1931 the channel narrowed by 9\% (max. 30\%) and its cross-sectional area decreased by an average of 2\% (max. 22\%). These in-channel processes increased flood levels by an average of 13 cm (max. 134 cm). Simultaneously, the flood conveyance capacity of the floodplain decreased by overbank floodplain accumulation, which is 1.1 m in average (max. 2.6 m). The accelerated aggradation further increased the flood levels by 112 cm. The land-use of the floodplain also changed considerably (from meadows to forests). As the vegetation roughness (Manning’s n) increased from 0.048 to 0.11, it increased flood levels by 42 cm (Scenario A roughness increase: 10\%) or 139 cm (Scenario B roughness increase: 30\%) on average.

By overlapping these data, the results showed that since the 19th century river regulation works the actual flood level increased by an average of 175 cm (maximum = 350 cm) in the case of Scenario A and 272 cm (maximum = 443 cm) in the case of Scenario B. The latter is more consistent with the actual flood stage measurements. 

As these processes are still active, further increase in the flood level could be expected. In addition, the height of artificial levees decreased by an average of 23 cm (max 75 cm). Therefore, some levee sections became more susceptible to overtopping during record high floods, especially along the eastern levee. 

Based on this approach, local hydrological managers can identify the processes that contribute more to peak flow level increase at a given location, and determine the correct management actions at the correct locations, which could lead to decrease in local peak flow levels.
}
%EndOfAbstractContent
%references
{
}
%EndOfReferences
%end of conference contribution

%---------------------------------------------------------------------------------------------------
%start of conference contribution
\abstract
% Title 
{Linking Geomorphological and Archaeological Research at Geoarcheological Site Holedná Hill in Brno, Czech Republic} 
% EndOfTitle
%short author -- toc 
{Kubalíková et al.} 
%End short author -- toc
% Author(s) 
{Lucie Kubalíková*, Karel Kirchner, František Kuda} 
% EndOfAuthor(s)
{\KLtag} 
%Tag, can be: empty, \KLtag (keynote lecture), \IStag (invited speaker), \CTtag (contributed talk) or \ITtag (invited talk)
% Affiliation(s)
{Institute of Geonics of the Czech Academy of Sciences, Drobného 28, 602 00 Brno, Czech Republic
}
%} % EndOfAffiliation(s)
{lucie.kubalikova@ugn.cas.cz}  %e-mail
%keywords
{geoarcheology, geoheritage, cultural heritage, Bronze Age, environmental education, geotourism}
%EndOfKeywords
%abstract content
{The Holedná Hill (Brno, Czech Republic) can be considered an example of geocultural (or more precisely geoarcheological) site. It is important from the geomorphological point of view and it includes specific cultural and archaeological issues that are closely related to the geodiversity (especially anthropogenic landforms). The site is situated in the north-western part of the Brno City and represents the end of ridge above the deep incised Svratka Valley. Based on the analysis of LiDAR Data, unknown elongated structures were identified. The fieldworks proved the anthropogenic origin of these landforms (embankments composed of stones and small boulders, sometimes with shallow ditches along them). The structures symmetrically surround the dominant peak of the Holedná Hill. Consequently, an archaeological research was conducted. Based on the occurrence of Neolithic and Eneolithic localities in the vicinity, a similar age was estimated. Nevertheless, the radiocarbon dating of organic material confirms the age between 1200 -- 1050 BC which refers to the Bronze Age. Until now, only the accidental occurrence of ceramics was found. The question is what the purpose of the structures on the Holedná Hill was: the hypothesis about the sacred place or abandoned unfinished fortress was formulated.
	
The Holedná Hill is also important from the geological point of view: the bedrock consists of Proterozoic rocks of the Brno massif (diorites with veins of metaryoliths), the occurrence of remnants of a crust rich in iron has been noticed and the area is important thanks to the presence of specific hydrogeological features as well. Other cultural features are represented by historical border stones dating back to 16th Century. Moreover, the site has high relevance concerning the living nature (natural oak-beech forests, occurrence of protected species). The Holedná Hill also represents a traditional tourist and recreational background for Brno citizens thanks to the presence of forests and dense tourist network including basic tourist infrastructure. The new lookout tower complements the tourist attractiveness of the study area. These aspects represent a significant potential for the development of geotourism and environmental education. Based on the evaluation within the geomorphosite concept, proposals for further use are designed including those related to the legal protection of the natural and cultural heritage.

}
%EndOfAbstractContent
%references
{Kubalíková L, Kirchner K, Kuda F (2021) New opportunities for geotourism development at geoarcheological site Holedná Hill (Brno, Czech Republic). In Public recreation and landscape protection - with sense hand in hand! Conference proceedings, Mendelova univerzita v Brně, pp 312-316

Kirchner K, Unger J, Velek J, Kuda F (2019) Nálezová zpráva geomorfologického průzkumu – Jundrov, Holedná - Zjišťovací výzkum. Výzkumná zpráva. ÚGN AV ČR, 15 p.

Kirchner K, Unger J, Velek J, Kuda F, Kubalíková L (2019) Lokalita Holedná – hradisko z mladší doby bronzové v západní části Brna zjištěné geomorfologickým průzkumem. In Kleprlíková L, Plichta A, Turek T (eds) Sborník abstraktů. Konference 25. Kvartér, p 36, Masarykova univerzita Brno. ISBN 978-80-210-9470-3

Kirchner K, Unger J (2020) Brno (k. ú. Jundrov, okr. Brno-město). Holedná. Mladší doba bronzová. Ohrazená lokalita. Geomorfologický průzkum. Zprávy o výzkumech za rok 2019: Doba bronzová - Přehled výzkumů 61/1, 2020, pp 168-170.
}
%EndOfReferences
%end of conference contribution

%---------------------------------------------------------------------------------------------------

