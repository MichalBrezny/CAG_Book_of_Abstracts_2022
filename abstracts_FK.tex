%start of conference contribution
\abstract
% Title 
{Development and recent dynamics of rockslide in the Hercynian orogeny (the Rudohorský potok basin; the Hrubý Jeseník Mts.)} 
% EndOfTitle
%short author -- toc 
{Fabiánová A.} 
%End short author -- toc
% Author(s) 
{Andrea Fabiánová\textsuperscript{1}*} 
% EndOfAuthor(s)
{\KLtag} 
%Tag, can be: empty, \KLtag (keynote lecture), \IStag (invited speaker), \CTtag (contributed talk) or \ITtag (invited talk)
% Affiliation(s)
{
\textsuperscript{1}Faculty of Science, University of Ostrava, Ostrava, Czech Republic
}
%} % EndOfAffiliation(s)
{andrea.fabianova@osu.cz}  %e-mail
%keywords
{Rockslide, dendrogeomorphology, structural compass measurements, kinematic analysis, Schmidt-hammer, the Hrubý Jeseník Mts., I\textsubscript{t} index.}
%EndOfKeywords
%abstract content
{A rockslide in the Hrubý Jeseník mountains was studied to unravel its development and recent activity. Four methods were applied: dendrogeomorphologic dating, rock hardness measurement using Schmidt-hammer, structural compass measurement, and kinematic analysis. For the purposes of dendrogeomorphologic dating, reaction wood, eccentric growth and sudden growth suppression of 43 Norway spruces (\textit{Picea abies} (L.) Karst.) were used. The Schmidt-hammer was used to measure hardness of rocks of rock walls and colluvial accumulations. Structural compass measurements were used to reveal the strike and dip direction of joints and foliation planes of the landslide slope. Kinematic analysis was performed to reveal possible slope predisposition to planar failures, wedge failures and toppling. According to the results, the rockslide is in its dormant state. Despite its dormancy, movements of shallow-bedded blocks were recorded as well as probable movements of a deeper character of a larger landslide unit. Movements are indistinct, probably within mm to cm. The landslide blocks move by sliding in the direction of the foliation inclination, toppling of the blocks is present as well, and rotational character of rock block movement is also probable. Overall, the surface of the landslide body is considerably weathered evenly over its entire surface. It results in rock walls copying tectonic faults, and numerous rock accumulations originating from the rockfalls, covering the rest of the slope surface and accumulating at its base. In case of dendrogeomorphology, it was the first use of this dating method on almost bare rock surfaces. It turned out that it is possible to detect movements of rock blocks, but at least two trees per block are needed alongside with evaluation of their spatio-temporality, whether the trees react logically on the landslide blocks/units. Moreover, it is suitable to use a three-year period during which the trees on the blocks could react. The result of this work offers a unique view of the only landslide examined so far in the Hrubý Jeseník mountains and tests/complements the methodological procedures for dendrogeomorphologic dating of these specific slope deformations.
}
%EndOfAbstractContent
%references
{
}
%EndOfReferences
%end of conference contribution

%---------------------------------------------------------------------------------------------------
\abstract
{Verification of the "Čirá - Kopanina" fault zone using morphostructural analysis and geophysical methods} % Title
{Findžová et al.} %short author -- toc
{Leona Findžová$^1$, Petr Tábořík$^1,2$, Jakub Stemberk$^2$, Petra Štěpančíková$^2$} % Author(s)
{\KLtag} % Tag, can be: empty, \KLtag (keynote lecture), \IStag (invited speaker), \CTtag (contributed talk) or \ITtag (invited talk)
{$^1$Charles University, Faculty of Sciences, Prague\\
	$^2$) Czech Academy of Sciences, Institute of Rock Structure and Mechanics, Prague
} % Affiliation(s)
{}  %e-mail
{Krušné hory Mts, fault zone, active tectonics, digital elevation model, morphostructural analysis, geophysical survey, electrical resistivity tomography}%keywords
{The aim of the research is to verify the existence of the "Čirá - Kopanina" fault zone in the western part of the Jindřichovice Highlands (Krušné hory Mts), which appears to be a potential source area for earthquake swarms at the boundary of the Cheb Basin and the Krušné hory crystalline complex. It is a possibly seismically active (seismogenic) structure and therefore a potentially active tectonic area. At the same time, it is a structure that is partly manifested on the surface, i.e. in the morphology of the terrain. The work builds on research already carried out in the area and aims to validate the hypothesis of a relatively less known fault structure that could predispose tectonic activity in the area. Thus, the research is focused on (i) characterizing the manifestations of the studied tectonic structure in the landscape by means of morphometric and morphostructural analyses of the area and (ii) subsequent verification of its occurrence directly in the field using geophysical methods. Morphometric and morphostructural analyses of the detailed digital elevation model (DMR 5G) were chosen as the basic methods of investigation, on the basis of which survey sites were selected for subsequent verification of the course of the searched fault zone using applied geophysical methods. Electrical resistivity tomography was selected as the primary method of geophysical survey as it is commonly used to verify the fault zones course. On the selected profile, the geoelectrical survey will also be complemented with seismic and gravity measurements. Preliminary results of the DEM analyses and initial geophysical measurements suggest that the investigated fault could indeed predispose the morphotectonic evolution of the area.
}%abstract
{}%references
%end of abstract


\abstract
{Transport, Retention and Geomorphic Impact of Large Wood in a Meandering River
} % Title
{Galia et al.} %short author -- toc
{Tomáš Galia$^1$*, Václav Škarpich$^1$, Matěj Horáček$^1$, Virginia Ruiz-Villanueva$^2$} % Author(s)
{\KLtag} % Tag, can be: empty, \KLtag (keynote lecture), \IStag (invited speaker), \CTtag (contributed talk) or \ITtag (invited talk)
{$^1$Faculty of Science, University of Ostrava, Ostrava, Czech Republic\\
	$^2$Institute of Earth Surface Dynamics., University of Lausanne, Switzerland
} % Affiliation(s)
{tomas.galia@osu.cz}  %e-mail
{meandering river, large wood, river morphodynamics, Odra}%keywords
{Large wood (LW) is an integral part of rivers that supports habitat heterogeneity and biodiversity by its impact on flow hydraulics, channel morphodynamics and sediment transport (Gurnell et al., 2002). The majority of the research related to the geomorphic impact of LW was realised in wadeable channels, and we lack complex knowledge of the processes related to LW from large meandering rivers wider than the height of riparian trees (Wohl, 2017). This contribution presents the predictors of the interannual variability of the LW and its mobility during the 2016-2021 period in a 3.65 km long active meandering reach of Odra (Oder), Czechia (Galia et al., in review). We employed a repeated complete LW inventory and monitoring of tagged LW pieces. We found interannual variations in LW volumes (8.3-9.2 m3/ha), which did not allow us to develop any robust single model predicting LW volumes at the scale of meander bends (n = 14). Characteristics derived from riparian stands were found as important predictors of LW volumes. However, both positive and negative correlations between these characteristics and LW volumes were observed, which likely points to the complex recruitment-retention role of riparian stands. The dimensions of LW (i.e., length and diameter) together with the initial anchorage of the LW (i.e., its partial burial in sediments or racking by living trees or other LW) were determined as predictors of the LW mobility. The ratio between the LW channel width and the LW length was of greater importance for narrower channel segments, which points on some degree of equimobility of LW in the widest channel sections independently on the LW length. We also observed that local channel morphodynamics can be driven by the presence of single but stable LW. On the other hand, the noticeable spatiotemporal variation in jams between 2016 and 2021 was not only a product of the (suggested) frequent transport of relatively short LW pieces and related destruction of jams but also of the burial of jams in fine sediments during the process of channel migration. These field observations demonstrate the complex relationship between the channel morphodynamics and the biogeomorphic impact of vegetation in meandering rivers.
}%abstract

{Galia T, Horáček M, Ruiz-Villanueva V, Škarpich V (in review) Large wood retention and mobility in a large meandering river: insights from a 5-year monitoring in the Odra River (Czechia). Geomorphology.
	
	Gurnell, AM, Piégay H, Swanson FJ, Gregory SV (2002) Large wood and fluvial processes. Freshwater Biology 47: 601–619.
	
	Wohl E (2017) Bridging the gaps: An overview of wood across time and space in diverse rivers. Geomorphology 279: 3–26.
}%references
%end of abstract

%start of conference contribution
\abstract
% Title 
{Spatiotemporal Dynamics of Gravel Bars and Vegetation Cover of Two Carpathian Rivers} 
% EndOfTitle
%short author -- toc 
{Holušová and Galia} 
%End short author -- toc
% Author(s) 
{Adriana Holušová\textsuperscript{1}, Tomáš Galia1)\textsuperscript{1}} 
% EndOfAuthor(s)
{\KLtag} 
%Tag, can be: empty, \KLtag (keynote lecture), \IStag (invited speaker), \CTtag (contributed talk) or \ITtag (invited talk)
% Affiliation(s)
{
\textsuperscript{1}Faculty of Science, University of Ostrava, Ostrava, Czech Republic
}
%} % EndOfAffiliation(s)
{adriana.holusova@osu.cz}  %e-mail
%keywords
{river, gravel bar, vegetation cover, channel heterogeneity}
%EndOfKeywords
%abstract content
{Carpathian gravel-bed rivers are characterized by the abundance of gravels and when naturally behaved, they form a braided river system along with the formation of gravel bars. Most of the Carpathian gravel-bed rivers were regulated or managed in a way that changed the course of the river to a single-threaded channel with weirs and, in some cases, they were dammed by valley dams (Škarpich et al., 2020). These changes affect the sediment regime (sediment deficits, limited discharges) and thus the formation and behaviour of the gravel bars. Examples are including weir induced occurrence of gravel bars, overgrown with vegetation and a general decrease of new gravel bars (Hajdukiewicz and Wyżga, 2019).

The aim of our research is to evaluate the spatial and temporal changes of the gravel bars in two Czech Carpathian rivers: the Olše (73 km) and the Ostravice (downstream of the valley dam, 46 km) in 20 years (2000–2020), with a focus on the progression of the succession of vegetation cover compared to hydrological events, and the possible influence of the heterogeneity of the river channel on river bar occurrence including sinuosity, relative channel width, weirs, and tributaries. We used 9 orthophotos, as well as hydrological data from the studied period, and ArcGIS Pro software for mapping and spatial analysis. 

In the study period, the reference year of 2000 was represented by a relatively high percentage of the total unvegetated bar area in both rivers (the Olše = 58\%, the Ostravice = 46\%), which shows the probable influence of major floods in 1997 and another flood event in 1999. Vegetation cover progressively increased from 2003 to 2009, when it reached 81,8\% in the Olše and 92\% in the Ostravice river. Subsequently, in year 2012, vegetation cover decreased to 62,2\% in the Olše and 68\% in the Ostravice, as a result of major floods in 2010. Vegetation continued to decrease in 2014 (the Olše = 53,2\%, the Ostravice = 54,3\%) due to another flood event that occurred shortly before the aerial images were taken. This year we also observed the highest occurrence of unvegetated gravel bars (the Olše = 13,3\%, the Ostravice = 14,1\%). The following years showed a significant increase in total vegetation cover in the bars (the Olše = 87,1--93\%, the Ostravice = 84,5--96,4\%) and the occurrence of completely vegetated bars, particularly in the Ostravice which is regulated by a valley dam (the Olše = 23,5--36,1\%, the Ostravice = 26,8--58,3\%). From the channel heterogeneity analysis, only the relative width of the channel and, in some cases weirs and tributary locations could be associated with fluctuations in the occurrence of the gravel bar throughout the studied period. The sinuosity index per km of the river showed only a few fluctuations in the case of the Olše river and very stable values in the case of Ostravice. Generally, the areas with the highest average bar areas were the locations of relatively wider channels, not dammed by weirs or other grade control structures, and in some cases, there was evident support by sediment flux from a tributary.

Our findings suggest the importance of major floods in the dynamics of the gravel bar vegetation, as well as significant progress in the vegetation cover on bars over the past few years. Although the gravel bars are associated with inducing occurrence in the vicinity of weirs, the locations with the largest gravel bars are associated with undammed, wide-channelled areas or tributary locations.
}
%EndOfAbstractContent
%references
{Hajdukiewicz, H., Wyżga, B. (2019): Aerial photo-based analysis of the hydromorphological changes of a mountain river over the last six decades: The Czarny Dunajec, Polish Carpathians. Science of the Total Environment, 648: 1598-1613.

Škarpich, V., Macurová, T., Galia, T., Ruman, S., Hradecký, J. (2020): Degradation of multi-thread gravel-bed rivers in medium-high mountain settings: Quantitave analysis and possible solutions. Ecological Engineering, 148:105795.

}
%EndOfReferences
%end of conference contribution

%---------------------------------------------------------------------------------------------------
\abstract
{Geomorphological Approach to Identification of Flood Hazard Hotspots Within Marginalized Roma Communities in Slovakia} % Title
{Jančovič and Kidová} %short author -- toc
{Marián Jančovič\textsuperscript{1}*, Anna Kidová\textsuperscript{1}} % Author(s)
{\KLtag} % Tag, can be: empty, \KLtag (keynote lecture), \IStag (invited speaker), \CTtag (contributed talk) or \ITtag (invited talk)
{\textsuperscript{1}Institute of Geography, Slovak Academy of Sciences, Bratislava, Slovak Republic
} % Affiliation(s)
{geogjanc@savba.sk}  %e-mail
{flood hazard, height above nearest drainage, downslope distance to stream, Roma, segregation, Slovakia}%keywords
{Climate change has become one of the most acute problems of human society in recent decades. One of its manifestations is the worldwide increase in the intensity and frequency of extreme hydrological events, such as floods or droughts, which pose a direct threat to affected communities. In Slovakia, a significant number of settlements lie close to rivers and are therefore potentially exposed to flood hazard. Their local character also determines the uneven distribution of the threat they pose to individual communities. In addition, the unevenness of flood risk is reinforced by the different resilience and coping capacities of different social groups. Processes such as spatial segregation can then affect an excluded community twice. On the one hand, by pushing them into otherwise uninhabited floodplains, which are more susceptible to flooding and, on the other hand, by increasing their flood vulnerability. A prime example of spatial segregation in Slovakia are marginalised Roma communities (Rochovská a Rusnáková 2018). While the aspect of their increased vulnerability has been addressed in several studies (Filčák 2012; Harper, Steger, a Filčák 2009), the natural hazards that their environment poses to them has not been addressed so far. This presentation therefore focuses on the assessment of the flood hazard in those communities, based on their distances to nearest drainage, namely height above the nearest drainage (Rennó et al. 2008) and downslope distance to stream. Since the extent of the inudation zone is flow-dependent, the interpretation of the same distance differs for e.g., a mountain stream and a river of continental significance. We therefore attempted to minimize these differences by classification of streams based on flow accumulation. The geographic representation of marginalized Roma communities in the centroid form was derived from the buildings layer of the ZB GIS. In this way, we were able to automatically extract 120 out of 697 concentrations of segregated Roma population mentioned in the Atlas rómskych komunít (2019), which contains information only at the level of the cadastral territory of the municipality. By comparing this representation with the layer of flood occurrence in Slovak towns and villages between 1996-2019, we were able to identify hotspots of flood hazard within marginalised Roma communities, which will be addressed in future research.
	
	This research was supported by the Science Grant Agency (VEGA) of the Ministry of Education of the Slovak Republic and the Slovak Academy of Sciences (02/0086/21).
}%abstract
{Filčák, Richard. 2012. “Environmental Justice and the Roma Settlements of Eastern Slovakia: Entitlements, Land and the Environmental Risks*”. Sociologicky Casopis 48 (3): 737–62. \doi{10.13060/00380288.2012.48.3.07}.
	
	Harper, Krista, Tamara Steger, a Richard Filčák. 2009. “Environmental Justice and Roma Communities in Central and Eastern Europe”. Environmental Policy and Governance 19 (4): 251–68. \doi{10.1002/eet.511}.
	
	Atlas rómskych komunít. 2019. Ministerstvo vnútra SR - Rómske komunity. \url{https://www.minv.sk/?atlas-romskych-komunit-2019}.
	
	Rennó, Camilo Daleles, Antonio Donato Nobre, Luz Adriana Cuartas, João Vianei Soares, Martin G. Hodnett, Javier Tomasella, a Maarten J. Waterloo. 2008. “HAND, a New Terrain Descriptor Using SRTM-DEM: Mapping Terra-Firme Rainforest Environments in Amazonia”. Remote Sensing of Environment 112 (9): 3469–81. \doi{10.1016/j.rse.2008.03.018}.
	
	Rochovská, Alena, a Jurina Rusnáková. 2018. “Poverty, Segregation and Social Exclusion of Roma Communities in Slovakia”. Bulletin of Geography. Socio-Economic Series 42 (42): 195–212. \doi{10.2478/bog-2018-0039}.
}%references
%end of abstract
%start of conference contribution
\abstract
% Title 
{Local Geomorphological Factors Influencing Flood Hazard, Lower Tisza, Hungary} 
% EndOfTitle
%short author -- toc 
{Kiss and Fehérváry} 
%End short author -- toc
% Author(s) 
{Tímea Kiss\textsuperscript{1}*, István Fehérváry\textsuperscript{1,2}} 
% EndOfAuthor(s)
{\KLtag} 
%Tag, can be: empty, \KLtag (keynote lecture), \IStag (invited speaker), \CTtag (contributed talk) or \ITtag (invited talk)
% Affiliation(s)
{
	\textsuperscript{1}Department of Geoinformatics, Physical and Environmental Geography, University of Szeged, Szeged, 6722, Hungary, Egyetem u. 2–6
	\textsuperscript{2}Directorate for Environmental Protection and Water Management of Lower Tisza District,
	Szeged, 6722, Hungary, Stefánia 4.
}
%} % EndOfAffiliation(s)
{kisstimi@gmail.com}  %e-mail
%keywords
{flood hazard, local factors, aggradation, channel narrowing, riparian vegetation, levee lowering}
%EndOfKeywords
%abstract content
{Besides catchment-scale processes, various local processes influence the peak flow level, however, they are often neglected in flood management. The aim of our study is to analyse those local geomorphological processes that contribute to rising local flood levels in the regulated channel and on the artificially confined floodplain of the Tisza River, Hungary. Our goals were to evaluate the role of (1) cross-sectional channel changes, (2) overbank floodplain aggradation, (3) riparian vegetation changes on local flood level increases since the late 19th century and early 20th century; and (4) to assess the elevation changes of artificial levees. 
	
Along the Lower Tisza (92 km) since 1931 the channel narrowed by 9\% (max. 30\%) and its cross-sectional area decreased by an average of 2\% (max. 22\%). These in-channel processes increased flood levels by an average of 13 cm (max. 134 cm). Simultaneously, the flood conveyance capacity of the floodplain decreased by overbank floodplain accumulation, which is 1.1 m in average (max. 2.6 m). The accelerated aggradation further increased the flood levels by 112 cm. The land-use of the floodplain also changed considerably (from meadows to forests). As the vegetation roughness (Manning’s n) increased from 0.048 to 0.11, it increased flood levels by 42 cm (Scenario A roughness increase: 10\%) or 139 cm (Scenario B roughness increase: 30\%) on average.

By overlapping these data, the results showed that since the 19th century river regulation works the actual flood level increased by an average of 175 cm (maximum = 350 cm) in the case of Scenario A and 272 cm (maximum = 443 cm) in the case of Scenario B. The latter is more consistent with the actual flood stage measurements. 

As these processes are still active, further increase in the flood level could be expected. In addition, the height of artificial levees decreased by an average of 23 cm (max 75 cm). Therefore, some levee sections became more susceptible to overtopping during record high floods, especially along the eastern levee. 

Based on this approach, local hydrological managers can identify the processes that contribute more to peak flow level increase at a given location, and determine the correct management actions at the correct locations, which could lead to decrease in local peak flow levels.
}
%EndOfAbstractContent
%references
{
}
%EndOfReferences
%end of conference contribution

%---------------------------------------------------------------------------------------------------

%start of conference contribution
\abstract
% Title 
{The Impact of the River Training on the Multi-Thread River System as a Part of the Natura 2000 Protected Area} 
% EndOfTitle
%short author -- toc 
{Kidová et al.} 
%End short author -- toc
% Author(s) 
{Anna Kidová\textsuperscript{1}*, Artur Radecki-Pawlik\textsuperscript{2}, Miloš Rusnák\textsuperscript{1}, Karol Plesiński\textsuperscript{3}} 
% EndOfAuthor(s)
{\KLtag} 
%Tag, can be: empty, \KLtag (keynote lecture), \IStag (invited speaker), \CTtag (contributed talk) or \ITtag (invited talk)
% Affiliation(s)
{
	\textsuperscript{1}Institute of Geography, Slovak Academy of Sciences, Bratislava, Slovak Republic\\
	\textsuperscript{2}Faculty of Civil Engineering, Cracow University of Technology, Krakow, Poland\\
	\textsuperscript{3}Faculty of Environmental Engineering and Land Surveying, University of Agriculture in Krakow, Krakow, Poland
}
%} % EndOfAffiliation(s)
{geogkido@savba.sk}  %e-mail
%keywords
{river management, morphology, hydraulics, multi-thread river, river training, Belá River}
%EndOfKeywords
%abstract content
{The results of the river training impact investigation deals with the evaluation of the morphology and hydraulic parameters of the Belá River belonging to the Natura 2000 protected area in Slovak Carpathians (Kidová et al. 2021). The most serious consequences of the river training related to the simplification of the river planform, the reduction of the geomorphic diversity (Kidová et al. 2016) of the in-channel landforms, as well as the reduction and loss of lateral connectivity (Lehotský et al. 2018) between the low flow channel and floodplain. Analysed hydraulic parameters before and after river training revealed an increased flow rate, increased values of shear stress, as well as water energy on all studied cross-sections. Identifying the main conflicts in the management of the Belá River by using an objective scientific approach contributed to revealing the negative consequences of the current engineering approach of stakeholders in the studied area. On the basis of these results, the authors of the study were officially asked to provide the results of this research for the needs of the Žilina Environmental Inspectorate within the administrative tort proceedings pursuant to § 90 para. 1 letter a) of Act no. 543/2002 Coll. in the matter of a change in the state of the watercourse, which should have occurred during the implementation of river training after the flood of 19 July 2018 in selected river reaches of the Belá River without the consent of the competent nature protection authority. The presented analyses could help in future management issues as well as in the more critical decision-making process in vulnerable and scarce braided river systems in the present when we are losing so many natural rivers by human decisions.

\noindent
\textbf{Acknowledgement:}\textit{
This research was supported by the Science Grant Agency (VEGA) of the Ministry of Education of the Slovak Republic and the Slovak Academy of Sciences (02/0086/21).}
}
%EndOfAbstractContent
%references
{Kidová A., Lehotský M., Rusnák M. (2016) Geomorphic diversity in the braided-wandering Belá River, Slovak Carpathians, as a response to flood variability and environmental changes. Geomorphology 272:137–149.\doi{10.1016/j.geomorph.2016.01.002}

Kidová, A., Radecki-Pawlik, A., Rusnák, M. et al. (2021) Hydromorphological evaluation of the river training impact on a multi-thread river system (Belá River, Carpathians, Slovakia). Scientific Reports 11, 6289. \doi{10.1038/s41598-021-85805-2}

Lehotský, M., Rusnák, M., Kidová, A., Dudžák, J. (2018) Multitemporal assessment of coarse sediment connectivity along a braided-wandering river. In Land Degradation \& Development, 2018, vol. 29, no. 4, p. 1249-1261. \doi{10.1002/ldr.2870}
}
%EndOfReferences
%end of conference contribution

%---------------------------------------------------------------------------------------------------

%start of conference contribution
\abstract
% Title 
{Activity of landslide processes in the area of Hřebečský hřbet} 
% EndOfTitle
%short author -- toc 
{Kozák and Šilhán} 
%End short author -- toc
% Author(s) 
{Michal Kozák\textsuperscript{1}*, Karel Šilhán\textsuperscript{1}} 
% EndOfAuthor(s)
{\POtag} 
%Tag, can be: empty, \KLtag (keynote lecture), \IStag (invited speaker), \CTtag (contributed talk) or \ITtag (invited talk)
% Affiliation(s)
{
\textsuperscript{1}Faculty of Science, University of Hradec Králové, Czech Republic
}
%} % EndOfAffiliation(s)
{kozakmi1@uhk.cz}  %e-mail
%keywords
{Hřebečský hřbet, dendrogeomorphology, landslides, cuest}
%EndOfKeywords
%abstract content
{Slope movements are currently the main geomorphological factor and, in terms of their sudden and unpredictable activities, can cause not only material damage. Therefore, it is necessary to monitor and document these events. The activity of slope movements is historically known in the area of the Hřebečský hřbet, several recent landslide events even threatened the Hřebeč tunnel and a busy road. Thus, the study may also have a practical impact on the determination of landslide hazard and risk. Hřebečský hřbet is a typical example of cuest, thanks to which there are different manifestations of landslide activity on different slopes within one locality. Despite the same precipitation conditions, the geological structure and slope of the layers play a key role in the nature of the events. Based on dendrochronological methods were growth disturbances found, which can  be read from the core of the annual ring of selected trees. The definition of mechanism of slope movement and the structure of the geological subsoil on differently oriented slopes can be based on the character of the events (duration, intensity, spatial extent, ratio of reaction wood). From the first slope with an orientation to the SE were taken 100 samples and confirmed 16 events in the last 60 years. The character of the landslides is mostly shallow and plastic on the layered faces. The M-W test confirmed that in landslide years there was an above-average amount of precipitation in 3-day and monthly intervals compared to the years without a recorded event, and the test did not confirm this in 1-day and 5-day intervals. The M-W test confirmed that in landslide years there was an above-average amount of precipitation in 3-day and monthly intervals compared to the years without a recorded event, and the test did not confirm this in 1-day and 5-day intervals. On the second slope with NW orientation were taken 74 samples and confirmed 4 events in the last 40 years. The character of the landslide and the spatial behavior is, in contrast to the first slope, deep and block in the middle layers. Different depth and internal structure were verified by ERT method. On this slope, the M-W test did not confirm any extreme precipitation fluctuations at the examined intervals. Although the detected events correspond to the general assumption of a specific type of landslide on differently oriented slopes, it is likely that some of the dated events are the result of the creeping movements themselves, which are also corresponding to the nature of the slope. The data obtained from this study can potentially help predict processes regarding landslide behavior in the past. This information can be compared with a number of values of extreme hydrometeorological or seismic indicators and  established a most accurate relationship for them, which can be approximated in the future.
}
%EndOfAbstractContent
%references
{Lang A, Moya J, Corominas J, Dikau R (1999) Classic and new dating methods for assessing the temporal occurrence of mass movements. Geomorphology 30, 33–52.

Šilhán K (2020) Dendrogeomorphology of landslides: principles, results and perspectives. Landslides 17: 2421–2441. 

Stoffel M, Bollschweiler M (2008) Tree-ring analysis in natural hazards research - an overview. Natural Hazards and Earth System Sciences 8: 187–202.
}
%EndOfReferences
%end of conference contribution

%---------------------------------------------------------------------------------------------------


%start of conference contribution
\abstract
% Title 
{3D Rockfall Modeling, Natural Hazard and Risk Assessment for Rockfall Protection in Hřensko (Czechia)} 
% EndOfTitle
%short author -- toc 
{Kusák M.} 
%End short author -- toc
% Author(s) 
{Michal Kusák\textsuperscript{1}*} 
% EndOfAuthor(s)
{\POtag} 
%Tag, can be: empty, \KLtag (keynote lecture), \IStag (invited speaker), \CTtag (contributed talk) or \ITtag (invited talk)
% Affiliation(s)
{
	\textsuperscript{1}The Institute of Rock Structure and Mechanics of the Czech Academy of Sciences
}
%} % EndOfAffiliation(s)
{kusak@irsm.cas.cz}  %e-mail
%keywords
{Rockfall, 3D modeling, Hazard, Risk, HY-STONE, Rockfall assesment, Dynamic Barriers, Hřensko (Czechia)}
%EndOfKeywords
%abstract content
{The Hřensko area (Czechia) is characterised by sandstone landscape with rock plateaus, deep canyons with several levels of steep cliffs, which forms favourable conditions for rockfalls. Šafránek (2016) describes several dozen rockfall events with volume \textasciitilde1~m\textsuperscript{3} per year here. Therefore, the rockfall barriers were installed in Hřensko area in 2015. These barriers are 4 m high and their total length is 2.5 km. However, many other rockfall events are sitruated in ares without rockfall monitoring and the damages of buildings and infrastructure are quite probable. 

In these cases, the numerical simulations of rockfall trajectories are a standard procedure for evaluating rockfall hazards. For these simulations, corresponding software codes must be calibrated and evaluated based on field data. This study addresses 3D modelling of repeatable rockfall barrier tests, and risk determination in other readings of Hřensko.

The aim of this paper is to provide a knowledge base for researchers and practitioners involved in projects dealing with rockfall protection. And the results of rockfall barriers test should be considered by design engineers and could potentially lead to future improvements in the design of rockfall protection.
}
%EndOfAbstractContent
%references
{Kusák M, Valagussa A, Frattini P (2019) Key Issues in 3D Rockfall Modeling, Natural Hazard and Risk Assessment for Rockfall Protection in Hřensko (Czechia). Acta Geodynamica et Geomaterialia 16, 4(196): 393–408.

Šafránek J (2016) Monitoring svahových pohybů v NP České Švýcarsko. Ochrana přírody 1: 18–22. (Czech).
}
%EndOfReferences
%end of conference contribution

%---------------------------------------------------------------------------------------------------



















%start of conference contribution
\abstract
% Title 
{Linking Geomorphological and Archaeological Research at Geoarcheological Site Holedná Hill in Brno, Czech Republic} 
% EndOfTitle
%short author -- toc 
{Kubalíková et al.} 
%End short author -- toc
% Author(s) 
{Lucie Kubalíková*, Karel Kirchner, František Kuda} 
% EndOfAuthor(s)
{\KLtag} 
%Tag, can be: empty, \KLtag (keynote lecture), \IStag (invited speaker), \CTtag (contributed talk) or \ITtag (invited talk)
% Affiliation(s)
{Institute of Geonics of the Czech Academy of Sciences, Drobného 28, 602 00 Brno, Czech Republic
}
%} % EndOfAffiliation(s)
{lucie.kubalikova@ugn.cas.cz}  %e-mail
%keywords
{geoarcheology, geoheritage, cultural heritage, Bronze Age, environmental education, geotourism}
%EndOfKeywords
%abstract content
{The Holedná Hill (Brno, Czech Republic) can be considered an example of geocultural (or more precisely geoarcheological) site. It is important from the geomorphological point of view and it includes specific cultural and archaeological issues that are closely related to the geodiversity (especially anthropogenic landforms). The site is situated in the north-western part of the Brno City and represents the end of ridge above the deep incised Svratka Valley. Based on the analysis of LiDAR Data, unknown elongated structures were identified. The fieldworks proved the anthropogenic origin of these landforms (embankments composed of stones and small boulders, sometimes with shallow ditches along them). The structures symmetrically surround the dominant peak of the Holedná Hill. Consequently, an archaeological research was conducted. Based on the occurrence of Neolithic and Eneolithic localities in the vicinity, a similar age was estimated. Nevertheless, the radiocarbon dating of organic material confirms the age between 1200 -- 1050 BC which refers to the Bronze Age. Until now, only the accidental occurrence of ceramics was found. The question is what the purpose of the structures on the Holedná Hill was: the hypothesis about the sacred place or abandoned unfinished fortress was formulated.

The Holedná Hill is also important from the geological point of view: the bedrock consists of Proterozoic rocks of the Brno massif (diorites with veins of metaryoliths), the occurrence of remnants of a crust rich in iron has been noticed and the area is important thanks to the presence of specific hydrogeological features as well. Other cultural features are represented by historical border stones dating back to 16th Century. Moreover, the site has high relevance concerning the living nature (natural oak-beech forests, occurrence of protected species). The Holedná Hill also represents a traditional tourist and recreational background for Brno citizens thanks to the presence of forests and dense tourist network including basic tourist infrastructure. The new lookout tower complements the tourist attractiveness of the study area. These aspects represent a significant potential for the development of geotourism and environmental education. Based on the evaluation within the geomorphosite concept, proposals for further use are designed including those related to the legal protection of the natural and cultural heritage.
}
%EndOfAbstractContent
%references
{Kubalíková L, Kirchner K, Kuda F (2021) New opportunities for geotourism development at geoarcheological site Holedná Hill (Brno, Czech Republic). In Public recreation and landscape protection - with sense hand in hand! Conference proceedings, Mendelova univerzita v Brně, pp 312-316

Kirchner K, Unger J, Velek J, Kuda F (2019) Nálezová zpráva geomorfologického průzkumu – Jundrov, Holedná - Zjišťovací výzkum. Výzkumná zpráva. ÚGN AV ČR, 15 p.

Kirchner K, Unger J, Velek J, Kuda F, Kubalíková L (2019) Lokalita Holedná – hradisko z mladší doby bronzové v západní části Brna zjištěné geomorfologickým průzkumem. In Kleprlíková L, Plichta A, Turek T (eds) Sborník abstraktů. Konference 25. Kvartér, p 36, Masarykova univerzita Brno. ISBN 978-80-210-9470-3

Kirchner K, Unger J (2020) Brno (k. ú. Jundrov, okr. Brno-město). Holedná. Mladší doba bronzová. Ohrazená lokalita. Geomorfologický průzkum. Zprávy o výzkumech za rok 2019: Doba bronzová - Přehled výzkumů 61/1, 2020, pp 168-170.
}
%EndOfReferences
%end of conference contribution

%---------------------------------------------------------------------------------------------------

